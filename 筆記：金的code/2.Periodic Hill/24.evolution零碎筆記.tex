\documentclass[12pt]{article}
\usepackage[a4paper,margin=2cm]{geometry} % 明確設定四邊
\usepackage{fontspec} % 字體設定
\usepackage{xeCJK}
\usepackage{setspace} % 設定行距
\linespread{1.2}
\usepackage{titling} % 預設標題下移0.6in
\usepackage{enumitem}
\usepackage{amsmath} % 數學方程式
\usepackage{graphicx} %圖片
\usepackage{float} % 在導言區,讓圖片強制插在原地
\usepackage{xcolor} %字體加入顏色
\usepackage{listings}
\usepackage{physics} % 物理符號
\usepackage{wrapfig} % 文字環繞圖
\usepackage{array} % 表格對齊控制
\usepackage{caption}
\usepackage{amsthm}
\usepackage{tabularx}
\usepackage{booktabs}
\usepackage{multirow}
\usepackage{etoolbox}
\usepackage{titlesec}
\usepackage{tocloft}
\usepackage{hyperref}
\usepackage{makecell}
\usepackage{xcolor}
\usepackage{colortbl}
\hypersetup{
  colorlinks=true,
  linkcolor=blue,
  urlcolor=blue,
  citecolor=blue,
  linktoc=all
}
\usepackage{tcolorbox}
\tcbuselibrary{skins,breakable}
\usepackage{tikz}
\usetikzlibrary{arrows.meta,positioning,shapes.geometric}
\tcbset{
  highlightblock/.style={
    enhanced,
    breakable,
    colback=gray!8,
    colframe=gray!60,
    boxrule=0.4pt,
    arc=0pt,
    boxsep=0pt,
    left=4pt,
    right=4pt,
    top=4pt,
    bottom=4pt,
    before skip=0pt,
    after skip=0pt
  }
}
\usepackage{longtable}

%1 註記行距設定
\setlist[itemize]{itemsep=1.2pt, parsep=0pt, topsep=1pt}
\setlist[enumerate]{itemsep=1.2pt, parsep=0pt, topsep=1pt}
%2 圖表標題設定
\captionsetup{
    labelfont={footnotesize,bf},    % 標籤:小字體+粗體
    skip=10pt                        % 標題與圖表的間距
}
%3 表格間距設定
\setlength{\tabcolsep}{8pt} % 列間距(增加)
\renewcommand{\arraystretch}{1.4} % 行間距(增加)
%4 腳註設定
\usepackage[bottom,hang]{footmisc} % 腳註置於頁面底部,懸掛縮排
\setlength{\footnotesep}{10pt} % 腳註之間的間距(增加)
\setlength{\skip\footins}{12pt plus 5pt minus 2pt} % 正文與腳註之間的間距(增加)
\setlength{\footnotemargin}{1.5em} % 腳註標號與文字的間距(增加)
\renewcommand{\footnoterule}{\vspace*{-3pt}\hrule width 0.4\columnwidth height 0.4pt\vspace*{3pt}} % 腳註分隔線
%5 使用直立字體的定理樣式
\newtheoremstyle{upright}
  {6pt}{6pt}  % 定理環境前後間距(增加)
  {\normalfont}% 使用正常字體,不使用斜體
  {0pt}{\bfseries}{:}{0.5em}{}
%6 定理環境定義
\theoremstyle{upright}
\newtheorem{definition}{定義}[section]
\newtheorem{theorem}{定理}[section]
\newtheorem{lemma}{引理}[section]
\newtheorem{corollary}{推論}[section]
\newtheorem{example}{例子}[section]
% 重新正確定義 remark 環境 - 確保完全靠左對齊
\makeatletter
\@ifundefined{c@remark}{}{\renewcommand{\theremark}{}}
\newenvironment{remark}{%
  \par\vspace{0.0\baselineskip}%
  \begingroup% 開始一個組以限制設置的範圍
  \setlength{\parindent}{0pt}% 設置段落縮進為零
  \setlength{\leftskip}{0pt}% 設置左邊距為零
  \noindent\textbf{註記:}\\% 添加標題並強制換行
  \ignorespaces% 忽略可能的空格
  \setlist[enumerate]{itemsep=0pt, parsep=0pt, topsep=0.0pt, leftmargin=1.5em}%
  \setlist[itemize]{itemsep=0pt, parsep=0pt, topsep=0.0pt, leftmargin=1.5em}%
}{%
  \endgroup% 結束組
  \par\vspace{0.5\baselineskip}
}
\makeatother
%7 手動定義中文數字(不含標點符號)
\newcommand{\chinese}[1]{%
  \ifcase#1 零\or 一\or 二\or 三\or 四\or 五\or 六\or 七\or 八\or 九\or 十\or
  十一\or 十二\or 十三\or 十四\or 十五\or 十六\or 十七\or 十八\or 十九\or 二十\fi
}

%8 重新定義章節編號格式
\renewcommand{\thesection}{\chinese{\value{section}}}
\renewcommand{\thesubsection}{\chinese{\value{section}}、\arabic{subsection}}
\renewcommand{\theequation}{\chinese{\value{section}}.\arabic{equation}}
\renewcommand{\thesubsubsection}{\chinese{\value{section}}、\arabic{subsection}.\arabic{subsubsection}}
\renewcommand{\thefigure}{\chinese{\value{section}}.\arabic{figure}}
\renewcommand{\thedefinition}{\thesection、\arabic{definition}}
\renewcommand{\thetheorem}{\thesection、\arabic{theorem}}
\renewcommand{\theexample}{\thesection、\arabic{example}}

% 为不同级别的标题增加编号后的空格
\setlength{\cftsecnumwidth}{2.5em}  % section 编号宽度
\setlength{\cftsubsecnumwidth}{3.5em}  % subsection 编号宽度
\setlength{\cftsubsubsecnumwidth}{4.5em}  % subsubsection 编号宽度

% 添加点线(可选)
\renewcommand{\cftsubsubsecleader}{\cftdotfill{\cftdotsep}}
\renewcommand{\cftsubsecleader}{\cftdotfill{\cftdotsep}}
\renewcommand{\cftsecleader}{\cftdotfill{\cftdotsep}}
%9 讓方程式計數器在每個section重置
\counterwithin{equation}{section}
%10 字體設定 - 優化以減少警告
\IfFontExistsTF{Times New Roman}{%
  \setmainfont{Times New Roman}%
}{%
  \setmainfont{TeX Gyre Termes}%
}

%\setCJKmainfont[
%    BoldFont={Kaiti TC Bold},
%    ItalicFont={Kaiti TC},
%    BoldItalicFont={Kaiti TC Bold}
%]{Kaiti TC}
%\setCJKmonofont{Kaiti TC}
\IfFontExistsTF{DFKai-SB}{%
  \setCJKmainfont[
    BoldFont={DFKai-SB},
    ItalicFont={DFKai-SB},
    BoldItalicFont={DFKai-SB}
  ]{DFKai-SB}%
  % 為 xeCJK 明確指定等寬字型以消除 \CJKttdefault 警告
  \setCJKmonofont{DFKai-SB}%
}{%
  \IfFontExistsTF{Kaiti TC}{%
    \setCJKmainfont[
      BoldFont={Kaiti TC Bold},
      ItalicFont={Kaiti TC},
      BoldItalicFont={Kaiti TC Bold}
    ]{Kaiti TC}%
    % 為 xeCJK 明確指定等寬字型以消除 \CJKttdefault 警告
    \setCJKmonofont{Kaiti TC}%
  }{%
    \IfFontExistsTF{Songti TC}{%
      \setCJKmainfont{Songti TC}%
      % 為 xeCJK 明確指定等寬字型以消除 \CJKttdefault 警告
      \setCJKmonofont{Songti TC}%
    }{%
      \setCJKmainfont{Heiti TC}%
      % 為 xeCJK 明確指定等寬字型以消除 \CJKttdefault 警告
      \setCJKmonofont{Heiti TC}%
    }%
  }%
}
%11 標題設定
\setlength{\droptitle}{-1in} % 上移標題1in
\title{24.evolution零碎筆記}
\author{Chen Peng Chung}
\setcounter{section}{0}
% 12调整 subsection 的间距
\titlespacing*{\section}
{0pt}                    % 左边距
{0.0in}                 % 标题前间距
{0.0em}                  % 标题后间距
\titlespacing*{\subsection}
{0pt}                    % 左边距
{-0.05in}                % 标题前间距
{0.0em}                  % 标题后间距
% 调整 subsubsection 的间距  
\titlespacing*{\subsubsection}
{0pt}                    % 左边距
{0.0in}                 % 标题前间距
{0.0em}                  % 标题后间距
%13.

% ========== Monokai Light 配色方案(淺灰背景)==========
\definecolor{monokailight-bg}{RGB}{240,240,240}      % 淺灰背景 #F0F0F0
\definecolor{monokailight-fg}{RGB}{39,40,34}         % 深色文字 #272822
\definecolor{monokailight-comment}{RGB}{117,113,94}  % 註解 #75715E
\definecolor{monokailight-string}{RGB}{152,118,24}   % 字串 #987618
\definecolor{monokailight-keyword}{RGB}{244,0,95}    % 關鍵字 #F4005F
\definecolor{monokailight-function}{RGB}{121,162,0}  % 函數 #79A200
\definecolor{monokailight-number}{RGB}{137,89,168}   % 數字 #8959A8
\definecolor{monokailight-type}{RGB}{0,129,152}      % 型別 #008198
\definecolor{monokailight-operator}{RGB}{244,0,95}   % 運算符 #F4005F
\definecolor{monokailight-border}{RGB}{210,210,210}  % 邊框 #D2D2D2

% ========== CUDA 語言定義 ==========
\lstdefinelanguage{CUDA}{
    language=C++,
    morekeywords={__global__, __device__, __host__, __shared__, 
                  __constant__, __syncthreads, dim3, cudaMalloc,
                  cudaMemcpy, cudaFree, cudaMemcpyHostToDevice,
                  cudaMemcpyDeviceToHost, cudaDeviceSynchronize},
    % 型別關鍵字
    morekeywords=[2]{double, float, int, char, void, size_t,
                     uint, uint2, uint3, uint4,
                     int2, int3, int4,
                     float2, float3, float4,
                     double2, double3, double4},
    % CUDA 內建變數
    morekeywords=[3]{blockIdx, threadIdx, blockDim, gridDim,
                     warpSize},
    sensitive=true
}

% ========== Monokai Light 樣式 ==========
\lstdefinestyle{monokailight}{
    language=CUDA,
    backgroundcolor=\color{monokailight-bg},
    basicstyle=\color{monokailight-fg}\ttfamily\footnotesize,
    commentstyle=\color{monokailight-comment}\itshape,
    keywordstyle=\color{monokailight-keyword}\bfseries,
    keywordstyle=[2]\color{monokailight-type}\bfseries,      % 型別
    keywordstyle=[3]\color{monokailight-function}\bfseries,  % CUDA 變數
    stringstyle=\color{monokailight-string},
    numberstyle=\tiny\color{monokailight-comment},
    % 數字顏色
    literate=
        {0}{{{\color{monokailight-number}0}}}1
        {1}{{{\color{monokailight-number}1}}}1
        {2}{{{\color{monokailight-number}2}}}1
        {3}{{{\color{monokailight-number}3}}}1
        {4}{{{\color{monokailight-number}4}}}1
        {5}{{{\color{monokailight-number}5}}}1
        {6}{{{\color{monokailight-number}6}}}1
        {7}{{{\color{monokailight-number}7}}}1
        {8}{{{\color{monokailight-number}8}}}1
        {9}{{{\color{monokailight-number}9}}}1,
    % 版面設定
    breaklines=true,
    breakatwhitespace=false,
    captionpos=b,
    keepspaces=true,
    numbers=left,
    numbersep=8pt,
    showspaces=false,
    showstringspaces=false,
    showtabs=false,
    tabsize=4,
    % 框線設定
    frame=single,
    rulecolor=\color{monokailight-border},
    framerule=0.8pt,
    % 其他設定
    columns=flexible,
    escapeinside={(*@}{@*)},
    xleftmargin=2em,
    xrightmargin=0.5em,
    framexleftmargin=1.5em
}

% ========== 設為預設樣式 ==========
\lstset{style=monokailight}
%14.
% ========== 表格樣式設定 ==========
\setlength{\arrayrulewidth}{0.5pt}  % 表格線條粗細
\renewcommand{\arraystretch}{1.3}   % 行高
%15listing 設定:
\lstset{
  language=C++,
  basicstyle=\ttfamily\small,
  numbers=left,
  numberstyle=\tiny,
  stepnumber=1,
  numbersep=8pt,
%
  keepspaces=true,      % ★保留空白
  showspaces=false,
  showstringspaces=false,
  showtabs=false,
%
  tabsize=2,            % ★tab 視為 2 個空白(可改 4)
  breaklines=true,
  breakatwhitespace=false,
%
  frame=single,
  captionpos=b
}



\begin{document}
\maketitle
\tableofcontents
\newpage
以下為其他程式碼的補充。
補充:
\begin{lstlisting}[language=C, caption={variables.h10-19}, label={variables.h10-19}]                                                                                                    
#define  NX   32     //x方向物理空間網格數:32
#define  NY   128    //y方向物理空間網格數:128
#define  NZ   64     //z方向物理空間網格數:64

#define  jp   4      //GPU的使用數量:4

#define  NX6 (NX+7)  
//(1)x方向物理空間節點數:NX(Link-wise)(2)x方向總節點數:NX+7(定義采用前3後4)
#define  NYD6 (NY/jp+7)//還不知道為什麼
#define  NY6 (NY+7)  
//(1)y方向物理空間節點數:NY(Link-wise)(2)y方向總節點數:NY+7(定義采用前3後4)
#define  NZ6 (NZ+6)  //(?)我覺得他寫錯了,應該是NZ+7                                                                                                                                                                   
\end{lstlisting}
\noindent 在Link-wise節點佈局中,物理空間變的儲存點個數 = 物理空間之網格數 

\noindent {\color{red}為什麽總節點數相較於物理間總節點數多6個單位?} \\
因為本程式碼為用到Lagrange 6階空間精度內插公式:                                                                                                                                                                 
\begin{equation*}                                                                                                                                                             
  f(x)=\sum_{i=0}^{6}f(x_{i})\prod_{\substack{j=0 \\ j\neq i}}^{6}\frac{x-x_{j}}{x_{i}-x_{j}}                                                                                 
\end{equation*}                                                                                                                                                                
來取值執行collision step前的分佈函數,在三維空間中,上述之內插公式則需要7*7*7個鄰近空間點(stencil)的函數值來決定:\\
\noindent 以下將列出三維空間7點Lagrange內插公式:\\
\noindent\begin{tcolorbox}[highlightblock]
Let $(x_{\hat{i}}, y_{\hat{j}}, z_{\hat{k}})$ denote a reference grid point.
Define the normalized offsets
\[
\delta x = \frac{x - x_{\hat{i}}}{\Delta x}, \quad
\delta y = \frac{y - y_{\hat{j}}}{\Delta y}, \quad
\delta z = \frac{z - z_{\hat{k}}}{\Delta z},
\]
where $\delta x, \delta y, \delta z \in 
{R}$.

The three-dimensional 7-point Lagrange interpolation in tensor-product form is written as
\begin{equation}
\label{eq:3d_lagrange_tensor}
f\!\left(
x_{\hat{i}} + \delta x \Delta x,\,
y_{\hat{j}} + \delta y \Delta y,\,
z_{\hat{k}} + \delta z \Delta z
\right)
=
\sum_{i=-3}^{3}
\sum_{j=-3}^{3}
\sum_{k=-3}^{3}
f_{\hat{i}+i,\;\hat{j}+j,\;\hat{k}+k}\,
\ell_i(\delta x)\,
\ell_j(\delta y)\,
\ell_k(\delta z),
\end{equation}
where the one-dimensional Lagrange basis functions are defined as
\begin{align}
\ell_i(\delta x) &= \prod_{\substack{m=-3 \\ m \neq i}}^{3}
\frac{\delta x - m}{i - m}, \\[6pt]
\ell_j(\delta y) &= \prod_{\substack{n=-3 \\ n \neq j}}^{3}
\frac{\delta y - n}{j - n}, \\[6pt]
\ell_k(\delta z) &= \prod_{\substack{q=-3 \\ q \neq k}}^{3}
\frac{\delta z - q}{k - q}.
\end{align}
\end{tcolorbox}
\vspace{1.5em}
\noindent 所以一邊之邊界點最多需要向外插值三個點的值來完成Lagrange七點內插。
而對於完整的一維空間內插設置,則需要多設置六個節點,這六層向外使用的節點層稱為buffer layers。
\begin{equation}
\label{eq:3d_lagrange_tensor_modified}
f\!\left(
x_{\hat{i}} + \delta x \Delta x,\,
y_{\hat{j}} + \delta y \Delta y,\,
z_{\hat{k}} + \delta z \Delta z
\right)
=
\sum_{i=-3}^{3}
\sum_{j=-3}^{3}
\sum_{k=-3}^{3}
f_{\hat{i}+i,\;\hat{j}+j,\;\hat{k}+k}\,
\ell_i(\delta x)\,
\ell_j(\delta y)\,
\ell_k(\delta z),
\end{equation}

\noindent 因此,在每一個方向上,都需要多新增3層budder layers來提供插值所需的鄰近點值。
故總節點數 = 物理空間節點數 + 6 (3層buffer layers)。又因為週期性邊條件在Stream-Wise以及Span-Wise方向,需要複製壁面來向的撞遷移過程,
所以在維持相同碰撞 pattern 的條件下,多設立一層作為實現週期性邊條件的技巧。
\newpage
\noindent 
{\color{red}Lagrange 七點內插公式在Latticeb Boltzmann Method 中的應用:}

\noindent Interpolation-Based Lattice Boltzmann Method : \\
\noindent 所以,插值,是插誰? 插【非均勻網格節點的遷移前空間點的碰撞前分佈函數】,如圖說明:
\noindent 12/29 更新: 插值 : 【(R(T(A物理空間計算點)的(遷移前空間點))的(碰撞後分佈函數))】所以整體碰撞演算法為: 插值>碰撞>遷移。先透過插值的方式得到不存在的分佈函數,再碰撞更新半步長的時間步,最後遷移得到往格中心點實際存在的碰撞遷移後分佈函數。
\begin{figure}[H]
    \centering
    \includegraphics[width=0.3\linewidth]{5.png}
    \caption{Schematic illustration of the interpolation-based Lattice Boltzmann method (LBM).}
    \caption*{Source: Y.-H. Chiu, \textit{Simulations of turbulent Poiseuille duct flow with the lattice Boltzmann method on non-uniform meshes}, 2025.}
\end{figure}
\noindent 所以插值晶格波茲曼法依據如下方程式執行:
\begin{equation}
  \label{eq:precollision_non-grid}
f_{\text{遷移前空間點(非網格節點)}}^{pre-collision(time = n)}
=
\sum_{i=-3}^{3}
\sum_{j=-3}^{3}
\sum_{k=-3}^{3}
f_{\hat{i}+i,\;\hat{j}+j,\;\hat{k}+k}\,
\ell_i(\delta x)\,
\ell_j(\delta y)\,
\ell_k(\delta z),
\end{equation}
其中,$f_{\text{遷移前空間點(非網格節點)}}^{pre-collision(time = n)}$為(D(S(A非均勻往個節點)的(遷移前空間點))的(碰撞前分佈函數))
\begin{equation}
  f_{\text{遷移前空間點(非網格節點)}}^{\star}
=
f_{\text{遷移前空間點(非網格節點)}}^{pre-collision} \\[1.5ex]
- \Omega(f^{eq}(\vec{r},t), f(\vec{r},t))
\end{equation}
\noindent 其中,$\Omega(f^{eq}(\vec{r},t), f(\vec{r},t))$為碰撞算子(Collision Operator),且上述方程式稱為LBM之碰撞步(collision step)。
\begin{equation}
  f_{\text{遷移後空間點(網格節點)}}^{n+1} = f_{\text{遷移前空間點(非網格節點)}}^{\star} \\[1.5ex]
\end{equation}
\noindent 上式則稱為LBM之遷移步(Streaming Step)。

\vspace{1.0ex}
\noindent {\color{blue}以下為程式碼補充:} 
\noindent {\color{blue}補充一:}

\begin{lstlisting}[language=C, caption={interpolationHillISLBM.h9-11}, label={interpolationHillISLBM.h9-11}]                                                                                                    
#define F0_Intrpl7(f, i, j, k)      \
    idx = j*nface + k*nline + i;    \
    F0_in = ( f[idx] );                                                                                                                                                                      
\end{lstlisting}  
\noindent  在程式碼規則中,\#define 稱為"宏定義",宏定義內部的變數 idx不需要特別宣告,只需要在使用到此宏定義的函數或者變數的區域內加以宣告即可。
"宏定義"為對特定變數的擬合器,如上片段程式碼 \ref{interpolationHillISLBM.h9-11} 所示,"F0\_Intrpl7"作為一個定義式,只要在其他函數中,一講出"F0\_Intrpl7",編譯器就會自動將其替換成定義式內的內容: 
F0\_in = ( f[idx] ) : 

\begin{table}[H]
\centering
\caption{巨集定義與分佈函數變數之對應關係}
\centering
\begin{tabular}{@{}lc@{}}
\toprule
\textbf{宏定義} & \textbf{對應變數} \\
\midrule
\texttt{\#define F0\_Intrpl7}  & $\texttt{F0\_Intrpl7} \Leftrightarrow \texttt{F0\_in}$ \\
\texttt{\#define F1\_Intrpl7}  & $\texttt{F1\_Intrpl7} \Leftrightarrow \texttt{F1\_in}$ \\
\texttt{\#define F2\_Intrpl7}  & $\texttt{F2\_Intrpl7} \Leftrightarrow \texttt{F2\_in}$ \\
\texttt{\#define F3\_Intrpl7}  & $\texttt{F3\_Intrpl7} \Leftrightarrow \texttt{F3\_in}$ \\
\texttt{\#define F4\_Intrpl7}  & $\texttt{F4\_Intrpl7} \Leftrightarrow \texttt{F4\_in}$ \\
\texttt{\#define F5\_Intrpl7}  & $\texttt{F5\_Intrpl7} \Leftrightarrow \texttt{F5\_in}$ \\
\texttt{\#define F6\_Intrpl7}  & $\texttt{F6\_Intrpl7} \Leftrightarrow \texttt{F6\_in}$ \\
\texttt{\#define F7\_Intrpl7}  & $\texttt{F7\_Intrpl7} \Leftrightarrow \texttt{F7\_in}$ \\
\multicolumn{2}{c}{$\vdots$} \\
\texttt{\#define F18\_Intrpl7} & $\texttt{F18\_Intrpl7} \Leftrightarrow \texttt{F18\_in}$\\
\bottomrule
\end{tabular}
\end{table}

\noindent {\color{blue}補充二:}
\noindent {\color{red}討論:Z方向的網格劃分與非均勻設定 : NZ6 = NZ + 6 }\\
\noindent 在 hyperbolic tangent function 中,z方向採用如下函數做非均勻網格劃分:
\[
y_j = \frac{h}{a} \tanh\!\left( \frac{\xi_j}{2} \ln \frac{1+a}{1-a} \right)
\]
\noindent 其中,$\xi_j$為正規化座標,$\xi_j \in [-1,+1]$
當 $j=0$ 時,$\xi_j = -1$;當 $j=N$ 時,$\xi_j = 1$,所以分母的N為切割之網格數而非切割後的節點數。
程式碼在 {\color{blue} initializationTool.h4-7} 中有定義這個非均勻網格化分之函數:

\begin{lstlisting}[language=C, caption={initializationTool.h4-7}, label={initializationTool.h4-7}]                                                                                                  
#define tanhFunction( L, LatticeSize, a, j, N )     \
(           \
    L/2.0 + LatticeSize/2.0 + ((L/2.0)/a)*tanh((-1.0+2.0*(double)(j)/(double)(N))/2.0*log((1.0+a)/(1.0-a)))     \
)
\end{lstlisting}
\noindent 而在 {\color{blue} initialization.h} 的函數式 {\color{blue} GenerateMesh\_Z()} 中,做了如下定義來分割z方向的非均勻網格:

\begin{lstlisting}[language=C, caption={initialization.h153-162}, label={initialization.h153-162}]                                                                                                  
void GenerateMesh_Z(){...//省略
    double a = GetNonuniParameter();//用以取得最合適之非均勻網格劃分參數
    for( int j = 0; j < NYD6; j++ ){//範圍:遍歷y方向之總節點數
        double total = LZ - HillFunction( y_h[j] ) - minSize; 
        //LZ為含山坡之總高度 ; 

        for( int k = bfr; k < NZ6-bfr; k++ ){
          z_h[j*NZ6+k] = tanhFunction(total,minSize,a,(k-3),(NZ6-7)) + 
                         HillFunction(y_h[j]);
        }
        z_h[j*NZ6+2] = HillFunction( y_h[j] );
        z_h[j*NZ6+(NZ6-3)] = (double)LZ;
    }
    for( int k = bfr; k < NZ6-bfr; k++ ){
      xi_h[k] = tanhFunction( LXi, minSize, a, (k-3), (NZ6-7) ) - minSize/2.0;
    } 
    ....//省略
}
\end{lstlisting}
從上述程式碼可以發現,Z方向切割後的網格數量"NZ6-6",進一步推得,Z方向之節點數應為"NZ6-6"。
所以如果NZ被定義為物理空間切割之網格數(計算點在Link-Wise systerm中作為網格中心點),則NZ+6就是總(包含buffer)之計算點數。

\noindent {\color{blue}補充三:y\_h[i] 在哪裡被宣告?為什麼可以直接使用該矩陣?}
 
\noindent {\color{red}對於指標變數 double* y\_h 的宣告,為什麼單單只在 主程式:main.cu 中宣告,而卻在 initialization.h 直接使用到y\_h[i] ? (未將指標變數映射到某一個特定記憶體)}
\\ \noindent 我的問題是:如果今天宣告一個矩陣 A ,只要一宣告,則 A 自動作為一個指標行記憶,且映射到A矩陣第一個元素的初始存放位址。
\begin{lstlisting}[language=C++, caption={problem disussion}, label={lst:problem-discussion}]
double A[10] ; //宣告一個尺寸10的矩陣
double B = A ; 
//則 B 為 A矩陣 第一個元素的初始存放位址
\end{lstlisting}
但是如果單單宣告一個指標變數 double* y\_h ,則y\_h 作為一個指標變數,還沒有分配記憶體的位址(映射到某一個記憶體),並不能直接使用矩陣元素 y\_h[i],且利用指標運算子(*) : *(y\_h + i) 來表示相應矩陣 y\_h 第i個元素(y\_h[i]) \\
\begin{lstlisting}[language=C++, caption={main.cu35}, label={main.cu35}]
double *x_h, *y_h, *z_h, *xi_h ; //宣告y_h作為一個指標變數,尚未映射到任何記憶體
\end{lstlisting}
\begin{lstlisting}[language=C++, caption={memory.h78-80}, label={memory.h78-80}]
nBytes = NYD6 * sizeof(double);
AllocateHostArray(  nBytes, 4,  &y_h, &Ydep_h[0], &Ydep_h[1], &Ydep_h[2]);
AllocateDeviceArray(nBytes, 4,  &y_d, &Ydep_d[0], &Ydep_d[1], &Ydep_d[2]);
\end{lstlisting}
\begin{lstlisting}[language=C++, caption={memory.h4-14}, label={memory.h4-14}]
void AllocateHostArray(const size_t nBytes, const int num_arrays, ...) {
 va_list args;
 //筆記:va_list, va_start, va_arg, va_end 為C/C++中處理可變參數函數的標準宏。
 //va_list : 可變參數列表型記憶體
 // nBytes 為固定參數 。
 //va_start(可變參數列表型記憶體, 不可變參數型記憶體):將“不可變參數型記憶體的後一個參數位址”映射到“可變參數列表型記憶體”
 va_start( args, num_arrays );//定義可變參數
 for( int i = 0; i < num_arrays; i++ ) {
  double **tmp = va_arg(args, double**);//挑出型別為 double** 的記憶體
	CHECK_CUDA( cudaMallocHost( (void**)tmp, nBytes) );
  //對可變參數列表的所有記憶體都分配 nBytes 大小的記憶體空間
 }
 va_end( args );
}
\end{lstlisting}

\begin{lstlisting}[language=C++, caption={AllocateHostArray的模板}, label={lst:AllocateHostArray-template}]
#include <cstdarg>
#include <iostream>
// 计算任意数量整数的总和
int sum(int count, ...) {  // ... 表示可变参数

  va_list args;          // 声明参数列表
  va_start(args, count); // 初始化,count是最后一个固定参数

  int total = 0;
  for(int i = 0; i < count; i++) {
    total += va_arg(args, int);  // 取出下一个 int 类型参数
  }

  va_end(args);  // 清理
  return total;
}

int main() {
  std::cout << sum(3, 10, 20, 30) << std::endl;  // 输出: 60
  std::cout << sum(5, 1, 2, 3, 4, 5) << std::endl;  // 输出: 15
}
\end{lstlisting}
其中,$\texttt{AllocateHostArray}$ 稱為映射記憶體函數 , 兩件事情可以回答為什麼 double* y\_h 可以映射為 尺寸大小為「$\texttt{NYD6 * sizeof(double)}$」的矩陣:y\_h[$\texttt{NYD6 * sizesaof(double)}$] 的第零個元素。
\begin{itemize}
    \item \texttt{AllocateHostArray} 內部呼叫
  \texttt{cudaMallocHost}:

    \begin{enumerate}
      \item 第一次迴圈 ($i=0$):
        \begin{lstlisting}
tmp = va_arg(args, double**)  // tmp 得到 &y_h
cudaMallocHost((void**)&y_h, NYD6*sizeof(double))\end{lstlisting}
        $\Rightarrow$ 分配記憶體nBytes 給 \texttt{y\_h}
        \item 第二次迴圈 ($i=1$):
        \begin{lstlisting}
tmp = va_arg(args, double**)  // tmp 得到 &Ydep_h[0]
cudaMallocHost((void**)&Ydep_h[0], NYD6*sizeof(double))\end{lstlisting}
        $\Rightarrow$ 分配記憶體nBytes 給 \texttt{Ydep\_h[0]}
      \item 第三次迴圈 ($i=2$):\\
        tmp 得到 \texttt{\&Ydep\_h[1]},分配記憶體 nBytes 給 \texttt{Ydep\_h[1]}
      \item 第四次迴圈 ($i=3$):\\
        tmp 得到 \texttt{\&Ydep\_h[2]},分配記憶體 nBytes 給 \texttt{Ydep\_h[2]}
    \end{enumerate}
  \end{itemize}

\noindent 所以 cudaMallocHost實現了 以指標變數為名稱的矩陣存在,
我只要有了一個特定指標變數,透過函數:\texttt{cudaMallocHost}
將此指標變數已指定記憶體大小映射為相對應名稱的矩陣。
下面再給更詳細的說明關於函數: \texttt{cudaMallocHost} 的功能:
\begin{lstlisting}[language=C, caption={cudaMallocHost function}, label={lst:cudaMallocHost-function}]
double *y_h;  // 只是一個指標變數名稱

cudaMallocHost((void **)&y_h, 3*sizeof(double));
// 分配記憶體後,y_h 就可以當作陣列使用了!

y_h[0], y_h[1], y_h[2]  // 像陣列一樣存取
\end{lstlisting}

\noindent {\color{blue}補充結束}
\end{document}