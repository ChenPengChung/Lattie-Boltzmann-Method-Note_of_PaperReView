\documentclass[12pt]{article}
\usepackage[a4paper,margin=2cm]{geometry} % 明確設定四邊
\usepackage{fontspec} % 字體設定
\usepackage[english]{babel} % English document language
% Note: xeCJK and CJK font setup removed for English environment
\usepackage{setspace} % 設定行距
\linespread{1.2}
\usepackage{titling} % 預設標題下移0.6in
\usepackage{enumitem}
\usepackage{amsmath} % 數學方程式
\usepackage{graphicx} %圖片
\usepackage{float} % 在導言區,讓圖片強制插在原地
\usepackage{xcolor} %字體加入顏色
\usepackage{listings}
\usepackage{physics} % 物理符號
\usepackage{wrapfig} % 文字環繞圖
\usepackage{array} % 表格對齊控制
\usepackage{caption}
\usepackage{amsthm}
\usepackage{subcaption}
\usepackage{tabularx}
\usepackage{booktabs}
\usepackage{multirow}
\usepackage{etoolbox}
\usepackage{titlesec}
\usepackage{tocloft}
\usepackage{hyperref}
\usepackage{makecell}
\usepackage{xcolor}
\usepackage{colortbl}
\hypersetup{
  colorlinks=true,
  linkcolor=blue,
  urlcolor=blue,
  citecolor=blue,
  linktoc=all
}
\usepackage{tcolorbox}
\tcbuselibrary{skins,breakable}
\usepackage{tikz}
\usetikzlibrary{arrows.meta,positioning,shapes.geometric}
\tcbset{
  highlightblock/.style={
    enhanced,
    breakable,
    colback=gray!8,
    colframe=gray!60,
    boxrule=0.4pt,
    arc=0pt,
    boxsep=0pt,
    left=4pt,
    right=4pt,
    top=4pt,
    bottom=4pt,
    before skip=0pt,
    after skip=0pt
  }
}
\usepackage{longtable}

%1 註記行距設定
\setlist[itemize]{itemsep=1.2pt, parsep=0pt, topsep=1pt}
\setlist[enumerate]{itemsep=1.2pt, parsep=0pt, topsep=1pt}
%2 圖表標題設定
\captionsetup{
    labelfont={footnotesize,bf},    % 標籤:小字體+粗體
    skip=10pt                        % 標題與圖表的間距
}
%3 表格間距設定
\setlength{\tabcolsep}{8pt} % 列間距(增加)
\renewcommand{\arraystretch}{1.4} % 行間距(增加)
%4 腳註設定
\usepackage[bottom,hang]{footmisc} % 腳註置於頁面底部,懸掛縮排
\setlength{\footnotesep}{10pt} % 腳註之間的間距(增加)
\setlength{\skip\footins}{12pt plus 5pt minus 2pt} % 正文與腳註之間的間距(增加)
\setlength{\footnotemargin}{1.5em} % 腳註標號與文字的間距(增加)
\renewcommand{\footnoterule}{\vspace*{-3pt}\hrule width 0.4\columnwidth height 0.4pt\vspace*{3pt}} % 腳註分隔線
%5 使用直立字體的定理樣式
\newtheoremstyle{upright}
  {6pt}{6pt}  % 定理環境前後間距(增加)
  {\normalfont}% 使用正常字體,不使用斜體
  {0pt}{\bfseries}{.}{0.5em}{}
%6 定理環境定義
\theoremstyle{upright}
\newtheorem{definition}{Definition}[section]
\newtheorem{theorem}{Theorem}[section]
\newtheorem{lemma}{Lemma}[section]
\newtheorem{corollary}{Corollary}[section]
\newtheorem{example}{Example}[section]
% 重新正確定義 remark 環境 - 確保完全靠左對齊
\makeatletter
\@ifundefined{c@remark}{}{\renewcommand{\theremark}{}}
\newenvironment{remark}{%
  \par\vspace{0.0\baselineskip}%
  \begingroup
  \setlength{\parindent}{0pt}%
  \setlength{\leftskip}{0pt}%
  \noindent\textbf{Remark:}\\
  \ignorespaces
  \setlist[enumerate]{itemsep=0pt, parsep=0pt, topsep=0.0pt, leftmargin=1.5em}%
  \setlist[itemize]{itemsep=0pt, parsep=0pt, topsep=0.0pt, leftmargin=1.5em}%
}{%
  \endgroup
  \par\vspace{0.5\baselineskip}
}
\makeatother
% define a simple derivation environment for derivations/proofs
\newenvironment{derivation}{\par\medskip\noindent\textbf{Derivation.}\ }{\par\medskip}
%7 手動定義中文數字(不含標點符號)
\newcommand{\chinese}[1]{%
  \ifcase#1 零\or 一\or 二\or 三\or 四\or 五\or 六\or 七\or 八\or 九\or 十\or
  十一\or 十二\or 十三\or 十四\or 十五\or 十六\or 十七\or 十八\or 十九\or 二十\fi
}

%8 重新定義章節編號格式 - 使用阿拉伯數字以符合英文文件與常見期刊格式
% (原先使用中文數字表示,改回標準阿拉伯編號)
\renewcommand{\thesection}{\arabic{section}}
\renewcommand{\thesubsection}{\arabic{section}.\arabic{subsection}}
\renewcommand{\thesubsubsection}{\arabic{section}.\arabic{subsection}.\arabic{subsubsection}}
\renewcommand{\theequation}{\arabic{section}.\arabic{equation}}
\renewcommand{\thefigure}{\arabic{section}.\arabic{figure}}
\renewcommand{\thedefinition}{\thesection.\arabic{definition}}
\renewcommand{\thetheorem}{\thesection.\arabic{theorem}}
\renewcommand{\theexample}{\thesection.\arabic{example}}

% 为不同级别的标题增加编号后的空格
\setlength{\cftsecnumwidth}{2.5em}  % section 编号宽度
\setlength{\cftsubsecnumwidth}{3.5em}  % subsection 编号宽度
\setlength{\cftsubsubsecnumwidth}{4.5em}  % subsubsection 编号宽度

% 添加点线(可选)
\renewcommand{\cftsubsubsecleader}{\cftdotfill{\cftdotsep}}
\renewcommand{\cftsubsecleader}{\cftdotfill{\cftdotsep}}
\renewcommand{\cftsecleader}{\cftdotfill{\cftdotsep}}
%9 讓方程式計數器在每個section重置
\counterwithin{equation}{section}
%10 字體設定 - 優化以減少警告
\IfFontExistsTF{Times New Roman}{%
  \setmainfont{Times New Roman}%
}{%
  \setmainfont{TeX Gyre Termes}%
}

%11 標題設定
\setlength{\droptitle}{-1in} % 上移標題1in
\title{\text{4.Volumetric Lattice Boltzmann Models in General Curvature}}
\author{Chen Peng Chung}
\setcounter{section}{0}
% 12调整 subsection 的间距
\titlespacing*{\section}
{0pt}                    % 左边距
{0.0in}                 % 标题前间距
{0.0em}                  % 标题后间距
\titlespacing*{\subsection}
{0pt}                    % 左边距
{-0.05in}                % 标题前间距
{0.0em}                  % 标题后间距
% 调整 subsubsection 的间距  
\titlespacing*{\subsubsection}
{0pt}                    % 左边距
{0.0in}                 % 标题前间距
{0.0em}                  % 标题后間距
%13.

% ========== Monokai Light 配色方案(淺灰背景)==========
\definecolor{monokailight-bg}{RGB}{240,240,240}      % 淺灰背景 #F0F0F0
\definecolor{monokailight-fg}{RGB}{39,40,34}         % 深色文字 #272822
\definecolor{monokailight-comment}{RGB}{117,113,94}  % 註解 #75715E
\definecolor{monokailight-string}{RGB}{152,118,24}   % 字串 #987618
\definecolor{monokailight-keyword}{RGB}{244,0,95}    % 關鍵字 #F4005F
\definecolor{monokailight-function}{RGB}{121,162,0}  % 函數 #79A200
\definecolor{monokailight-number}{RGB}{137,89,168}   % 數字 #8959A8
\definecolor{monokailight-type}{RGB}{0,129,152}      % 型別 #008198
\definecolor{monokailight-operator}{RGB}{244,0,95}   % 運算符 #F4005F
\definecolor{monokailight-border}{RGB}{210,210,210}  % 邊框 #D2D2D2

% ========== CUDA 語言定義 ==========
\lstdefinelanguage{CUDA}{
    language=C++,
    morekeywords={__global__, __device__, __host__, __shared__, 
                  __constant__, __syncthreads, dim3, cudaMalloc,
                  cudaMemcpy, cudaFree, cudaMemcpyHostToDevice,
                  cudaMemcpyDeviceToHost, cudaDeviceSynchronize},
    % 型別關鍵字
    morekeywords=[2]{double, float, int, char, void, size_t,
                     uint, uint2, uint3, uint4,
                     int2, int3, int4,
                     float2, float3, float4,
                     double2, double3, double4},
    % CUDA 內建變數
    morekeywords=[3]{blockIdx, threadIdx, blockDim, gridDim,
                     warpSize},
    sensitive=true
}

% ========== Monokai Light 樣式 ==========
\lstdefinestyle{monokailight}{
    language=CUDA,
    backgroundcolor=\color{monokailight-bg},
    basicstyle=\color{monokailight-fg}\ttfamily\footnotesize,
    commentstyle=\color{monokailight-comment}\itshape,
    keywordstyle=\color{monokailight-keyword}\bfseries,
    keywordstyle=[2]\color{monokailight-type}\bfseries,      % 型別
    keywordstyle=[3]\color{monokailight-function}\bfseries,  % CUDA 變數
    stringstyle=\color{monokailight-string},
    numberstyle=\tiny\color{monokailight-comment},
    % 數字顏色
    literate=
        {0}{{{\color{monokailight-number}0}}}1
        {1}{{{\color{monokailight-number}1}}}1
        {2}{{{\color{monokailight-number}2}}}1
        {3}{{{\color{monokailight-number}3}}}1
        {4}{{{\color{monokailight-number}4}}}1
        {5}{{{\color{monokailight-number}5}}}1
        {6}{{{\color{monokailight-number}6}}}1
        {7}{{{\color{monokailight-number}7}}}1
        {8}{{{\color{monokailight-number}8}}}1
        {9}{{{\color{monokailight-number}9}}}1,
    % 版面設定
    breaklines=true,
    breakatwhitespace=false,
    captionpos=b,
    keepspaces=true,
    numbers=left,
    numbersep=8pt,
    showspaces=false,
    showstringspaces=false,
    showtabs=false,
    tabsize=4,
    % 框線設定
    frame=single,
    rulecolor=\color{monokailight-border},
    framerule=0.8pt,
    % 其他設定
    columns=flexible,
    escapeinside={(*@}{@*)},
    xleftmargin=2em,
    xrightmargin=0.5em,
    framexleftmargin=1.5em
}

% ========== 設為預設樣式 ==========
\lstset{style=monokailight}
%14.
% ========== 表格樣式設定 ==========
\setlength{\arrayrulewidth}{0.5pt}  % 表格線條粗細
\renewcommand{\arraystretch}{1.3}   % 行高
%15listing 設定:
\lstset{
  language=C++,
  basicstyle=\ttfamily\small,
  numbers=left,
  numberstyle=\tiny,
  stepnumber=1,
  numbersep=8pt,
%
  keepspaces=true,      % ★保留空白
  showspaces=false,
  showstringspaces=false,
  showtabs=false,
%
  tabsize=2,            % ★tab 視為 2 個空白(可改 4)
  breaklines=true,
  breakatwhitespace=false,
%
  frame=single,
  captionpos=b
}



\begin{document}
\maketitle
\tableofcontents
\newpage
\section{General Interpolation LBM}
This note presents an extension of the Interpolation-Supplemented Lattice Boltzmann Method (ISLBM) for use in curvilinear coordinate systems. The first strategy is to transform the Cartesian coordinate system to a general curvilinear coordinate system through conformal mapping.
Different from previous papers, this method extends \textbf{ISLBM} without changing the lattice system to accommodate curved motion.
Note that because the transformation is based on coordinate mapping, we still compute the curved particle paths in the computational domain, as shown in the figure below:
\begin{figure}[H]
  \centering 
  \begin{subfigure}{0.45\textwidth}
    \centering
    \includegraphics[width=\textwidth]{1.png}
    \caption{curvilinear coordinate system for SLBM}
    \label{fig:slbm_a}
  \end{subfigure}
  \hfill
  \begin{subfigure}{0.45\textwidth}
    \centering
    \includegraphics[width=\textwidth]{2.png}
    \caption{General Interpolation LBM}
    \label{fig:slbm_b}
  \end{subfigure}
  \caption{2 types of curvilinear coordinate system}
  \label{fig:slbm_curvilinear}
\end{figure}
\section{Transformation}
\subsection{Jacobian relation}
This section woul give the proven of the Jacobian relation : 
\begin{equation}
\begin{bmatrix}
  \xi_{x} & \xi_{y}\\
  \eta_{x} & \eta_{y} 
\end{bmatrix}
= \frac{1}{\mathrm{J}}
\begin{bmatrix}
  y_{\eta} & -x_{\eta} \\
  -y_{\xi} & x_{\xi} 
\end{bmatrix}
\end{equation}
For general curvilinear coordinate in two dimensions, we have to define Lamé coefficient for analyzing:
\begin{equation}\begin{aligned}
\vec{dr}_{+1}|_{q_{2},q_{3}} (\vec{r}) &\equiv \vec{r}(q_{1} + \Delta q_{1} , q_{2} , q_{3}) - \vec{r}(q_{1} , q_{2} , q_{3})
= \left.\frac{\partial \vec{r}}{\partial q_{1}}\right|_{q_{2},q_{3}} dq_{1} 
= \frac{\left.\frac{\partial \vec{r}}{\partial q_{1}}\right|_{q_{2},q_{3}}}{|\left.\frac{\partial \vec{r}}{\partial q_{1}}\right|_{q_{2},q_{3}}|}|\left.\frac{\partial \vec{r}}{\partial q_{1}}\right|_{q_{2},q_{3}}| dq_{1}
\end{aligned}\end{equation}
From the expression above, we can define unit vector and coefficient for differential geometry.
\begin{equation}
\begin{aligned}
h_{1} &\equiv \left|\frac{\partial \vec{r}}{\partial q_{1}}\right|_{q_{2},q_{3}} \\[1.5ex]
\vec{e}_{1} &\equiv \frac{1}{h_{1}}\left.\frac{\partial \vec{r}}{\partial q_{1}}\right|_{q_{2},q_{3}}
\end{aligned}
\end{equation}
where symbol $h_{1}$ is a Lamé coefficient for the coordinate component. 
Let's take the definition to defferential of position vector about variable $(\xi , \eta)$.
\begin{equation}\begin{aligned}\label{eq:r5} 
  &\frac{\partial \vec{r}}{\partial \xi} = h_{\xi} \vec{e}_{\xi} = \vec{e}_{x} x_{\xi} + \vec{e}_{y} y_{\xi} \\[1.5ex]
  &\frac{\partial \vec{r}}{\partial \eta} = h_{\eta} \vec{e}_{\eta}  = \vec{e}_{x} x_{\eta} + \vec{e}_{y} y_{\eta} \\[1.5ex]
  &\frac{\partial \vec{r}}{\partial x} = \vec{e}_{x} = h_{\xi}\vec{e}_{\xi} \xi_{x} +  h_{\eta}\vec{e}_{\eta} \eta_{x} \\[1.5ex]
  &\frac{\partial \vec{r}}{\partial y} = \vec{e}_{y} = h_{\xi}\vec{e}_{\xi} \xi_{y} +  h_{\eta}\vec{e}_{\eta} \eta_{y} \\[1.5ex]
\end{aligned}\end{equation}
we can do some simple computation, and get the result below : 
\begin{equation}\begin{aligned}
  &\frac{h_{\xi}}{x_{\xi}} \vec{e_{\xi}} - \frac{h_{\eta}}{x_{\eta}} \vec{e_{\eta}} = \vec{e}_{y} \left(\frac{y_{\xi}}{x_{\xi}} - \frac{y_{\eta}}{x_{\eta}}\right) = \vec{e}_{y} \frac{y_{\xi}x_{\eta} - y_{\eta}x_{\xi}}{x_{\eta}x_{\xi}} \\[1.5ex]
  &\frac{h_{\xi}}{y_{\xi}} \vec{e_{\xi}} - \frac{h_{\eta}}{y_{\eta}} \vec{e_{\eta}} = \vec{e}_{x} \left(\frac{x_{\xi}}{y_{\xi}} - \frac{x_{\eta}}{y_{\eta}}\right) = \vec{e}_{x} \frac{x_{\xi}y_{\eta} - x_{\eta}y_{\xi}}{y_{\eta}y_{\xi}} \\[1.5ex]
\end{aligned}\end{equation}
\noindent Furthermore, we can rewrite this as:
\begin{equation}\begin{aligned}\label{eq:e6}
  -x_{\eta} h_{\xi} \vec{e_{\xi}} + x_{\xi}h_{\eta} \vec{e_{\eta}} &= \vec{e}_{y} \left(x_{\xi}y_{\eta} - x_{\eta}y_{\xi}\right) \\[1.5ex]
  y_{\eta} h_{\xi} \vec{e_{\xi}} - y_{\xi}h_{\eta} \vec{e_{\eta}}  &= \vec{e}_{x} \left(x_{\xi}y_{\eta} - x_{\eta}y_{\xi}\right)\\[1.5ex]
\end{aligned}\end{equation}
Substitution the equation \eqref{eq:e6} into \eqref{eq:r5} 
\begin{equation}
\begin{bmatrix}
  \xi_{x} & \xi_{y}\\
  \eta_{x} & \eta_{y} 
\end{bmatrix}
= \frac{1}{\left(x_{\xi}y_{\eta} - x_{\eta}y_{\xi}\right)}
\begin{bmatrix}
  y_{\eta} & -x_{\eta} \\
  -y_{\xi} & x_{\xi} 
\end{bmatrix}
\end{equation}
The equation above is Jacobian relation.

\subsection{Transform for Lattice Boltzmann Equation}
First, review the basic governing equation for standard lattice boltzman method i.e., lattice boltzmann equation in Cartesian coordinate with 3 dimension: 
\begin{equation}\begin{aligned}
 f_{i}(\vec{x} + \vec{c}_{i} \Delta t , t + 1 ) = f_{i}(\vec{x} , t) + \Omega(f_{i} (\vec{x} , t), f_{i}^{eq}(\rho , \vec{u}, t))\
\end{aligned}\end{equation}
where the symbol $\Omega$ is a coillision operator, so it can apart two types of forms.

\begin{equation}\begin{aligned}
    f_{i}(\vec{x} + \vec{c}_{i} \Delta t , t + 1 ) = f_{i}(\vec{x} , t) + \omega(f_{i} (\vec{x} , t), f_{i}^{eq}(\rho , \vec{u}, t)) 
\end{aligned}\end{equation}
If the collision operator is chosen as the BGK operator, we call the equation the \textbf{LBGK} equation. 
On the other hand, we can allow each momentum component to have a different relaxation effect. 
To achieve this, we introduce a basis transformation matrix that transforms the function from velocity space to momentum space. 
This means that the streaming step and collision step are performed in separate spaces.
Let's show the transform below : 
\begin{equation}\begin{aligned}
      \textbf{M} f_{i}(\vec{x} + \vec{c}_{i \Delta t , t + 1 }) &= \textbf{M}f_{i}(\vec{x} , t) + \textbf{S}\textbf{M}(f_{i} (\vec{x} , t ,  f_{i}^{eq}(\rho , \vec{u}, t)) \\[1.5ex]
      &= \textbf{M}f_{i}(\vec{x} , t) + \textbf{S} \left(\vec{m}(\vec{x} , t) - \vec{m}^{eq}(\vec{x} , t ,  f_{i}^{eq}(\rho , \vec{u}, t))\right)
\end{aligned}\end{equation}
In this work, we first apply the LBGK equation and use coordinate mapping to modify the streaming process, thereby extending ISLBM to curvilinear coordinate systems. The limitation of this approach is that we have not yet extended the MRT operator to general coordinate systems, as we have not fully grasped the underlying mathematical theory and physical mechanisms.
\subsubsection{Transform Collision Step}

\subsubsection{Transform Straming Step }
\subsection{Boundary Condition }
\subsection{Algorithm}

\end{document}



