\documentclass[12pt]{article}
\usepackage[a4paper,margin=2cm]{geometry} % 明確設定四邊
\usepackage{fontspec} % 字體設定
\usepackage[english]{babel} % English document language
% Note: xeCJK and CJK font setup removed for English environment
\usepackage{setspace} % 設定行距
\linespread{1.2}
\usepackage{titling} % 預設標題下移0.6in
\usepackage{enumitem}
\usepackage{amsmath} % 數學方程式
\usepackage{graphicx} %圖片
\usepackage{float} % 在導言區,讓圖片強制插在原地
\usepackage{xcolor} %字體加入顏色
\usepackage{listings}
\usepackage{physics} % 物理符號
\usepackage{wrapfig} % 文字環繞圖
\usepackage{array} % 表格對齊控制
\usepackage{caption}
\usepackage{amsthm}
\usepackage{subcaption}
\usepackage{tabularx}
\usepackage{booktabs}
\usepackage{multirow}
\usepackage{etoolbox}
\usepackage{titlesec}
\usepackage{tocloft}
\usepackage{hyperref}
\usepackage{makecell}
\usepackage{xcolor}
\usepackage{colortbl}
\hypersetup{
  colorlinks=true,
  linkcolor=blue,
  urlcolor=blue,
  citecolor=blue,
  linktoc=all
}
\usepackage{tcolorbox}
\tcbuselibrary{skins,breakable}
\usepackage{tikz}
\usetikzlibrary{arrows.meta,positioning,shapes.geometric}
\tcbset{
  highlightblock/.style={
    enhanced,
    breakable,
    colback=gray!8,
    colframe=gray!60,
    boxrule=0.4pt,
    arc=0pt,
    boxsep=0pt,
    left=4pt,
    right=4pt,
    top=4pt,
    bottom=4pt,
    before skip=0pt,
    after skip=0pt
  }
}
\usepackage{longtable}

%1 註記行距設定
\setlist[itemize]{itemsep=1.2pt, parsep=0pt, topsep=1pt}
\setlist[enumerate]{itemsep=1.2pt, parsep=0pt, topsep=1pt}
%2 圖表標題設定
\captionsetup{
    labelfont={footnotesize,bf},    % 標籤:小字體+粗體
    skip=10pt                        % 標題與圖表的間距
}
%3 表格間距設定
\setlength{\tabcolsep}{8pt} % 列間距(增加)
\renewcommand{\arraystretch}{1.4} % 行間距(增加)
%4 腳註設定
\usepackage[bottom,hang]{footmisc} % 腳註置於頁面底部,懸掛縮排
\setlength{\footnotesep}{10pt} % 腳註之間的間距(增加)
\setlength{\skip\footins}{12pt plus 5pt minus 2pt} % 正文與腳註之間的間距(增加)
\setlength{\footnotemargin}{1.5em} % 腳註標號與文字的間距(增加)
\renewcommand{\footnoterule}{\vspace*{-3pt}\hrule width 0.4\columnwidth height 0.4pt\vspace*{3pt}} % 腳註分隔線
%5 使用直立字體的定理樣式
\newtheoremstyle{upright}
  {6pt}{6pt}  % 定理環境前後間距(增加)
  {\normalfont}% 使用正常字體,不使用斜體
  {0pt}{\bfseries}{.}{0.5em}{}
%6 定理環境定義
\theoremstyle{upright}
\newtheorem{definition}{Definition}[section]
\newtheorem{theorem}{Theorem}[section]
\newtheorem{lemma}{Lemma}[section]
\newtheorem{corollary}{Corollary}[section]
\newtheorem{example}{Example}[section]
% 重新正確定義 remark 環境 - 確保完全靠左對齊
\makeatletter
\@ifundefined{c@remark}{}{\renewcommand{\theremark}{}}
\newenvironment{remark}{%
  \par\vspace{0.0\baselineskip}%
  \begingroup
  \setlength{\parindent}{0pt}%
  \setlength{\leftskip}{0pt}%
  \noindent\textbf{Remark:}\\
  \ignorespaces
  \setlist[enumerate]{itemsep=0pt, parsep=0pt, topsep=0.0pt, leftmargin=1.5em}%
  \setlist[itemize]{itemsep=0pt, parsep=0pt, topsep=0.0pt, leftmargin=1.5em}%
}{%
  \endgroup
  \par\vspace{0.5\baselineskip}
}
\makeatother
% define a simple derivation environment for derivations/proofs
\newenvironment{derivation}{\par\medskip\noindent\textbf{Derivation.}\ }{\par\medskip}
%7 手動定義中文數字(不含標點符號)
\newcommand{\chinese}[1]{%
  \ifcase#1 零\or 一\or 二\or 三\or 四\or 五\or 六\or 七\or 八\or 九\or 十\or
  十一\or 十二\or 十三\or 十四\or 十五\or 十六\or 十七\or 十八\or 十九\or 二十\fi
}

%8 重新定義章節編號格式 - 使用阿拉伯數字以符合英文文件與常見期刊格式
% (原先使用中文數字表示,改回標準阿拉伯編號)
\renewcommand{\thesection}{\arabic{section}}
\renewcommand{\thesubsection}{\arabic{section}.\arabic{subsection}}
\renewcommand{\thesubsubsection}{\arabic{section}.\arabic{subsection}.\arabic{subsubsection}}
\renewcommand{\theequation}{\arabic{section}.\arabic{equation}}
\renewcommand{\thefigure}{\arabic{section}.\arabic{figure}}
\renewcommand{\thedefinition}{\thesection.\arabic{definition}}
\renewcommand{\thetheorem}{\thesection.\arabic{theorem}}
\renewcommand{\theexample}{\thesection.\arabic{example}}

% 为不同级别的标题增加编号后的空格
\setlength{\cftsecnumwidth}{2.5em}  % section 编号宽度
\setlength{\cftsubsecnumwidth}{3.5em}  % subsection 编号宽度
\setlength{\cftsubsubsecnumwidth}{4.5em}  % subsubsection 编号宽度

% 添加点线(可选)
\renewcommand{\cftsubsubsecleader}{\cftdotfill{\cftdotsep}}
\renewcommand{\cftsubsecleader}{\cftdotfill{\cftdotsep}}
\renewcommand{\cftsecleader}{\cftdotfill{\cftdotsep}}
%9 讓方程式計數器在每個section重置
\counterwithin{equation}{section}
%10 字體設定 - 優化以減少警告
\IfFontExistsTF{Times New Roman}{%
  \setmainfont{Times New Roman}%
}{%
  \setmainfont{TeX Gyre Termes}%
}

%11 標題設定
\setlength{\droptitle}{-1in} % 上移標題1in
\title{\text{4.Volumetric Lattice Boltzmann Models in General Curvature}}
\author{Chen Peng Chung}
\setcounter{section}{0}
% 12调整 subsection 的间距
\titlespacing*{\section}
{0pt}                    % 左边距
{0.0in}                 % 标题前间距
{0.0em}                  % 标题后间距
\titlespacing*{\subsection}
{0pt}                    % 左边距
{-0.05in}                % 标题前间距
{0.0em}                  % 标题后间距
% 调整 subsubsection 的间距  
\titlespacing*{\subsubsection}
{0pt}                    % 左边距
{0.0in}                 % 标题前间距
{0.0em}                  % 标题后間距
%13.

% ========== Monokai Light 配色方案(淺灰背景)==========
\definecolor{monokailight-bg}{RGB}{240,240,240}      % 淺灰背景 #F0F0F0
\definecolor{monokailight-fg}{RGB}{39,40,34}         % 深色文字 #272822
\definecolor{monokailight-comment}{RGB}{117,113,94}  % 註解 #75715E
\definecolor{monokailight-string}{RGB}{152,118,24}   % 字串 #987618
\definecolor{monokailight-keyword}{RGB}{244,0,95}    % 關鍵字 #F4005F
\definecolor{monokailight-function}{RGB}{121,162,0}  % 函數 #79A200
\definecolor{monokailight-number}{RGB}{137,89,168}   % 數字 #8959A8
\definecolor{monokailight-type}{RGB}{0,129,152}      % 型別 #008198
\definecolor{monokailight-operator}{RGB}{244,0,95}   % 運算符 #F4005F
\definecolor{monokailight-border}{RGB}{210,210,210}  % 邊框 #D2D2D2

% ========== CUDA 語言定義 ==========
\lstdefinelanguage{CUDA}{
    language=C++,
    morekeywords={__global__, __device__, __host__, __shared__, 
                  __constant__, __syncthreads, dim3, cudaMalloc,
                  cudaMemcpy, cudaFree, cudaMemcpyHostToDevice,
                  cudaMemcpyDeviceToHost, cudaDeviceSynchronize},
    % 型別關鍵字
    morekeywords=[2]{double, float, int, char, void, size_t,
                     uint, uint2, uint3, uint4,
                     int2, int3, int4,
                     float2, float3, float4,
                     double2, double3, double4},
    % CUDA 內建變數
    morekeywords=[3]{blockIdx, threadIdx, blockDim, gridDim,
                     warpSize},
    sensitive=true
}

% ========== Monokai Light 樣式 ==========
\lstdefinestyle{monokailight}{
    language=CUDA,
    backgroundcolor=\color{monokailight-bg},
    basicstyle=\color{monokailight-fg}\ttfamily\footnotesize,
    commentstyle=\color{monokailight-comment}\itshape,
    keywordstyle=\color{monokailight-keyword}\bfseries,
    keywordstyle=[2]\color{monokailight-type}\bfseries,      % 型別
    keywordstyle=[3]\color{monokailight-function}\bfseries,  % CUDA 變數
    stringstyle=\color{monokailight-string},
    numberstyle=\tiny\color{monokailight-comment},
    % 數字顏色
    literate=
        {0}{{{\color{monokailight-number}0}}}1
        {1}{{{\color{monokailight-number}1}}}1
        {2}{{{\color{monokailight-number}2}}}1
        {3}{{{\color{monokailight-number}3}}}1
        {4}{{{\color{monokailight-number}4}}}1
        {5}{{{\color{monokailight-number}5}}}1
        {6}{{{\color{monokailight-number}6}}}1
        {7}{{{\color{monokailight-number}7}}}1
        {8}{{{\color{monokailight-number}8}}}1
        {9}{{{\color{monokailight-number}9}}}1,
    % 版面設定
    breaklines=true,
    breakatwhitespace=false,
    captionpos=b,
    keepspaces=true,
    numbers=left,
    numbersep=8pt,
    showspaces=false,
    showstringspaces=false,
    showtabs=false,
    tabsize=4,
    % 框線設定
    frame=single,
    rulecolor=\color{monokailight-border},
    framerule=0.8pt,
    % 其他設定
    columns=flexible,
    escapeinside={(*@}{@*)},
    xleftmargin=2em,
    xrightmargin=0.5em,
    framexleftmargin=1.5em
}

% ========== 設為預設樣式 ==========
\lstset{style=monokailight}
%14.
% ========== 表格樣式設定 ==========
\setlength{\arrayrulewidth}{0.5pt}  % 表格線條粗細
\renewcommand{\arraystretch}{1.3}   % 行高
%15listing 設定:
\lstset{
  language=C++,
  basicstyle=\ttfamily\small,
  numbers=left,
  numberstyle=\tiny,
  stepnumber=1,
  numbersep=8pt,
%
  keepspaces=true,      % ★保留空白
  showspaces=false,
  showstringspaces=false,
  showtabs=false,
%
  tabsize=2,            % ★tab 視為 2 個空白(可改 4)
  breaklines=true,
  breakatwhitespace=false,
%
  frame=single,
  captionpos=b
}



\begin{document}
\maketitle
\tableofcontents
\newpage
\section{General Interpolation LBM}
This note presents an extension of the Interpolation-Supplemented Lattice Boltzmann Method (ISLBM) for use in curvilinear coordinate systems. The first strategy is to transform the Cartesian coordinate system to a general curvilinear coordinate system through conformal mapping.
Different from previous papers, this method extends \textbf{ISLBM} without changing the lattice system to accommodate curved motion.
Note that because the transformation is based on coordinate mapping, we still compute the curved particle paths in the computational domain, as shown in the figure below:
\begin{figure}[H]
  \centering 
  \begin{subfigure}{0.45\textwidth}
    \centering
    \includegraphics[width=\textwidth]{1.png}
    \caption{curvilinear coordinate system for SLBM}
    \label{fig:slbm_a}
  \end{subfigure}
  \hfill
  \begin{subfigure}{0.45\textwidth}
    \centering
    \includegraphics[width=\textwidth]{2.png}
    \caption{General Interpolation LBM}
    \label{fig:slbm_b}
  \end{subfigure}
  \caption{2 types of curvilinear coordinate system}
  \label{fig:slbm_curvilinear}
\end{figure}
\section{Transformation}
\subsection{Jacobian relation}
This section will give the proof of the Jacobian relation: 
\begin{equation}
\begin{bmatrix}
  \xi_{x} & \xi_{y}\\
  \eta_{x} & \eta_{y} 
\end{bmatrix}
= \frac{1}{\mathrm{J}}
\begin{bmatrix}
  y_{\eta} & -x_{\eta} \\
  -y_{\xi} & x_{\xi} 
\end{bmatrix}
\end{equation}
For general curvilinear coordinates in two dimensions, we have to define the Lamé coefficient for analysis:
\begin{equation}\begin{aligned}
\vec{dr}_{+1}|_{q_{2},q_{3}} (\vec{r}) &\equiv \vec{r}(q_{1} + \Delta q_{1} , q_{2} , q_{3}) - \vec{r}(q_{1} , q_{2} , q_{3})
= \left.\frac{\partial \vec{r}}{\partial q_{1}}\right|_{q_{2},q_{3}} dq_{1} 
= \frac{\left.\frac{\partial \vec{r}}{\partial q_{1}}\right|_{q_{2},q_{3}}}{|\left.\frac{\partial \vec{r}}{\partial q_{1}}\right|_{q_{2},q_{3}}|}|\left.\frac{\partial \vec{r}}{\partial q_{1}}\right|_{q_{2},q_{3}}| dq_{1}
\end{aligned}\end{equation}
From the expression above, we can define unit vector and coefficient for differential geometry.
\begin{equation}
\begin{aligned}
h_{1} &\equiv \left|\frac{\partial \vec{r}}{\partial q_{1}}\right|_{q_{2},q_{3}} \\[1.5ex]
\vec{e}_{1} &\equiv \frac{1}{h_{1}}\left.\frac{\partial \vec{r}}{\partial q_{1}}\right|_{q_{2},q_{3}}
\end{aligned}
\end{equation}
where the symbol $h_{1}$ is a Lamé coefficient for the coordinate component. 
Let's take the definition to the differential of the position vector about variable $(\xi , \eta)$.
\begin{equation}\begin{aligned}\label{eq:r5} 
  &\frac{\partial \vec{r}}{\partial \xi} = h_{\xi} \vec{e}_{\xi} = \vec{e}_{x} x_{\xi} + \vec{e}_{y} y_{\xi} \\[1.5ex]
  &\frac{\partial \vec{r}}{\partial \eta} = h_{\eta} \vec{e}_{\eta}  = \vec{e}_{x} x_{\eta} + \vec{e}_{y} y_{\eta} \\[1.5ex]
  &\frac{\partial \vec{r}}{\partial x} = \vec{e}_{x} = h_{\xi}\vec{e}_{\xi} \xi_{x} +  h_{\eta}\vec{e}_{\eta} \eta_{x} \\[1.5ex]
  &\frac{\partial \vec{r}}{\partial y} = \vec{e}_{y} = h_{\xi}\vec{e}_{\xi} \xi_{y} +  h_{\eta}\vec{e}_{\eta} \eta_{y} \\[1.5ex]
\end{aligned}\end{equation}
we can do some simple computation, and get the result below : 
\begin{equation}\begin{aligned}
  &\frac{h_{\xi}}{x_{\xi}} \vec{e_{\xi}} - \frac{h_{\eta}}{x_{\eta}} \vec{e_{\eta}} = \vec{e}_{y} \left(\frac{y_{\xi}}{x_{\xi}} - \frac{y_{\eta}}{x_{\eta}}\right) = \vec{e}_{y} \frac{y_{\xi}x_{\eta} - y_{\eta}x_{\xi}}{x_{\eta}x_{\xi}} \\[1.5ex]
  &\frac{h_{\xi}}{y_{\xi}} \vec{e_{\xi}} - \frac{h_{\eta}}{y_{\eta}} \vec{e_{\eta}} = \vec{e}_{x} \left(\frac{x_{\xi}}{y_{\xi}} - \frac{x_{\eta}}{y_{\eta}}\right) = \vec{e}_{x} \frac{x_{\xi}y_{\eta} - x_{\eta}y_{\xi}}{y_{\eta}y_{\xi}} \\[1.5ex]
\end{aligned}\end{equation}
\noindent Furthermore, we can rewrite this as:
\begin{equation}\begin{aligned}\label{eq:e6}
  -x_{\eta} h_{\xi} \vec{e_{\xi}} + x_{\xi}h_{\eta} \vec{e_{\eta}} &= \vec{e}_{y} \left(x_{\xi}y_{\eta} - x_{\eta}y_{\xi}\right) \\[1.5ex]
  y_{\eta} h_{\xi} \vec{e_{\xi}} - y_{\xi}h_{\eta} \vec{e_{\eta}}  &= \vec{e}_{x} \left(x_{\xi}y_{\eta} - x_{\eta}y_{\xi}\right)\\[1.5ex]
\end{aligned}\end{equation}
Substitution the equation \eqref{eq:e6} into \eqref{eq:r5} 
\begin{equation}
\begin{bmatrix}
  \xi_{x} & \xi_{y}\\
  \eta_{x} & \eta_{y} 
\end{bmatrix}
= \frac{1}{\left(x_{\xi}y_{\eta} - x_{\eta}y_{\xi}\right)}
\begin{bmatrix}
  y_{\eta} & -x_{\eta} \\
  -y_{\xi} & x_{\xi} 
\end{bmatrix}
\end{equation}
The equation above is Jacobian relation.

\subsection{Transform for Lattice Boltzmann Equation}
First, review the basic governing equation for the standard lattice Boltzmann method, i.e., the lattice Boltzmann equation in Cartesian coordinates with 3 dimensions: 
\begin{equation}\begin{aligned}
 f_{i}(\vec{x} + \vec{c}_{i} \Delta t , t + 1 ) = f_{i}(\vec{x} , t) + \Omega(f_{i} (\vec{x} , t), f_{i}^{eq}(\rho , \vec{u}, t))\
\end{aligned}\end{equation}
where the symbol $\Omega$ is a collision operator, so it can be separated into two types of forms.
\begin{equation}\begin{aligned}\label{eq:1}
    f_{i}(\vec{x} + \vec{c}_{i} \Delta t , t + 1 ) = f_{i}(\vec{x} , t) + \omega(f_{i} (\vec{x} , t)- f_{i}^{eq}(\rho , \vec{u}, t)) 
\end{aligned}\end{equation}
If the collision operator is chosen as the BGK operator, we call the equation the \textbf{LBGK} equation. 
On the other hand, we can allow each momentum component to have a different relaxation effect. 
To achieve this, we introduce a basis transformation matrix that transforms the function from velocity space to momentum space. 
This means that the streaming step and collision step are performed in separate spaces.
Let's show the transform below: 
\begin{equation}\begin{aligned}\label{eq:2}
      \textbf{M} \vec{f}(\vec{x} + \vec{c}_{i} \Delta t , t + 1 ) &= \textbf{M}\vec{f}(\vec{x} , t) + \textbf{S}\textbf{M}(\vec{f} (\vec{x} , t) -  \vec{f}^{eq}(\rho , \vec{u}, t)) \\[1.5ex]
      &= \textbf{M}\vec{f}(\vec{x} , t) + \textbf{S} \left(\vec{m}(\vec{x} , t) - \vec{m}^{eq}(\vec{x} , t ,  f_{i}^{eq}(\rho , \vec{u}, t))\right)
\end{aligned}\end{equation}
In this work, we first apply the LBGK equation and use coordinate mapping to modify the streaming process, thereby extending ISLBM to curvilinear coordinate systems. The limitation of this approach is that we have not yet extended the MRT operator to general coordinate systems, as we have not fully grasped the underlying mathematical theory and physical mechanisms.
\subsubsection{Transform Collision Step}
Equations \eqref{eq:1} and \eqref{eq:2} are described in three-dimensional Cartesian coordinates. To transform these equations to general curvilinear coordinates, we note that the position vector only appears at discrete grid nodes, and the collision operator does not involve derivatives with respect to position. Therefore, we can directly substitute $\vec{x}$ with $\vec{\xi}$, where $\vec{x}$ denotes the Cartesian coordinate position and $\vec{\xi}$ denotes the curvilinear coordinate position.

We can say that position information only has position variables located at the grid nodes, so we can simply transform by substituting.
\begin{equation}\begin{aligned}
    f_{i}^{\star}(\vec{\xi} , t) = f_{i}(\vec{\xi} , t) + \omega(f_{i} (\vec{\xi} , t) - f_{i}^{eq}(\rho , \vec{u}, t)) 
\end{aligned}\end{equation}
\subsubsection{Transform Streaming Step }
\paragraph{Definition of normalized discrete velocity set}
In general curvilinear coordinates, the variation of the position vector or velocity or any physical variable related to the length dimension needs to consider the transform factor for basis transformation. 
This is complex for analyzing the motion of particles. Let's define the normalized discrete velocity set. 
\begin{equation}\begin{aligned}
  \vec{e}_{\alpha} = \underbrace{c_{\alpha}^{i}}_{\text{non-dimension}} \ \underbrace{\vec{g}_{i}(\vec{\xi})}_{\text{curvature tangent vector}}  \ \underbrace{\frac{\Delta x}{\Delta t}} _{\text{(lattice speed )}\approx 1} 
\end{aligned}\end{equation}
where the vector $\vec{e}_{\alpha}$ is the non-dimensional discrete particle velocity set, and the vector $\vec{g}_{i}$ is the tangent vector related to the curvature at the position. $\frac{\Delta x}{\Delta t}$ is the lattice speed, and its function in this definition is velocity dimension.
So we rewrite the definition to another form below based on differential geometry.
\begin{equation}\begin{aligned}
  \vec{e}_{\alpha}^{\ j} = c_{\alpha}^{i}  \frac{\partial \xi_{j}}{\partial x_{i}}
\end{aligned}\end{equation}
where $x$ represents the index of the Cartesian coordinate and $\xi$ is the index of the general curvilinear coordinate, and the differential represents the basis transform from Cartesian coordinates to curvilinear coordinates. 
Therefore, we can see the particle velocity distortion through the factor of the transform.
\paragraph{Path integration of the particle velocity}
The streaming step equation $$f_{\alpha}(\vec{x} , t) = f_{\alpha}^{\star}(\vec{x} - \vec{c}_{\alpha} \delta t, t)$$
We can see that the essential issue is computing the integration of the discrete particle velocity to know the position where the post-streaming position and streaming length are. The length of the streaming step is defined as below: 
\begin{equation}\begin{aligned}
  \delta \vec{\xi}_{\alpha} = \int_{0}^{\Delta t} d \vec{\xi}_{\alpha} = \int_{0}^{\Delta t} \vec{e}_{\alpha} dt 
\end{aligned}\end{equation}
\noindent 1st explicit Euler method: 
\begin{equation}\begin{aligned}
  \delta \vec{\xi}_{\alpha} \approx \vec{e}_{\alpha} \Delta t \dots O(\Delta t)
\end{aligned}\end{equation}
\noindent 2nd order Runge-Kutta method: \\
\noindent Using the 2nd order Runge-Kutta method to solve the integration, it is separated into two steps: 
\begin{equation}\begin{aligned}\label{eq:RK}
  &\text{step 1 : } \Delta  \vec{\xi}_{\alpha}^{(1)} = \frac{1}{2} \Delta t \vec{e}_{\alpha} \\[1.5ex]
  &\text{step 2 : } \Delta  \vec{\xi}_{\alpha} =  \Delta t \vec{e}_{\alpha} (\vec{\xi}  - \vec{\xi}_{\alpha}^{(1)})
\end{aligned}\end{equation}
The result of discrete integration \eqref{eq:RK} has 2nd order accuracy in time space. The figure shows the method of 2nd order Runge-Kutta to treat integration.
\begin{figure}[H]
\centering
\includegraphics[width=0.5\textwidth]{2.png}
\caption{integration skill-2nd Runge-Kutta}
\label{fig:RK}
\end{figure} 
\noindent In Interpolation-Supplemented LBM, the streaming step is implemented by pulling data backward from the pre-streaming position to the current grid node through interpolation. This approach ensures that information does not spread from grid nodes to non-physical locations. Therefore, when integrating the particle path, we integrate backward from $\vec{\xi}_{\alpha}$ to $\vec{\xi}_{\alpha}-\Delta \vec{\xi}_{\alpha}$, rather than forward from $\vec{\xi}_{\alpha}$ to $\vec{\xi}_{\alpha} + \Delta \vec{\xi}_{\alpha}$.
\paragraph{Streaming step after transformation}
When we get the result of the path integration of particle discrete velocity along the direction, we can rewrite the streaming step equation below: 
The discrete process is a little similar to the process of distribution for time-space discretizing. However, the term discretized is different; one is time integration of distribution and the other is time integration of particle path. 
In curvilinear coordinates, we have 
\begin{equation}\begin{aligned}
  f_{\alpha} (\vec{\xi} , t ) = f_{\alpha}^{\star}(\vec{\xi}-\Delta \vec{\xi}_{\alpha} , t)
\end{aligned}\end{equation}
where $\Delta \vec{\xi}_{\alpha}$, the change of position, has 2nd order accuracy in time space, and the R.H.S. term is the post-collision function.
\paragraph{Accuracy effect in time space for lattice Boltzmann equation - BGK operator}
\noindent LBGK in curvilinear coordinates can be rewritten below: 
\begin{equation}\begin{aligned}\label{eq:4}
  f_{\alpha} (\vec{\xi}  , t + 1) - f_{\alpha} (\vec{\xi} - \Delta \vec{\xi}_{\alpha}, t ) = \omega (f_{\alpha}(\vec{\xi} - \Delta \vec{\xi}_{\alpha}, t ) - f_{\alpha}^{eq}(\rho , \vec{u} , t)) \dots (\text{curvilinear coordinate})
\end{aligned}\end{equation}
and we can define a material derivative for the discrete-velocity Boltzmann equation:
\begin{equation}\begin{aligned}
   \mathcal{D}_{t} = \partial_{t} + \frac{\delta\vec{\xi}_{\alpha}}{\delta t} \cdot \vec{\nabla}_{\xi}
\end{aligned}\end{equation}
We call the above operator the "discrete-velocity material derivative".
Using Taylor expansion on equation \eqref{eq:4}:  
\begin{equation}\begin{aligned}
  &\left.\left(\partial_{t} + \frac{\delta\vec{\xi}_{\alpha}}{\delta t} \cdot \vec{\nabla}_{\xi}\right)f_{\alpha} \Delta t  + \frac{1}{2}\left(\partial_{t} + \frac{\delta\vec{\xi}_{\alpha}}{\delta t} \cdot \vec{\nabla}_{\xi}\right)^2f_{\alpha} \Delta t ^{2}\right |_{(\vec{\xi} - \Delta \vec{\xi}_{\alpha}, t )}= \omega (f_{\alpha}(\vec{\xi} - \Delta \vec{\xi}_{\alpha}, t ) - f_{\alpha}^{eq}(\rho , \vec{u} , t)) \\[1.5ex]
\end{aligned}\end{equation}
Left-hand side and right-hand side are both divided by $\Delta t$, which can recover the N-S equation in differential form.
\begin{equation}\begin{aligned}
  &\left.\left(\partial_{t} + \frac{\delta\vec{\xi}_{\alpha}}{\delta t} \cdot \vec{\nabla}_{\xi}\right)f_{\alpha} + \frac{1}{2}\left(\partial_{t} + \frac{\delta\vec{\xi}_{\alpha}}{\delta t} \cdot \vec{\nabla}_{\xi}\right)^2f_{\alpha} \Delta t \right|_{(\vec{\xi} - \Delta \vec{\xi}_{\alpha}, t )} + O(\Delta t^{2})= \omega (f_{\alpha}(\vec{\xi} - \Delta \vec{\xi}_{\alpha}, t ) - f_{\alpha}^{eq}(\rho , \vec{u} , t)) \\[1.5ex]
\end{aligned}\end{equation}
The second term of the velocity-discrete Boltzmann equation corresponds to the shear stress field through Chapman-Enskog expansion. 
If we use the explicit Euler method to discretize the particle path integration, the first term has only zeroth-order temporal accuracy, while the stress term has only first-order temporal accuracy. This is insufficient to recover the Navier-Stokes equation. 
Therefore, we must use a method that provides higher-order temporal accuracy for the path integration, ensuring that the differential term of the velocity-discrete Boltzmann equation has at least second-order temporal accuracy.
\subsection{Multiple Scale Expansion}
For collision and streaming in mesoscopic space, and advection and diffusion in macroscopic space, we have three types of time scales:
\begin{enumerate}
\item $K^{(0)}$: The first time scale, related to collision and streaming in mesoscopic space. 
\item $K^{(1)}$: The second time scale, related to convection in macroscopic space.
\item $K^{(2)}$: The third time scale, related to diffusion in macroscopic space. 
\end{enumerate}
If K = 0.01, then 
we can define three types of time variables: (the value is about $K^{(0)}, K^{(1)}$, $K^{(2)}$)\\
\begin{equation}\begin{aligned}
  &t \in K^{(2)}\\[1.5ex]
  &t^{(1)} \equiv K * t \in K^{(1)}\\[1.5ex] 
  &t^{(2)} \equiv K^{2} * t \in K^{(0)}
\end{aligned}\end{equation}
and define two types of space variables: 
\begin{equation}
  \vec{r}^{(1)} \equiv K * \vec{r} 
\end{equation}
Therefore, the differential about the time variable is:
\begin{equation}
  \frac{\partial }{\partial t} = K \frac{\partial }{\partial t^{(1)}} + K^{2} \frac{\partial }{\partial t^{(2)}}
= K \partial_{t}^{(1)} + K \partial_{t}^{(2)}
\end{equation}
In physical space, we have the relation about two variables: 
\begin{equation}\begin{aligned}
     \frac{\partial }{\partial \vec{\xi}} &= K \frac{\partial }{\partial \vec{\xi}^{(1)}}\\[1.5ex]
     \Rightarrow \vec{\nabla}_{\vec{\xi}} &= K \vec{\nabla}_{\vec{\xi}^{(1)}}
\end{aligned}\end{equation}
Finally, for three types of time variables, the multi-scale expansion of the distribution function: 
\begin{equation}
   f_{i} = f_{i}^{eq} + K f_{i}^{(1)}  + K^{2}f_{i}^{(2)} + \dots
\end{equation} 
We have to know that the time scale is a domain for time space like $K^{(0)}, K^{(1)}$, and the element of the time scale is a variable, and their scales have different properties. 
\paragraph{Multi-Scale Technique and LBGK}
Before analyzing the LBGK, first, review the form of the basic equation: 
\begin{equation}\begin{aligned}
f_{\alpha}(\vec{r}+\vec{e}_{\alpha} \delta t , t +\delta t) - f_{\alpha}(\vec{r} , t) = -\frac{\delta t}{\tau } \left[f^{neq}(\vec{r} ,t)\right]
\end{aligned}\end{equation}
Using particle material derivative: 
\begin{equation*}
  \mathcal{D}_{t} = \partial_{t} + \vec{e}_{\alpha} \cdot \vec{\nabla}
\end{equation*}
and Taylor expansion to analyze the left-hand side of the LBGK:
\begin{equation}\begin{aligned}
    \left.\mathcal{D}_{t}\delta t  + \frac{\delta t^{2}}{2}\mathcal{D}_{t}^{2} f_{\alpha} \right|_{\vec{r} , t} = -\frac{\delta t}{\tau } \left[f^{neq}(\vec{r} ,t)\right]
\end{aligned} \end{equation}
\begin{equation}\begin{aligned}
\left(\partial_{t} + \vec{e}_{\alpha} \cdot \vec{\nabla}\right) f_{\alpha}   + \frac{\delta t^{1}}{2}\left(\partial_{t} + \vec{e}_{\alpha} \cdot \vec{\nabla}\right)^{2} f_{\alpha} = -\frac{1}{\tau } \left[f^{neq}(\vec{r} ,t)\right]
\end{aligned} \end{equation}
substitute the above equation using the discrete velocity Boltzmann equation: 
$$\left(\partial_{t} + \vec{e}_{\alpha} \cdot \vec{\nabla}\right)f_{\alpha} = -\frac{1}{\tau } \left[f^{neq}(\vec{r} ,t)\right]  $$
we have the equation below: 
\begin{equation}\begin{aligned}\label{eq:Goback}
\left(\partial_{t} + \vec{e}_{\alpha} \cdot \vec{\nabla}\right) f_{\alpha}  -\frac{\delta t^{1}}{2}\left(\partial_{t} + \vec{e}_{\alpha} \cdot \vec{\nabla}\right) \frac{1}{\tau } \left[f_{\alpha}^{neq}(\vec{r} ,t)\right] = -\frac{1}{\tau } \left[f_{\alpha}^{neq}(\vec{r} ,t)\right]
\end{aligned}\end{equation}
For analyzing the equation above, review the multiple scale relation as shown below: 
\begin{equation}\begin{aligned}
&\partial_{t} = K \partial_{t}^{(1)} + K^{2} \partial_{t}^{(2)} + \dots \\[1.5ex]
&\vec{\nabla} = K \vec{\nabla}^{(1)} + K^{2} \vec{\nabla}^{(2)} + \dots \\[1.5ex]
&f_{\alpha}^{neq} = f_{\alpha} - f_{\alpha}^{eq} = K f_{\alpha}^{(1)} + K^{2} f_{\alpha}^{(2)} + \dots
\end{aligned}\end{equation}
and go back to the equation \eqref{eq:Goback}, we have 
\begin{equation}\begin{aligned}
  &K \left(\partial_{t}^{(1)}+ \vec{e}_{\alpha} \cdot \vec{\nabla}^{(1)}\right)  f_{\alpha} = -K\frac{1}{\tau} \left[f_{\alpha}^{(1)}(\vec{r}, t)\right]  \\[1.5ex]
  &K^{2}  \left(\partial_{t}^{(2)}+ \vec{e}_{\alpha} \cdot \vec{\nabla}^{(2)}\right)  f_{\alpha} -\frac{\delta t K^{2} }{2 \tau}\left(\partial_{t}^{(1)} + \vec{e}_{\alpha} \cdot \vec{\nabla}^{(1)} \right) \left[f_{\alpha}^{(1)}(\vec{r} ,t)\right]  = -K^{2}\frac{1}{\tau} \left[f_{\alpha}^{(2)}(\vec{r}, t)\right]  \\[1.5ex]
\end{aligned}\end{equation}
But $f_{\alpha} =f_{\alpha}^{eq}+f_{\alpha}^{neq} = f_{\alpha}^{eq} + K f_{\alpha}^{(1)}+\dots $, the equation can be shown below :
\begin{equation}\begin{aligned}
  &K \left(\partial_{t}^{(1)}+ \vec{e}_{\alpha} \cdot \vec{\nabla}^{(1)}\right)  f_{\alpha}^{eq} = -K\frac{1}{\tau} \left[f_{\alpha}^{(1)}(\vec{r}, t)\right]  \\[1.5ex]
  &K^{2}  \left(\partial_{t}^{(2)}+ \vec{e}_{\alpha} \cdot \vec{\nabla}^{(2)}\right)  f_{\alpha}^{eq} -\frac{\delta t K^{2} }{2 \tau}\left(\partial_{t}^{(1)} + \vec{e}_{\alpha} \cdot \vec{\nabla}^{(1)} \right) \left[f_{\alpha}^{(1)}(\vec{r} ,t)\right]  + K^{2} \left(\partial_{t}^{(1)}+ \vec{e}_{\alpha} \cdot \vec{\nabla}^{(1)}\right)  f_{\alpha}^{(1)}\\[1.5ex]
  &= -K^{2}\frac{1}{\tau} \left[f_{\alpha}^{(2)}(\vec{r}, t)\right]  \\[1.5ex]
\end{aligned}\end{equation}



\subsection{Boundary Condition }
In multiple scale analysis, the first time scale non-equilibrium distribution function is: 
\begin{equation}\begin{aligned}
  & -\omega \delta t\left(\partial_{t}^{(1)}+ \vec{e}_{\alpha} \cdot \vec{\nabla}^{(1)}\right)  f_{\alpha}^{eq} = \left[f_{\alpha}^{(1)}(\vec{r}, t)\right]  \\[1.5ex]
\end{aligned}\end{equation}
where the differential variable is the second time variable $t^{(1)}$, which belongs to the advection time scale $K^{(1)}$. 
And the Maxwell-Boltzmann equilibrium distribution function with Hermite polynomials expansion has: 
\begin{equation}
    f_{\alpha}^{eq} = w_{\alpha} \rho \left(1 +\frac{e_{\alpha}^{i}u_{i}}{c_s^{2}} + \frac{u_{i}u_{j}\left(e_{\alpha}^{i} e_{\alpha}^{j} - c_s^{2} \delta^{ij}\right)}{2c_{s}^{4}}\right)
\end{equation}
so we have to deal with the macroscopic derivative: 
\begin{equation}\begin{aligned}
    &\partial_{t}^{(1)} f_{\alpha}^{eq} =   \frac{\partial f_{\alpha}^{eq}}{\partial u_{i}}\partial_{t}^{(1)} u_{i}  +  \frac{\partial f_{\alpha}^{eq}}{\partial \rho }\partial_{t}^{(1)} \rho \\[1.5ex]
    &\vec{\nabla}^{(1)}f_{\alpha}^{eq}  =   \frac{\partial f_{\alpha}^{eq}}{\partial u_{i}}\vec{\nabla}^{(1)} u_{i}  +  \frac{\partial f_{\alpha}^{eq}}{\partial \rho } \vec{\nabla}^{(1)} \rho \\[1.5ex]  
\end{aligned}\end{equation}
\noindent where $t^{(1)}\in K^{(1)}$ , which is time scale about advection.
from equilibirium distribution function, the differential term have the result below : 
\begin{equation}\begin{aligned}
  \frac{\partial f_{\alpha}^{eq}}{\partial u_{i}}(\vec{r}, t) &= w_{\alpha}\rho \left[\frac{e_{\alpha}^{i}}{c_{s}^{2}} + \frac{u_{j}\left(e_{\alpha}^{i}e_{\alpha}^{j} - c_s^{2}\delta^{ij}\right)}{2c_s^{4}}\right] \\[1.5ex]
  &= \ w_{\alpha}\rho \left[\frac{1}{c_s^{2}}\left(e_{\alpha}^{i} - u_{i}\right) + \frac{u_{j}e_{\alpha}^{i}e_{\alpha}^{j}}{2c_s^{4}}\right]\\[1.5ex]
  &\approx \frac{e_{\alpha}^{i} - u_{i}}{c_{s}^{2}} f_{\alpha}^{eq}\\[1.5ex]
  \frac{\partial f_{\alpha}^{eq}}{\partial \rho}(\vec{r}, t) &= \frac{1}{\rho} f_{\alpha}^{eq}(\vec{r} ,t)
\end{aligned}\end{equation} 
\noindent 
Using macroscopic conservation law (include mass conservation and momentum conservation) to solve the time differential of density and vleocity.
\begin{equation}\begin{aligned}
& \frac{\partial^{(1)} \rho}{\partial t} =  - \vec{\nabla}^{(1)}\rho\vec{u} \\[1.5ex]
& \frac{\partial^{(1)} \vec{u}}{\partial t}= -\vec{u}\cdot \vec{\nabla}^{(1)} \vec{u} - \frac{1}{\rho} \vec{\nabla}^{(1)} p
\end{aligned}\end{equation}
\noindent For equilibilirium distribution function, the first scale time variable derivation have : 
\begin{equation}\begin{aligned}
   &\partial_{t}^{(1)} f_{\alpha}^{eq} =  f_{\alpha}^{eq} \left(\frac{e_{\alpha}^{i} - u_{i}}{c_{s}^{2}} (-u_{\beta}\frac{\partial^{(1)}}{\partial x_{\beta}} u_{i} - \frac{1}{\rho} \frac{\partial ^{(1)}}{\partial x_{i}} p)
   + \frac{-1}{\rho}\frac{\partial^{(1)}}{\partial x_{\beta}}\rho u_{\beta}\right)\\[1.5ex]
   &\vec{\nabla}^{(1)}f_{\alpha}^{eq}  =  f_{\alpha}^{eq} \left(\frac{e_{\alpha}^{i} - u_{i}}{c_{s}^{2}} (\vec{\nabla}^{(1)} u_{i})+ \frac{1}{\rho} \vec{\nabla}^{(1)} \rho\right)\\[1.5ex]
   &e_{\alpha}^{\gamma}\frac{\partial ^{(1)}}{\partial x_{\gamma}}f_{\alpha}^{eq} = e_{\alpha}^{\gamma} f_{\alpha}^{eq} \left(\frac{e_{\alpha}^{i} - u_{i}}{c_{s}^{2}} (\frac{\partial ^{(1)}}{\partial x_{\gamma}} u_{i})+ \frac{1}{\rho} \frac{\partial ^{(1)}}{\partial x_{\gamma}} \rho\right)\\[1.5ex]
  \end{aligned}\end{equation}
\noindent Finially, we can get the $\textbf{the first scale non-equibilirium distribution function}$
\begin{equation}\begin{aligned}
 f_{\alpha}^{(1)} = \omega \delta t \left( A \cdot [-u_{\beta}\frac{\partial^{(1)}}{\partial x_{\beta}} u_{i} \underbrace{- \frac{1}{\rho} \frac{\partial ^{(1)}}{\partial x_{i}} p}_{ignore}  + e_{\alpha}^{\gamma} \frac{\partial^{(1)}}{\partial x_{\gamma}}u_{i}]+ \underbrace{B\cdot [-\frac{\partial^{(1)}\rho u_{\beta}}{\partial x_{\beta}} + e_{\alpha}^{\gamma}\frac{\partial^{(1)}\rho}{\partial x_{\gamma}}]}_{ignore} \right)f_{\alpha}^{eq}
\end{aligned}\end{equation}
where 
\begin{equation}\begin{aligned}
  &A = \frac{e_{\alpha}^{i} - u_{i}}{c_{s}^{2}}\\[1.5ex]
  &B =  \frac{1}{\rho}\\
  \end{aligned}\end{equation}
\noindent In this section ,  we have to give the assumption : 
\begin{enumerate}
\item : $\text{pressure field}$ is a constant field, so $\frac{\partial^{(1)}}{\partial x_{i}}p = 0$
\item : $\text{density field}$ is a constant field, so $\frac{\partial^{(1)}}{\partial x_{\beta}}\rho = 0 $
\end{enumerate}
From the part of the equation above, I could find some zero term, let's show below : 
\begin{equation}\begin{aligned}
  &\frac{e_{\alpha}^{i} - u_{i}}{c_s^{2}} \cdot \left(\frac{-1}{\rho}\frac{\partial^{(1)}}{\partial x_{i}}p = 0\right) \dots (\text{second term of A})\\[1.5ex]
  &\frac{1}{\rho}\left((e_{\alpha}^{\beta} - u_{\beta})\frac{\partial \rho}{\partial x_{\beta}}\right) = 0 \dots (\text{in B})\\[1.5ex]
  & -\frac{\partial^{(1)} \rho u_{\beta}}{\partial x_{\beta}} = -u_{\beta}\frac{\partial^{(1)} \rho }{\partial x_{\beta}} - \rho\frac{\partial^{(1)}  u_{\beta}}{\partial x_{\beta}} \\
\end{aligned}\end{equation}
The distribution function is : (could be contolled by macroscopic variables)
\begin{equation}\begin{aligned}
  f_{\alpha} &= f_{\alpha}^{eq} + f_{\alpha}^{(1)} \\[1.5ex]
  &= f_{\alpha}^{eq}  + \omega \delta t \left(\frac{(e_{\alpha}^{i} - u_{i})(e_{\alpha}^{\beta} - u_{\beta})}{c_s^{2}}\frac{\partial ^{(1)}u_{i}}{\partial x_{\beta}}- \frac{\partial^{(1)}  u_{\beta}}{\partial x_{\beta}}\right)f_{\alpha}^{eq}
\end{aligned}\end{equation}
where Knuson number is 1, and gerneral distribution function can be calulate as above equation.
\subsection{Algorithm}

\end{document}


