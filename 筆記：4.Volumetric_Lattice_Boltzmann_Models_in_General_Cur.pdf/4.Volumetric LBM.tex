\documentclass[12pt]{article}
\usepackage[a4paper,margin=2cm]{geometry} % 明確設定四邊
\usepackage{fontspec} % 字體設定
\usepackage[english]{babel} % English document language
% Note: xeCJK and CJK font setup removed for English environment
\usepackage{setspace} % 設定行距
\linespread{1.2}
\usepackage{titling} % 預設標題下移0.6in
\usepackage{enumitem}
\usepackage{amsmath} % 數學方程式
\usepackage{graphicx} %圖片
\usepackage{float} % 在導言區,讓圖片強制插在原地
\usepackage{xcolor} %字體加入顏色
\usepackage{listings}
\usepackage{physics} % 物理符號
\usepackage{wrapfig} % 文字環繞圖
\usepackage{array} % 表格對齊控制
\usepackage{caption}
\usepackage{amsthm}
\usepackage{tabularx}
\usepackage{booktabs}
\usepackage{multirow}
\usepackage{etoolbox}
\usepackage{titlesec}
\usepackage{tocloft}
\usepackage{hyperref}
\usepackage{makecell}
\usepackage{xcolor}
\usepackage{colortbl}
\hypersetup{
  colorlinks=true,
  linkcolor=blue,
  urlcolor=blue,
  citecolor=blue,
  linktoc=all
}
\usepackage{tcolorbox}
\tcbuselibrary{skins,breakable}
\usepackage{tikz}
\usetikzlibrary{arrows.meta,positioning,shapes.geometric}
\tcbset{
  highlightblock/.style={
    enhanced,
    breakable,
    colback=gray!8,
    colframe=gray!60,
    boxrule=0.4pt,
    arc=0pt,
    boxsep=0pt,
    left=4pt,
    right=4pt,
    top=4pt,
    bottom=4pt,
    before skip=0pt,
    after skip=0pt
  }
}
\usepackage{longtable}

%1 註記行距設定
\setlist[itemize]{itemsep=1.2pt, parsep=0pt, topsep=1pt}
\setlist[enumerate]{itemsep=1.2pt, parsep=0pt, topsep=1pt}
%2 圖表標題設定
\captionsetup{
    labelfont={footnotesize,bf},    % 標籤:小字體+粗體
    skip=10pt                        % 標題與圖表的間距
}
%3 表格間距設定
\setlength{\tabcolsep}{8pt} % 列間距(增加)
\renewcommand{\arraystretch}{1.4} % 行間距(增加)
%4 腳註設定
\usepackage[bottom,hang]{footmisc} % 腳註置於頁面底部,懸掛縮排
\setlength{\footnotesep}{10pt} % 腳註之間的間距(增加)
\setlength{\skip\footins}{12pt plus 5pt minus 2pt} % 正文與腳註之間的間距(增加)
\setlength{\footnotemargin}{1.5em} % 腳註標號與文字的間距(增加)
\renewcommand{\footnoterule}{\vspace*{-3pt}\hrule width 0.4\columnwidth height 0.4pt\vspace*{3pt}} % 腳註分隔線
%5 使用直立字體的定理樣式
\newtheoremstyle{upright}
  {6pt}{6pt}  % 定理環境前後間距(增加)
  {\normalfont}% 使用正常字體,不使用斜體
  {0pt}{\bfseries}{.}{0.5em}{}
%6 定理環境定義
\theoremstyle{upright}
\newtheorem{definition}{Definition}[section]
\newtheorem{theorem}{Theorem}[section]
\newtheorem{lemma}{Lemma}[section]
\newtheorem{corollary}{Corollary}[section]
\newtheorem{example}{Example}[section]
% 重新正確定義 remark 環境 - 確保完全靠左對齊
\makeatletter
\@ifundefined{c@remark}{}{\renewcommand{\theremark}{}}
\newenvironment{remark}{%
  \par\vspace{0.0\baselineskip}%
  \begingroup
  \setlength{\parindent}{0pt}%
  \setlength{\leftskip}{0pt}%
  \noindent\textbf{Remark:}\\
  \ignorespaces
  \setlist[enumerate]{itemsep=0pt, parsep=0pt, topsep=0.0pt, leftmargin=1.5em}%
  \setlist[itemize]{itemsep=0pt, parsep=0pt, topsep=0.0pt, leftmargin=1.5em}%
}{%
  \endgroup
  \par\vspace{0.5\baselineskip}
}
\makeatother
% define a simple derivation environment for derivations/proofs
\newenvironment{derivation}{\par\medskip\noindent\textbf{Derivation.}\ }{\par\medskip}
%7 手動定義中文數字(不含標點符號)
\newcommand{\chinese}[1]{%
  \ifcase#1 零\or 一\or 二\or 三\or 四\or 五\or 六\or 七\or 八\or 九\or 十\or
  十一\or 十二\or 十三\or 十四\or 十五\or 十六\or 十七\or 十八\or 十九\or 二十\fi
}

%8 重新定義章節編號格式 - 使用阿拉伯數字以符合英文文件與常見期刊格式
% (原先使用中文數字表示,改回標準阿拉伯編號)
\renewcommand{\thesection}{\arabic{section}}
\renewcommand{\thesubsection}{\arabic{section}.\arabic{subsection}}
\renewcommand{\thesubsubsection}{\arabic{section}.\arabic{subsection}.\arabic{subsubsection}}
\renewcommand{\theequation}{\arabic{section}.\arabic{equation}}
\renewcommand{\thefigure}{\arabic{section}.\arabic{figure}}
\renewcommand{\thedefinition}{\thesection.\arabic{definition}}
\renewcommand{\thetheorem}{\thesection.\arabic{theorem}}
\renewcommand{\theexample}{\thesection.\arabic{example}}

% 为不同级别的标题增加编号后的空格
\setlength{\cftsecnumwidth}{2.5em}  % section 编号宽度
\setlength{\cftsubsecnumwidth}{3.5em}  % subsection 编号宽度
\setlength{\cftsubsubsecnumwidth}{4.5em}  % subsubsection 编号宽度

% 添加点线(可选)
\renewcommand{\cftsubsubsecleader}{\cftdotfill{\cftdotsep}}
\renewcommand{\cftsubsecleader}{\cftdotfill{\cftdotsep}}
\renewcommand{\cftsecleader}{\cftdotfill{\cftdotsep}}
%9 讓方程式計數器在每個section重置
\counterwithin{equation}{section}
%10 字體設定 - 優化以減少警告
\IfFontExistsTF{Times New Roman}{%
  \setmainfont{Times New Roman}%
}{%
  \setmainfont{TeX Gyre Termes}%
}

%11 標題設定
\setlength{\droptitle}{-1in} % 上移標題1in
\title{\text{4.Volumetric Lattice Boltzmann Models in General Curvature}}
\author{Chen Peng Chung}
\setcounter{section}{0}
% 12调整 subsection 的间距
\titlespacing*{\section}
{0pt}                    % 左边距
{0.0in}                 % 标题前间距
{0.0em}                  % 标题后间距
\titlespacing*{\subsection}
{0pt}                    % 左边距
{-0.05in}                % 标题前间距
{0.0em}                  % 标题后间距
% 调整 subsubsection 的间距  
\titlespacing*{\subsubsection}
{0pt}                    % 左边距
{0.0in}                 % 标题前间距
{0.0em}                  % 标题后間距
%13.

% ========== Monokai Light 配色方案(淺灰背景)==========
\definecolor{monokailight-bg}{RGB}{240,240,240}      % 淺灰背景 #F0F0F0
\definecolor{monokailight-fg}{RGB}{39,40,34}         % 深色文字 #272822
\definecolor{monokailight-comment}{RGB}{117,113,94}  % 註解 #75715E
\definecolor{monokailight-string}{RGB}{152,118,24}   % 字串 #987618
\definecolor{monokailight-keyword}{RGB}{244,0,95}    % 關鍵字 #F4005F
\definecolor{monokailight-function}{RGB}{121,162,0}  % 函數 #79A200
\definecolor{monokailight-number}{RGB}{137,89,168}   % 數字 #8959A8
\definecolor{monokailight-type}{RGB}{0,129,152}      % 型別 #008198
\definecolor{monokailight-operator}{RGB}{244,0,95}   % 運算符 #F4005F
\definecolor{monokailight-border}{RGB}{210,210,210}  % 邊框 #D2D2D2

% ========== CUDA 語言定義 ==========
\lstdefinelanguage{CUDA}{
    language=C++,
    morekeywords={__global__, __device__, __host__, __shared__, 
                  __constant__, __syncthreads, dim3, cudaMalloc,
                  cudaMemcpy, cudaFree, cudaMemcpyHostToDevice,
                  cudaMemcpyDeviceToHost, cudaDeviceSynchronize},
    % 型別關鍵字
    morekeywords=[2]{double, float, int, char, void, size_t,
                     uint, uint2, uint3, uint4,
                     int2, int3, int4,
                     float2, float3, float4,
                     double2, double3, double4},
    % CUDA 內建變數
    morekeywords=[3]{blockIdx, threadIdx, blockDim, gridDim,
                     warpSize},
    sensitive=true
}

% ========== Monokai Light 樣式 ==========
\lstdefinestyle{monokailight}{
    language=CUDA,
    backgroundcolor=\color{monokailight-bg},
    basicstyle=\color{monokailight-fg}\ttfamily\footnotesize,
    commentstyle=\color{monokailight-comment}\itshape,
    keywordstyle=\color{monokailight-keyword}\bfseries,
    keywordstyle=[2]\color{monokailight-type}\bfseries,      % 型別
    keywordstyle=[3]\color{monokailight-function}\bfseries,  % CUDA 變數
    stringstyle=\color{monokailight-string},
    numberstyle=\tiny\color{monokailight-comment},
    % 數字顏色
    literate=
        {0}{{{\color{monokailight-number}0}}}1
        {1}{{{\color{monokailight-number}1}}}1
        {2}{{{\color{monokailight-number}2}}}1
        {3}{{{\color{monokailight-number}3}}}1
        {4}{{{\color{monokailight-number}4}}}1
        {5}{{{\color{monokailight-number}5}}}1
        {6}{{{\color{monokailight-number}6}}}1
        {7}{{{\color{monokailight-number}7}}}1
        {8}{{{\color{monokailight-number}8}}}1
        {9}{{{\color{monokailight-number}9}}}1,
    % 版面設定
    breaklines=true,
    breakatwhitespace=false,
    captionpos=b,
    keepspaces=true,
    numbers=left,
    numbersep=8pt,
    showspaces=false,
    showstringspaces=false,
    showtabs=false,
    tabsize=4,
    % 框線設定
    frame=single,
    rulecolor=\color{monokailight-border},
    framerule=0.8pt,
    % 其他設定
    columns=flexible,
    escapeinside={(*@}{@*)},
    xleftmargin=2em,
    xrightmargin=0.5em,
    framexleftmargin=1.5em
}

% ========== 設為預設樣式 ==========
\lstset{style=monokailight}
%14.
% ========== 表格樣式設定 ==========
\setlength{\arrayrulewidth}{0.5pt}  % 表格線條粗細
\renewcommand{\arraystretch}{1.3}   % 行高
%15listing 設定:
\lstset{
  language=C++,
  basicstyle=\ttfamily\small,
  numbers=left,
  numberstyle=\tiny,
  stepnumber=1,
  numbersep=8pt,
%
  keepspaces=true,      % ★保留空白
  showspaces=false,
  showstringspaces=false,
  showtabs=false,
%
  tabsize=2,            % ★tab 視為 2 個空白(可改 4)
  breaklines=true,
  breakatwhitespace=false,
%
  frame=single,
  captionpos=b
}



\begin{document}
\maketitle
\tableofcontents
\newpage
\section{Gauss-Hermit quadrature rule }
This section shows the process of extending the 1-dimensional n-order Gauss-Hermite quadrature to the 3-dimensional n-order Gauss-Hermite quadrature rule. 
The equation below is the basic quadrature rule for evaluating integrals.
\begin{equation}\begin{aligned}
    \int_{-\infty}^{\infty} dr \omega(r) \mathrm{P}^{2n-1}(r )= \sum_{i = 1}^{n} w_{i}  \mathrm{P}^{2n-1}(r_{i})
\end{aligned}\end{equation}
The equation above is the one-dimensional n-order Gauss-Hermite quadrature rule, where the function $\mathrm{P}^{2n-1}$ is any (2n-1)-order polynomial,
so the rule applies to any (2n-1)-order polynomial. The one-dimensional n-order Gauss-Hermite quadrature rule
 can be extended to any function whose order is smaller than 2n-1. The list below provides the information needed to explain the equation above.
\begin{table}[H]
  \centering
  \begin{tabular}{|c|c|c|}
    \hline
    Generation function & $\omega(r)$ & $ \frac{1}{\sqrt{2\pi}} \exp(-\frac{r^{2}}{2})$ \\ \hline
    Nodes & $r_{i}$ & The roots of 1 dimension n orders Hermite polynomial $\mathrm{H}^{(n)}(r)$ \\ \hline
    Weights & $w_{i}$ & $ \frac{n!}{\left(n\mathrm{H}^{(n-1)}(r)\right)^{2}}$ \\ \hline
  \end{tabular}
\end{table} 
\noindent The generation function of one dimension n orders Hermite polynomial is $\omega(x)$, and weight function of 3 dimensions n orders Hermite polynomial is $\omega(\vec{r})$ .
\begin{equation}\begin{aligned}
  \omega(x) = \frac{1}{\sqrt{2 \pi}} e^{\frac{-x^{2}}{2}} \\[1.5ex]
  \omega(\vec{r}) = \frac{1}{\sqrt{2 \pi}} e^{\frac{-\vec{r}\cdot \vec{r}}{2}}
\end{aligned}\end{equation}
The variables $r_{i}$ are the roots of the one-dimensional n-order Hermite polynomial. Note that an n-order Hermite polynomial has n roots, i.e., the order is equal to the number of roots.
\vspace{-2.5ex}
\paragraph{3-dimensional n-order Hermite polynomials}
\noindent The definition of the 3-dimensional n-order Hermite polynomials is given below:
\begin{equation}\begin{aligned}
  \overleftrightarrow{H}^{(n)}(\vec{r}) \equiv (-1)^{n}  \frac{1}{\omega(\vec{r}) } \vec{\nabla}^{(n)} \omega(\vec{r})
\end{aligned}\end{equation}
\noindent The symbol $\vec{\nabla }$ denotes the gradient of the weight function. An n-order Hermite polynomial is an n-order tensor field. The order of the tensor field is equal to the number of gradient operators in the Hermite formula.
For example, the 2-dimensional second-order Hermite polynomial:
\begin{equation}\begin{aligned}
  \overleftrightarrow{H}^{(2)} (x,y) &= H^{(2)}_{xx} \vec{e}_{x}\vec{e}_{x} + H^{(2)}_{xy} \vec{e}_{x}\vec{e}_{y} + H^{(2)}_{yy} \vec{e}_{y}\vec{e}_{y} \\[1.5ex]
  &= \left(x^{2}-1\right) \vec{e}_{x}\vec{e}_{x} + xy\ \vec{e}_{x}\vec{e}_{y} +\left(y^{2}-1\right)\vec{e}_{y}\vec{e}_{y} \\
\end{aligned}\end{equation}
For any $(2n-1)$-order polynomial, the one-dimensional n-order Gauss-Hermite quadrature rule can evaluate the integral whose domain is $\ -\infty\ $to$\ \infty\ $.
\vspace{-2.5ex}
\paragraph{Gauss-Hermite promoted to high dimensions}
\noindent Consider a volume integration over the whole three-dimensional space:
\begin{equation}
  \int_{\Omega} d^{3} r \omega(\vec{r}) \mathrm{P}^{\mathrm{N}}(\vec{r}) = 
  \int_{\Omega} d^{3} r  \omega(\vec{r})  \sum_{a+b+c \le N} x^{a} y^{b} z^{c} 
\end{equation}
Because the summation has a finite number of terms, the integration can be moved into the summation.
We have 
\begin{equation}\begin{aligned}
 \sum_{a+b+c \le N } \int_{\Omega}d^{3} r \omega(\vec{r} ) x^{a} y^{b} z^{c} 
\end{aligned}\end{equation}
\noindent Now we use the property of the exponential function:
\begin{equation*}
  \omega(\vec{r})= \omega(x)\omega(y)\omega(z)
\end{equation*}
\begin{equation}\begin{aligned}
 \sum_{a+b+c \le N } \int_{\Omega}d^{3} r \omega(\vec{r} ) x^{a} y^{b} z^{c} 
 = \sum_{a+b+c \le N } \int_{\Omega}dx \omega(x) x^{a} \int_{\Omega}dy \omega(y) y^{b} \int_{\Omega}dz \omega(z) z^{c} 
\end{aligned}\end{equation}
Using the one-dimensional n-order Gauss-Hermite quadrature rule for each integration along different directions, we have
\begin{equation}\begin{aligned}
\sum_{a+b+c \le N} \sum_{i = 1}^{n_a}\omega(x_{i}) x_{i}^{a} \sum_{i = 1}^{n_b} \omega(y_{i}) y_{i}^{b} \sum_{i = 1}^{n_c} \omega(z_{i}) z_{i}^{c} 
\end{aligned}
\end{equation}
\noindent where $(2n_{a}-1) > a$, $(2n_{b}-1) > b$, $(2n_{c}-1) > c$. The details are as follows:
\begin{equation}
\begin{aligned}
  \int_{\Omega}dx \omega(x) x^{a} = \sum_{i = 1}^{n_a}\omega(x_{i}) x_{i}^{a} \ \text{(1 dimension $n_a$ order G-H rule)}
\end{aligned}
\end{equation}
\noindent where $x_{i}$ is the root of the one-dimensional $n_a$-order Hermite function. Usually, for each one-dimensional Gauss-Hermite quadrature rule, we take the order to match the integrated function, i.e., the maximum order among the polynomial terms, so we can take the summation to handle the integration above:
\begin{equation}\begin{aligned}
 \sum_{a+b+c \le N } \int_{\Omega}d^{3} r \omega(\vec{r} ) x^{a} y^{b} z^{c} = \sum \sum_{i = 1}^{(N-1)/2} \omega(x_{i}) x_{i}^{a} \sum_{i = 1}^{(N-1)/2}\omega(y_{i}) y_{i}^{b} \sum_{i = 1}^{(N-1)/2}  \omega(z_{i}) z_{i}^{c} 
\end{aligned}\end{equation}
\noindent For example, calculate the integral $\int_{-\infty}^{\infty}dx \int_{-\infty}^{\infty}dy \omega(x,y) x^{2}y$
\begin{equation*}\begin{aligned}
  \int_{-\infty}^{\infty}dx \int_{-\infty}^{\infty}dy \omega(x,y) x^{2}y = \int_{-\infty}^{\infty}dx \int_{-\infty}^{\infty}dy \frac{1}{\sqrt{2\pi}} e^{\frac{-(x^{2}+y^{2})}{2}} x^{2}y = \left(w_{1}y_{1} + w_{2}y_{2}\right)\left(w_{1}x_{1}^{2} + w_{2}x_{2}^{2}\right) 
\end{aligned}\end{equation*}

\section{Momentum Loss of Distribution Function in Propagation}
In general curvilinear coordinates, macroscopic values (i.e., $\rho , \rho\vec{u}$) experience losses due to curvature and nonuniform properties. This paper shows the value below:
\begin{equation}\begin{aligned}\label{eq:NLBGK}
  \delta N_{\alpha}(\vec{q} , t) \equiv N_{\alpha}(\vec{q} , t) - N_{\alpha}^{\star} (\vec{q} - \vec{e}_{\alpha}  \delta t, t-1) 
\end{aligned}\end{equation} 
where q is the nondimensional position, and its original form is $\vec{q} = (q_{1} , q_{2} , q_{3})$. The value $N^{\star}$ represents post-collision distribution function.
We can review the Lattice Boltzmann Equation-BGK (LBGK) and see where the term appears in the equation.
\begin{equation}
  N_{\alpha}(\vec{q} + \vec{e}_{\alpha}\delta t, t+1) = N_{\alpha}(\vec{q} , t) + \Omega_{\alpha} +  \delta N_{\alpha}(\vec{q} ,t)
\end{equation}
For general orthogonal curvilinear coordinates in three dimensions, the loss of momentum and the loss of density are 0, so we have the relation:
\begin{equation}\begin{aligned}
  \sum_{\alpha=0}^{q-1} \delta N_{\alpha}(\vec{q} , t) = \sum_{\alpha=0}^{q-1}N_{\alpha}(\vec{q} , t) - N_{\alpha}^{\star} (\vec{q} - \vec{e}_{\alpha} \delta t , t-1) = \sum_{\alpha=0}^{q-1}  N_{\alpha}(\vec{q} , t) - \sum_{\alpha=0}^{q-1} N_{\alpha}^{\star} ( \vec{q} - \vec{e}_{\alpha} \delta t , t-1)
= 0 
\end{aligned}
\end{equation} 
But for momentum loss, we first consider its physical meaning as an "inertial force."
The value in equation \eqref{eq:NLBGK} 
is a modified term in the process of the streaming step. The reason why we have to consider
the loss is to achieve momentum conservation by adding this term.
\begin{definition}
  \begin{equation}\label{eq:Theta}
   \vec{\Theta}_{i}^{j} (\vec{q} + \vec{e}_{\alpha} \delta t , \vec{q} ) = \left[\vec{g}_{i}(\vec{q} +\vec{e}_{\alpha} \delta t) - \vec{g}_{i}(\vec{q})\right] \cdot \vec{g}^{\ j} (\vec{q})
  \end{equation}
\end{definition}
\noindent We call this value the "nondimensional change of curvature tangent vector," where $\vec{g}^{j}$ can be defined as below:
\begin{equation}\begin{aligned}
   \vec{g}^{1} &= \frac{\vec{g}_{2}\times \vec{g}_{3}}{(\vec{g}_{2}\times \vec{g}_{3}) \cdot \vec{g}_{1}} 
\end{aligned}\end{equation}
Substituting the vector into equation \eqref{eq:Theta}
  \begin{equation}
   \vec{\Theta}_{i}^{j} (\vec{q} + \vec{e}_{\alpha} \delta t , \vec{q} ) = \left[\vec{g}_{i}(\vec{q} +\vec{e}_{\alpha} \delta t) - \vec{g}_{i}(\vec{q})\right]  \cdot \frac{\vec{g}_{2}\times \vec{g}_{3}}{(\vec{g}_{2}\times \vec{g}_{3}) \cdot \vec{g}_{1}}
  \end{equation}
\begin{definition}
  \begin{equation}\begin{aligned}
    &\vec{M}_{\mathrm{Loss}} (\vec{q},t) \equiv
     \sum_{\alpha = 1}^{q} \left[N_{\alpha}(\vec{q} ,t)\,\vec{e}_{\alpha}(\vec{q}) - N_{\alpha}^{\star}(\vec{q} - \vec{e}_{\alpha} \delta t, t-1)\,\vec{e}_{\alpha}(\vec{q} - \vec{e}_{\alpha} \delta t )\right]\\[1.5ex]
    &\vec{M}_{\mathrm{Loss}} (\vec{q} + \vec{e}_{\alpha} \delta t ,t+1) \equiv
     \sum_{\alpha = 1}^{q} \left[N_{\alpha}(\vec{q} + \vec{e}_{\alpha} \delta t ,t+1)\,\vec{e}_{\alpha}(\vec{q} + \vec{e}_{\alpha} \delta t ) - N_{\alpha}^{\star}(\vec{q},t)\,\vec{e}_{\alpha}(\vec{q})\right]\\[1.5ex]
  \end{aligned}\end{equation}
\end{definition}
\noindent For curvilinear coordinates, the loss of momentum always exists in the streaming step.
The basic reason for the loss of macroscopic value is:
\begin{equation}
  N_{\alpha} (\vec{q} , t) \neq N_{\alpha}^{\star}(\vec{q} - \vec{e}_{\alpha} \delta t , t - 1)
\end{equation}
The pre-streaming distribution function is different from the post-collision function, but the sum of distributions along the discrete velocity space is the same for the two function sets. We can see that density cannot be lost in the streaming step in curvilinear coordinates.
But this paper defines two values for inner and outer momentum loss at the point $(q,t)$, as shown, to prepare for proving the inertial force.
\begin{definition}
  \begin{equation}\begin{aligned}
    & \mathcal{J}(\vec{q} ) \,\vec{\chi}^{\mathrm{in}}(\vec{q} ,t) \equiv
    -\sum_{\alpha = 1}^{q} N_{\alpha}(\vec{q},t)\cdot \left(\vec{e}_{\alpha}(\vec{q}) - \vec{e}_{\alpha}(\vec{q} - \vec{e}_{\alpha} \delta t ) \right)\\
     & \mathcal{J}(\vec{q}) \,\vec{\chi}^{ \mathrm{out}}(\vec{q},t) \equiv
    -\sum_{\alpha = 1}^{q} \left(\vec{e}_{\alpha}(\vec{q} + \vec{e}_{\alpha} \delta t ) -\vec{e}_{\alpha}(\vec{q})\right)\cdot N_{\alpha}^{\star}(\vec{q},t)
  \end{aligned}\end{equation}
\end{definition}
\noindent where for the discrete particle velocity set  
\begin{equation}\begin{aligned}
 \vec{\tilde{c}}_{\alpha}(\vec{q}) = \vec{e}_{\alpha}(\vec{q} ) = \sum_{i = 1}^{3} (\vec{c}_{\alpha}\cdot \vec{g}_{i}) \frac{\Delta x}{\Delta t} = c_{\alpha}^{i} \,\vec{g}_{i} \,\frac{\Delta x}{\Delta t} 
\end{aligned}\end{equation}
\noindent In this paper, the momentum loss and its definition can be separated into two directions at the point $(\vec{q},t)$, i.e., inner and outer.
The difference from the true "momentum loss of the propagation" is that it takes the post-collision distribution function at $(\vec{q} - \vec{e}_{\alpha} \delta t ,t-1)$: $N_{\alpha}^{\star}(\vec{q} - \vec{e}_{\alpha} \delta t ,t-1)$;
and it takes the pre-collision function at $(\vec{q},t)$: $N_{\alpha}(\vec{q},t)$ for the curvilinear normalized discrete particle velocity set: 
$\vec{\tilde{e}}_{\alpha}(\vec{q} - \vec{e}_{\alpha} \delta t ,t-1)$ and $\vec{\tilde{e}}_{\alpha}(\vec{q},t)$.
The author takes the constraint for momentum in the process of propagation as shown:
\begin{equation}\begin{aligned}
&\sum_{\alpha = 1}^{q} \vec{e}_{\alpha}(\vec{q},t)\,\delta N_{\alpha}(\vec{q} ) = \sum_{\alpha = 1}^{q}  \vec{e}_{\alpha}(\vec{q} )\,N_{\alpha}(\vec{q})- \sum_{\alpha = 1}^{q}\vec{e}_{\alpha}(\vec{q})\, N_{\alpha}^{\star}(\vec{q} - \vec{e}_{\alpha} \delta t ,t-1) = \mathcal{J} \cdot \frac{\vec{\chi}^{\mathrm{in}}(\vec{q} ,t)+ \vec{\chi}^{\mathrm{out}}(\vec{q} ,t)}{2} \\[1.5ex]
& = \mathcal{J} \cdot \frac{
-\sum_{\alpha = 1}^{q} N_{\alpha}(\vec{q})\left(\vec{e}_{\alpha}(\vec{q})- \vec{e}_{\alpha}(\vec{q} - \vec{e}_{\alpha} \delta t )\right)
-\sum_{\alpha = 1}^{q} N_{\alpha}^{\star}(\vec{q} ,t) \left(\vec{e}_{\alpha}(\vec{q} + \vec{e}_{\alpha} \delta t )- \vec{e}_{\alpha}(\vec{q})\right)
}{2}
\end{aligned}
\end{equation}


\begin{definition}
  Velocity-space discretized force field -- specific case in curvilinear coordinates
  \begin{equation}\label{eq:problem}
    F^{i} (\vec{q} , t) = \frac{\vec{\chi}^{\mathrm{in}}(\vec{q} ,t) + \vec{\chi}^{\mathrm{out}}(\vec{q} ,t)}{2} \cdot \vec{g}^{\ i}(\vec{q}) 
  \end{equation}
\end{definition}
\noindent 
The macroscopic force is discretized in particle-velocity space, and this represents an inertial force produced in a curvilinear-coordinate computational domain.
We now review the statement of the whole process.
First, we impose two "symbols" for inner and outer momentum loss; then define the form of the velocity space-discretized force field, and finally prove the inertial force using the momentum loss.
However, the following expression appears to treat a symbol as a "variable":
$$\left(\sum_{\alpha = 1}^{q}\left(\vec{e}_{\alpha}(\vec{q} ) - \vec{e}_{\alpha}(\vec{q} - \vec{e}_{\alpha} \delta t )\right)\cdot N_{\alpha}(\vec{q} ,t)\right)$$
which is not exactly the momentum loss. The author then uses the true constraint to define the inertial force and maps that constraint to a velocity-space discretized force field.
\noindent The specific issues are:
\begin{enumerate}
  \item The definition of any force (including a velocity-discrete force field) should take a form like
  \begin{equation}\begin{aligned}
    &F^{i} (\vec{q} ,t) \equiv \frac{\text{Momentum Loss}}{\Delta t} \cdot \vec{e}_{i} \text{ (gerneral definition)}\\[1.5ex]
    &\mathcal{J} (\vec{q}) F^{i} (\vec{q} ,t) \equiv \frac{\text{Momentum Loss}}{ \Delta t} \cdot \vec{e}_{i} \frac{1}{|\vec{g}_{i}|}
  \end{aligned}\end{equation}
  where $\hat{g}_{\,i}=\dfrac{\vec{g}_{\,i}}{|\vec{g}_{\,i}|}$ is the unit vector in the $i$-th coordinate direction. This form uses a time difference (division by $\Delta t$) and projects onto the unit vector to obtain the component. The author's original expression omits the time scaling and the projection, which complicates interpretation.
  I thought that he set the time step is 1. 
  \item The second issue concerns the use of $\vec{g}^{\,i}$ in equation \eqref{eq:problem}. If $\vec{g}^{\,i}$ is intended to extract a component, it should be the unit vector $\hat{g}^{\,i}$. Multiplying by the magnitude $|\vec{g}^{\,i}|$ changes the scaling and can remove the proper coordinate-transformation property of the discrete force field. If a Jacobian or metric factor is required, it should be introduced explicitly and justified.
\end{enumerate}
\begin{derivation} Derivation of the equation
  \begin{equation}\begin{aligned}
    \mathcal{J} (\vec{q} )  F^{\ i}(\vec{q} , t) = \frac{-1}{2} (\sum_{\alpha = 1}^{q} N_{\alpha}(\vec{q} ,t)\left(c_{\alpha}^{i} \Theta_{i}^{j} (\vec{q} - \vec{e}_{\alpha}\delta t , \vec{q})\right)
    + \sum_{\alpha = 1}^{q}N_{\alpha}^{\star}(\vec{q},t) \left(c_{\alpha}^{i} \Theta_{i}^{j} (\vec{q} + \vec{e}_{\alpha}\delta t , \vec{q} \right))
  \end{aligned}\end{equation}
\noindent The principle of the derivation can be separated into three parts:
  \begin{enumerate} 
    \item $\sum_{\alpha=1}^{q}\vec{e}_{\alpha}(\vec{q})\ \delta N_{\alpha}(\vec{q},t)\equiv$ Momentum Loss of the Streaming Step 
    \item $\sum_{\alpha=1}^{q}\vec{e}_{\alpha}(\vec{q})\ \delta N_{\alpha}(\vec{q},t)\equiv \mathcal{J} (\vec{q}) \cdot \frac{\vec{\chi}^{\ in } + \vec{\chi}^{\ out}}{2}$ (for Volumetric LBM)
    \item Velocity space discretized force field $\mathcal{J} (\vec{q} ) F^{\ i}\equiv \frac{\text{Momentum Loss}}{\Delta t}\cdot \vec{e}_{j} \frac{1}{|\vec{g}_{j}|}$ \\
  \end{enumerate}
  \begin{equation}\begin{aligned}
    &\mathcal{J} (\vec{q}) F^{1} = \sum_{\alpha = 1}^{q}\vec{e}_{\alpha}(\vec{q}) \delta  N_{\alpha}(\vec{q} , t) \cdot \vec{g}^{\ 1} = \sum_{\alpha = 1}^{q}\vec{e}_{\alpha}(\vec{q}) \delta  N_{\alpha}(\vec{q} , t)  \cdot \frac{\vec{g_{2}} \cross \vec{g_{3}}}{(\vec{g_{2}} \cross \vec{g_{3}})\cdot \vec{g}_{1}}\\[1.5ex] 
    &= \frac{-1}{2} \left(\sum_{\alpha = 1}^{q} N_{\alpha}(\vec{q} ,t)\left(\vec{e}_{\alpha}(\vec{q}) - \vec{e}_{\alpha}(\vec{q} - \vec{e}_{\alpha} \delta t ) \right) + \sum_{\alpha = 1}^{q}N_{\alpha}^{\star}(\vec{q},t) \left(\vec{e}_{\alpha}(\vec{q} + \vec{e}_{\alpha} \delta t) - \vec{e}_{\alpha}(\vec{q}) \right)\right)\cdot \frac{\vec{g_{2}} \cross \vec{g_{3}}}{(\vec{g_{2}} \cross \vec{g_{3}})\cdot \vec{g}_{1}}\\[1.5ex] 
    &= \sum_{\alpha = 1}^{q}\vec{e}_{\alpha}(\vec{q}) \delta  N_{\alpha}(\vec{q} , t) \cdot \vec{e}_{j} \frac{1}{|\vec{g}_{j}|}\\[1.5ex] 
    &= \frac{-1}{2} \left(\sum_{\alpha = 1}^{q} N_{\alpha}(\vec{q} ,t)\left(\vec{e}_{\alpha}(\vec{q}) - \vec{e}_{\alpha}(\vec{q} - \vec{e}_{\alpha} \delta t ) \right)+ \sum_{\alpha = 1}^{q}N_{\alpha}^{\star}(\vec{q},t) \left(\vec{e}_{\alpha}(\vec{q} + \vec{e}_{\alpha} \delta t ) - \vec{e}_{\alpha}(\vec{q}) \right)\right)\cdot \vec{e}_{j} \frac{1}{|\vec{g}_{j}|}\\[1.5ex] 
    &= \frac{-1}{2} (\sum_{\alpha = 1}^{q} N_{\alpha}(\vec{q} ,t)\left(\vec{e}_{\alpha}(\vec{q})\cdot \vec{e}_{j} \frac{1}{|\vec{g}_{j}|}  - \vec{e}_{\alpha}(\vec{q} - \vec{e}_{\alpha} \delta t )\cdot \vec{e}_{j} \frac{1}{|\vec{g}_{j}|} \right)\\[1.5ex] 
    &+ \sum_{\alpha = 1}^{q}N_{\alpha}^{\star}(\vec{q},t) \left(\vec{e}_{\alpha}(\vec{q} + \vec{e}_{\alpha} \delta t )\cdot \vec{e}_{j} \frac{1}{|\vec{g}_{j}|}- \vec{e}_{\alpha}(\vec{q})\cdot \vec{e}_{j} \frac{1}{|\vec{g}_{j}|}\right))\\[1.5ex]
    &= \frac{-1}{2}  \left(\sum_{\alpha = 1}^{q} N_{\alpha}(\vec{q} ,t)\left(c_{\alpha}^{j} \cdot \frac{|\vec{g}_{j}|(\vec{q}) - |\vec{g}_{j}|(\vec{q} - \vec{e}_{\alpha}\delta t))}{|\vec{g}_{j}|} \right)
    + \sum_{\alpha = 1}^{q}N_{\alpha}^{\star}(\vec{q},t) \left(c_{\alpha}^{j} \cdot \frac{|\vec{g}_{j}|(\vec{q} +  \vec{e}_{\alpha}\delta t) - |\vec{g}_{j}|(\vec{q} ))}{|\vec{g}_{j}|} \right)\right)\\[1.5ex]
    &= \frac{-1}{2} \left(\sum_{\alpha = 1}^{q} N_{\alpha}(\vec{q} ,t)\left(c_{\alpha}^{j} \cdot \frac{{\color{red}|\vec{g}_{j}|(\vec{q}-\vec{e}_{\alpha}\delta t) - |\vec{g}_{j}|(\vec{q})}}{|\vec{g}_{j}|} \right)
    + \sum_{\alpha = 1}^{q}N_{\alpha}^{\star}(\vec{q},t) \left(c_{\alpha}^{j} \cdot \frac{|\vec{g}_{j}|(\vec{q} +  \vec{e}_{\alpha}\delta t) - |\vec{g}_{j}|(\vec{q} ))}{|\vec{g}_{j}|} \right)\right)\\[1.5ex]
    \end{aligned}\end{equation}
    \begin{equation}\begin{aligned}
    = \frac{-1}{2}  \left(\sum_{\alpha = 1}^{q} N_{\alpha}(\vec{q} ,t)\left(c_{\alpha}^{i} \Theta_{i}^{j} (\vec{q} - \vec{e}_{\alpha}\delta t , \vec{q})\right)
    + \sum_{\alpha = 1}^{q}N_{\alpha}^{\star}(\vec{q},t) \left(c_{\alpha}^{i} \Theta_{i}^{j} (\vec{q} + \vec{e}_{\alpha}\delta t , \vec{q} \right)\right)
    \end{aligned}\end{equation}
\end{derivation}

\subsection{Basic Equation for Curvilinear Coordinate}
 
\noindent These are basic formula for standard Latticve Boltzmann Method to achieve conservation, we can use them to recover Naver-Stokes Equation through Chapman Enskog expansion.
\begin{equation}\begin{aligned}
&\delta N_{\alpha}(\vec{q} , t) = w_{i}\left(\frac{c_{\alpha}^{i} F^{i}}{c_{s}^{2}} + \frac{c_{\alpha}^{i}c_{\alpha}^{j}-c_s^{2}\delta ^{i j }}{2c_{s}^{4}}\right)\\[1.5ex]
&\sum_{\alpha = 1}^{q} f^{eq}(\vec{q} , t) = \rho \\[1.5ex]
&\sum_{\alpha = 1}^{q} f^{eq}(\vec{q} , t) c_{\alpha}^{i} = \rho u^{i}\\[1.5ex]
&\sum_{\alpha = 1}^{q} f^{eq}(\vec{q} , t) c_{\alpha}^{i}c_{\alpha}^{j} = \rho c_s^{2} g^{ij} + \rho (u^{i} + \frac{1}{2}\frac{F^{i}}{\rho})(u^{j} + \frac{1}{2}\frac{F^{j}}{\rho})
\end{aligned}
\end{equation}
\end{document}



