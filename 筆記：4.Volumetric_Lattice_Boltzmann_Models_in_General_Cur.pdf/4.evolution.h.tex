\documentclass[12pt]{article}
\usepackage[a4paper,margin=2cm]{geometry} % 明確設定四邊
\usepackage{fontspec} % 字體設定
\usepackage[english]{babel} % English document language
% Note: xeCJK and CJK font setup removed for English environment
\usepackage{setspace} % 設定行距
\linespread{1.2}
\usepackage{titling} % 預設標題下移0.6in
\usepackage{enumitem}
\usepackage{amsmath} % 數學方程式
\usepackage{graphicx} %圖片
\usepackage{float} % 在導言區,讓圖片強制插在原地
\usepackage{xcolor} %字體加入顏色
\usepackage{listings}
\usepackage{physics} % 物理符號
\usepackage{wrapfig} % 文字環繞圖
\usepackage{array} % 表格對齊控制
\usepackage{caption}
\usepackage{amsthm}
\usepackage{tabularx}
\usepackage{booktabs}
\usepackage{multirow}
\usepackage{etoolbox}
\usepackage{titlesec}
\usepackage{tocloft}
\usepackage{hyperref}
\usepackage{makecell}
\usepackage{xcolor}
\usepackage{colortbl}
\hypersetup{
  colorlinks=true,
  linkcolor=blue,
  urlcolor=blue,
  citecolor=blue,
  linktoc=all
}
\usepackage{tcolorbox}
\tcbuselibrary{skins,breakable}
\usepackage{tikz}
\usetikzlibrary{arrows.meta,positioning,shapes.geometric}
\tcbset{
  highlightblock/.style={
    enhanced,
    breakable,
    colback=gray!8,
    colframe=gray!60,
    boxrule=0.4pt,
    arc=0pt,
    boxsep=0pt,
    left=4pt,
    right=4pt,
    top=4pt,
    bottom=4pt,
    before skip=0pt,
    after skip=0pt
  }
}
\usepackage{longtable}

%1 註記行距設定
\setlist[itemize]{itemsep=1.2pt, parsep=0pt, topsep=1pt}
\setlist[enumerate]{itemsep=1.2pt, parsep=0pt, topsep=1pt}
%2 圖表標題設定
\captionsetup{
    labelfont={footnotesize,bf},    % 標籤:小字體+粗體
    skip=10pt                        % 標題與圖表的間距
}
%3 表格間距設定
\setlength{\tabcolsep}{8pt} % 列間距(增加)
\renewcommand{\arraystretch}{1.4} % 行間距(增加)
%4 腳註設定
\usepackage[bottom,hang]{footmisc} % 腳註置於頁面底部,懸掛縮排
\setlength{\footnotesep}{10pt} % 腳註之間的間距(增加)
\setlength{\skip\footins}{12pt plus 5pt minus 2pt} % 正文與腳註之間的間距(增加)
\setlength{\footnotemargin}{1.5em} % 腳註標號與文字的間距(增加)
\renewcommand{\footnoterule}{\vspace*{-3pt}\hrule width 0.4\columnwidth height 0.4pt\vspace*{3pt}} % 腳註分隔線
%5 使用直立字體的定理樣式
\newtheoremstyle{upright}
  {6pt}{6pt}  % 定理環境前後間距(增加)
  {\normalfont}% 使用正常字體,不使用斜體
  {0pt}{\bfseries}{.}{0.5em}{}
%6 定理環境定義
\theoremstyle{upright}
\newtheorem{definition}{Definition}[section]
\newtheorem{theorem}{Theorem}[section]
\newtheorem{lemma}{Lemma}[section]
\newtheorem{corollary}{Corollary}[section]
\newtheorem{example}{Example}[section]
% 重新正確定義 remark 環境 - 確保完全靠左對齊
\makeatletter
\@ifundefined{c@remark}{}{\renewcommand{\theremark}{}}
\newenvironment{remark}{%
  \par\vspace{0.0\baselineskip}%
  \begingroup
  \setlength{\parindent}{0pt}%
  \setlength{\leftskip}{0pt}%
  \noindent\textbf{Remark:}\\
  \ignorespaces
  \setlist[enumerate]{itemsep=0pt, parsep=0pt, topsep=0.0pt, leftmargin=1.5em}%
  \setlist[itemize]{itemsep=0pt, parsep=0pt, topsep=0.0pt, leftmargin=1.5em}%
}{%
  \endgroup
  \par\vspace{0.5\baselineskip}
}
\makeatother
% define a simple derivation environment for derivations/proofs
\newenvironment{derivation}{\par\medskip\noindent\textbf{Derivation.}\ }{\par\medskip}
%7 手動定義中文數字(不含標點符號)
\newcommand{\chinese}[1]{%
  \ifcase#1 零\or 一\or 二\or 三\or 四\or 五\or 六\or 七\or 八\or 九\or 十\or
  十一\or 十二\or 十三\or 十四\or 十五\or 十六\or 十七\or 十八\or 十九\or 二十\fi
}

%8 重新定義章節編號格式 - 使用阿拉伯數字以符合英文文件與常見期刊格式
% (原先使用中文數字表示,改回標準阿拉伯編號)
\renewcommand{\thesection}{\arabic{section}}
\renewcommand{\thesubsection}{\arabic{section}.\arabic{subsection}}
\renewcommand{\thesubsubsection}{\arabic{section}.\arabic{subsection}.\arabic{subsubsection}}
\renewcommand{\theequation}{\arabic{section}.\arabic{equation}}
\renewcommand{\thefigure}{\arabic{section}.\arabic{figure}}
\renewcommand{\thedefinition}{\thesection.\arabic{definition}}
\renewcommand{\thetheorem}{\thesection.\arabic{theorem}}
\renewcommand{\theexample}{\thesection.\arabic{example}}

% 为不同级别的标题增加编号后的空格
\setlength{\cftsecnumwidth}{2.5em}  % section 编号宽度
\setlength{\cftsubsecnumwidth}{3.5em}  % subsection 编号宽度
\setlength{\cftsubsubsecnumwidth}{4.5em}  % subsubsection 编号宽度

% 添加点线(可选)
\renewcommand{\cftsubsubsecleader}{\cftdotfill{\cftdotsep}}
\renewcommand{\cftsubsecleader}{\cftdotfill{\cftdotsep}}
\renewcommand{\cftsecleader}{\cftdotfill{\cftdotsep}}
%9 讓方程式計數器在每個section重置
\counterwithin{equation}{section}
%10 字體設定 - 優化以減少警告
\IfFontExistsTF{Times New Roman}{%
  \setmainfont{Times New Roman}%
}{%
  \setmainfont{TeX Gyre Termes}%
}

%11 標題設定
\setlength{\droptitle}{-1in} % 上移標題1in
\title{\text{4.Volumetric Lattice Boltzmann Models in General Curvature}}
\author{Chen Peng Chung}
\setcounter{section}{0}
% 12调整 subsection 的间距
\titlespacing*{\section}
{0pt}                    % 左边距
{0.0in}                 % 标题前间距
{0.0em}                  % 标题后间距
\titlespacing*{\subsection}
{0pt}                    % 左边距
{-0.05in}                % 标题前间距
{0.0em}                  % 标题后间距
% 调整 subsubsection 的间距  
\titlespacing*{\subsubsection}
{0pt}                    % 左边距
{0.0in}                 % 标题前间距
{0.0em}                  % 标题后間距
%13.

% ========== Monokai Light 配色方案(淺灰背景)==========
\definecolor{monokailight-bg}{RGB}{240,240,240}      % 淺灰背景 #F0F0F0
\definecolor{monokailight-fg}{RGB}{39,40,34}         % 深色文字 #272822
\definecolor{monokailight-comment}{RGB}{117,113,94}  % 註解 #75715E
\definecolor{monokailight-string}{RGB}{152,118,24}   % 字串 #987618
\definecolor{monokailight-keyword}{RGB}{244,0,95}    % 關鍵字 #F4005F
\definecolor{monokailight-function}{RGB}{121,162,0}  % 函數 #79A200
\definecolor{monokailight-number}{RGB}{137,89,168}   % 數字 #8959A8
\definecolor{monokailight-type}{RGB}{0,129,152}      % 型別 #008198
\definecolor{monokailight-operator}{RGB}{244,0,95}   % 運算符 #F4005F
\definecolor{monokailight-border}{RGB}{210,210,210}  % 邊框 #D2D2D2

% ========== CUDA 語言定義 ==========
\lstdefinelanguage{CUDA}{
    language=C++,
    morekeywords={__global__, __device__, __host__, __shared__, 
                  __constant__, __syncthreads, dim3, cudaMalloc,
                  cudaMemcpy, cudaFree, cudaMemcpyHostToDevice,
                  cudaMemcpyDeviceToHost, cudaDeviceSynchronize},
    % 型別關鍵字
    morekeywords=[2]{double, float, int, char, void, size_t,
                     uint, uint2, uint3, uint4,
                     int2, int3, int4,
                     float2, float3, float4,
                     double2, double3, double4},
    % CUDA 內建變數
    morekeywords=[3]{blockIdx, threadIdx, blockDim, gridDim,
                     warpSize},
    sensitive=true
}

% ========== Monokai Light 樣式 ==========
\lstdefinestyle{monokailight}{
    language=CUDA,
    backgroundcolor=\color{monokailight-bg},
    basicstyle=\color{monokailight-fg}\ttfamily\footnotesize,
    commentstyle=\color{monokailight-comment}\itshape,
    keywordstyle=\color{monokailight-keyword}\bfseries,
    keywordstyle=[2]\color{monokailight-type}\bfseries,      % 型別
    keywordstyle=[3]\color{monokailight-function}\bfseries,  % CUDA 變數
    stringstyle=\color{monokailight-string},
    numberstyle=\tiny\color{monokailight-comment},
    % 數字顏色
    literate=
        {0}{{{\color{monokailight-number}0}}}1
        {1}{{{\color{monokailight-number}1}}}1
        {2}{{{\color{monokailight-number}2}}}1
        {3}{{{\color{monokailight-number}3}}}1
        {4}{{{\color{monokailight-number}4}}}1
        {5}{{{\color{monokailight-number}5}}}1
        {6}{{{\color{monokailight-number}6}}}1
        {7}{{{\color{monokailight-number}7}}}1
        {8}{{{\color{monokailight-number}8}}}1
        {9}{{{\color{monokailight-number}9}}}1,
    % 版面設定
    breaklines=true,
    breakatwhitespace=false,
    captionpos=b,
    keepspaces=true,
    numbers=left,
    numbersep=8pt,
    showspaces=false,
    showstringspaces=false,
    showtabs=false,
    tabsize=4,
    % 框線設定
    frame=single,
    rulecolor=\color{monokailight-border},
    framerule=0.8pt,
    % 其他設定
    columns=flexible,
    escapeinside={(*@}{@*)},
    xleftmargin=2em,
    xrightmargin=0.5em,
    framexleftmargin=1.5em
}

% ========== 設為預設樣式 ==========
\lstset{style=monokailight}
%14.
% ========== 表格樣式設定 ==========
\setlength{\arrayrulewidth}{0.5pt}  % 表格線條粗細
\renewcommand{\arraystretch}{1.3}   % 行高
%15listing 設定:
\lstset{
  language=C++,
  basicstyle=\ttfamily\small,
  numbers=left,
  numberstyle=\tiny,
  stepnumber=1,
  numbersep=8pt,
%
  keepspaces=true,      % ★保留空白
  showspaces=false,
  showstringspaces=false,
  showtabs=false,
%
  tabsize=2,            % ★tab 視為 2 個空白(可改 4)
  breaklines=true,
  breakatwhitespace=false,
%
  frame=single,
  captionpos=b
}



\begin{document}
\maketitle
\tableofcontents
\newpage
\section{Gauss-Hermit quadrature rule }
This section would show the perocess the 1 dimension n orders Gauss-Hermite quadrature to 3 dimension n orders Gauss-Hermite quadratuere rule. 
Below the equation is the basic quadrture rule for solving integration.
\begin{equation}\begin{aligned}
    \int_{-\infty}^{\infty} dr \omega(r) \mathrm{P}^{2n-1}(r )= \sum_{i = 1}^{n} w_{i}  \mathrm{P}^{2n-1}(r_{i})
\end{aligned}\end{equation}
Above is the one dimension n orders Gauss-Hermite quadrature rule, where the funtion $\mathrm{P}^{2n-1}$ is any 2n-1 order polynomials, 
so the rule is used for any 2n-1 orders poilynomials. And 1 dimension n orders Gauss-Hermite quadrture rule
 can be promoted to any function which order is smaller than 2n-1. List below will show some unknown information to explain the equaition above.
\begin{table}[H]
  \centering
  \begin{tabular}{|c|c|c|}
    \hline
    Generation function & $\omega(r)$ & $ \frac{1}{\sqrt{2\pi}} \exp(-\frac{r^{2}}{2})$ \\ \hline
    Nodes & $r_{i}$ & The roots of 1 dimension n orders Hermite polynomial $\mathrm{H}^{(n)}(r)$ \\ \hline
    Weights & $w_{i}$ & $ \frac{n!}{\left(n\mathrm{H}^{(n-1)}(r)\right)^{2}}$ \\ \hline
  \end{tabular}
\end{table} 
\noindent The generation function of one dimension n orders Hermite polynomial is $\omega(x)$, and weight function of 3 dimensions n orders Hermite polynomial is $\omega(\vec{r})$ .
\begin{equation}\begin{aligned}
  \omega(x) = \frac{1}{\sqrt{2 \pi}} e^{\frac{-x^{2}}{2}} \\[1.5ex]
  \omega(\vec{r}) = \frac{1}{\sqrt{2 \pi}} e^{\frac{-\vec{r}\cdot \vec{r}}{2}}
\end{aligned}\end{equation}
The variables $r_{i}$ is the roots of 1 dimension n orders Hermite polyhnomnial. A thing you have to realize is n orders Hernite polynomial have n roots i.e. the order of n order Hernite polynomial is equal to number of roots. 


\section{Momentum Loss of Distribution Function in Propagation}
In general curvilinear coordinates, for macroscopic values i.e., $\rho , \rho\vec{u}$ exist the loss due to the curvature and uniform properties. And this paper shows the value below: 
\begin{equation}\begin{aligned}\label{eq:NLBGK}
  \delta N_{\alpha}(\vec{q} , t) \equiv N_{\alpha}(\vec{q} , t) - N_{\alpha}^{\star} (\vec{q} - \vec{e}_{\alpha}  \delta t, t-1) 
\end{aligned}\end{equation} 
where q is the non-dimension position, and the origin form of it is $\vec{q} = (q_{1} , q_{2} , q_{3})$. The value $N^{\star}$ represents post-collision distribution function.
We can review the Lattice Boltzman Equation-BGK (LBGK) and see the position of the value in the equation.
\begin{equation}
  N_{\alpha}(\vec{q} + \vec{e}_{\alpha}\delta t, t+1) = N_{\alpha}(\vec{q} , t) + \Omega_{\alpha} +  \delta N_{\alpha}(\vec{q} ,t)
\end{equation}
For general orthogonal curvilinear coordinate in three dimensions, we have the loss of the momentum and loss of the density is 0, so we have the relation: 
\begin{equation}\begin{aligned}
  &\sum_{\alpha=0}^{q-1} \delta N_{\alpha}(\vec{q} , t) = \sum_{\alpha=0}^{q-1}N_{\alpha}(\vec{q} , t) - N_{\alpha}^{\star} (\vec{q} - \vec{e}_{\alpha} \delta t , t-1)\\[1.5ex]
  &= \sum_{\alpha=0}^{q-1}  N_{\alpha}(\vec{q} , t) - \sum_{\alpha=0}^{q-1} N_{\alpha}^{\star} ( \vec{q} - \vec{e}_{\alpha} \delta t , t-1)
= 0 \end{aligned}
\end{equation} 
But for momentum loss, we have to take the physical meaning as "inertial force" at first. 
The value in equation \eqref{eq:NLBGK} 
is modified term in the process of the streaming step. The reason why we have to consider
the loss is to achieve momentum conservation through adding the term. 
\begin{definition}
  \begin{equation}\label{eq:Theta}
   \vec{\Theta}_{i}^{j} (\vec{q} + \vec{e}_{\alpha} \delta t , \vec{q} ) = \left[\vec{g}_{i}(\vec{q} +\vec{e}_{\alpha} \delta t) - \vec{g}_{i}(\vec{q})\right] \cdot \vec{g}^{\ j} (\vec{q})
  \end{equation}
\end{definition}
\noindent we call the value is "non-dimension change of curvature tangent vector", where $\vec{g}^{j}$ can be defined as below: 
\begin{equation}\begin{aligned}
   \vec{g}^{1} &= \frac{\vec{g}_{2}\times \vec{g}_{3}}{(\vec{g}_{2}\times \vec{g}_{3}) \cdot \vec{g}_{1}} 
\end{aligned}\end{equation}
Substitute the vector in equation \eqref{eq:Theta}
  \begin{equation}
   \vec{\Theta}_{i}^{j} (\vec{q} + \vec{e}_{\alpha} \delta t , \vec{q} ) = \left[\vec{g}_{i}(\vec{q} +\vec{e}_{\alpha} \delta t) - \vec{g}_{i}(\vec{q})\right]  \cdot \frac{\vec{g}_{2}\times \vec{g}_{3}}{(\vec{g}_{2}\times \vec{g}_{3}) \cdot \vec{g}_{1}}
  \end{equation}
\begin{definition}
  \begin{equation}\begin{aligned}
    &\vec{M}_{\mathrm{Loss}} (\vec{q},t) \equiv
     \sum_{\alpha = 1}^{q} \left[N_{\alpha}(\vec{q} ,t)\,\vec{e}_{\alpha}(\vec{q}) - N_{\alpha}^{\star}(\vec{q} - \vec{e}_{\alpha} \delta t, t-1)\,\vec{e}_{\alpha}(\vec{q} - \vec{e}_{\alpha} \delta t )\right]\\[1.5ex]
    &\vec{M}_{\mathrm{Loss}} (\vec{q} + \vec{e}_{\alpha} \delta t ,t+1) \equiv
     \sum_{\alpha = 1}^{q} \left[N_{\alpha}(\vec{q} + \vec{e}_{\alpha} \delta t ,t+1)\,\vec{e}_{\alpha}(\vec{q} + \vec{e}_{\alpha} \delta t ) - N_{\alpha}^{\star}(\vec{q},t)\,\vec{e}_{\alpha}(\vec{q})\right]\\[1.5ex]
  \end{aligned}\end{equation}
\end{definition}
\noindent For curvilinear coordinates, the loss of momentum always exists in the streaming step. The basic reason of the loss of macroscopic value is: 
\begin{equation}
  N_{\alpha} (\vec{q} , t) \neq N_{\alpha}^{\star}(\vec{q} - \vec{e}_{\alpha} \delta t , t - 1)
\end{equation}
But this paper defines two values about inner and outer at the point $(q,t)$ as shown: 
\begin{definition}
  \begin{equation}\begin{aligned}
    & \mathcal{J}(\vec{q} ) \,\vec{\chi}^{\mathrm{in}}(\vec{q} ,t) \equiv
    -\sum_{\alpha = 1}^{q} N_{\alpha}(\vec{q},t)\cdot \left(\vec{e}_{\alpha}(\vec{q}) - \vec{e}_{\alpha}(\vec{q} - \vec{e}_{\alpha} \delta t ) \right)\\[1.5ex]
     & \mathcal{J}(\vec{q}) \,\vec{\chi}^{ \mathrm{out}}(\vec{q},t) \equiv
    -\sum_{\alpha = 1}^{q} \left(\vec{e}_{\alpha}(\vec{q} + \vec{e}_{\alpha} \delta t ) -\vec{e}_{\alpha}(\vec{q})\right)\cdot N_{\alpha}^{\star}(\vec{q},t)\\[1.5ex]
  \end{aligned}\end{equation}
\end{definition}
\noindent where, for the discrete particle velocity set  
\begin{equation}\begin{aligned}
 \vec{\tilde{c}}_{\alpha}(\vec{q}) = \vec{e}_{\alpha}(\vec{q} ) = \sum_{i = 1}^{3} (\vec{c}_{\alpha}\cdot \vec{g}_{i}) \frac{\Delta x}{\Delta t} = c_{\alpha}^{i} \,\vec{g}_{i} \,\frac{\Delta x}{\Delta t} 
\end{aligned}\end{equation}
\noindent In this paper, the momentum loss and its definition can be separated into two directions about the point $(\vec{q},t)$, i.e., inner and outer.
The difference from the true "momentum loss of the propagation" is: take post-collision distribution function at $(\vec{q} - \vec{e}_{\alpha} \delta t ,t-1)$: $N_{\alpha}^{\star}(\vec{q} - \vec{e}_{\alpha} \delta t ,t-1)$;
take the pre-collision function at $(\vec{q},t)$: $N_{\alpha}(\vec{q},t)$ for curvilinear normalized discrete particle velocity set: 
$\vec{\tilde{e}}_{\alpha}(\vec{q} - \vec{e}_{\alpha} \delta t ,t-1)$ and $\vec{\tilde{e}}_{\alpha}(\vec{q},t)$.
The author take the constraint for momentum in the process of propagation as shown : 
\begin{equation}\begin{aligned}
\sum_{\alpha = 1}^{q} \vec{e}_{\alpha}(\vec{q},t)\,\delta N_{\alpha}(\vec{q} ) &= \sum_{\alpha = 1}^{q}  \vec{e}_{\alpha}(\vec{q} )\,N_{\alpha}(\vec{q})- \sum_{\alpha = 1}^{q}\vec{e}_{\alpha}(\vec{q})\, N_{\alpha}^{\star}(\vec{q} - \vec{e}_{\alpha} \delta t ,t-1)\\[1.5ex]
&= \mathcal{J} \cdot \frac{\vec{\chi}^{\mathrm{in}}(\vec{q} ,t)+ \vec{\chi}^{\mathrm{out}}(\vec{q} ,t)}{2} \\[1.5ex]
\end{aligned}
\end{equation}
\begin{equation}\begin{aligned}
&\sum_{\alpha = 1}^{q} \vec{e}_{\alpha}(\vec{q} )\,\delta N_{\alpha}(\vec{q} ,t) = \\[1.5ex]
&\mathcal{J} \cdot \frac{
-\sum_{\alpha = 1}^{q} N_{\alpha}(\vec{q})\left(\vec{e}_{\alpha}(\vec{q})- \vec{e}_{\alpha}(\vec{q} - \vec{e}_{\alpha} \delta t )\right)
-\sum_{\alpha = 1}^{q} N_{\alpha}^{\star}(\vec{q} ,t) \left(\vec{e}_{\alpha}(\vec{q} + \vec{e}_{\alpha} \delta t )- \vec{e}_{\alpha}(\vec{q})\right)
}{2}
\end{aligned}
\end{equation}
\begin{definition}
  Velocity-space discretized force field -- specific case in curvilinear coordinates
  \begin{equation}\label{eq:problem}
    F^{i} (\vec{q} , t) = \frac{\vec{\chi}^{\mathrm{in}}(\vec{q} ,t) + \vec{\chi}^{\mathrm{out}}(\vec{q} ,t)}{2} \cdot \vec{g}^{\ i}(\vec{q}) 
  \end{equation}
\end{definition}
\noindent 
The macroscopic force is discretized in particle-velocity space, and this represents an inertial force produced in a curvilinear-coordinate computational domain.

We now review the statement of the whole process. First, impose two constraints for conservation of density and momentum; then define the discrete inertial force.

However, the following expression appears to treat a symbol as a "variable":
$$\left(\sum_{\alpha = 1}^{q}\left(\vec{e}_{\alpha}(\vec{q} ) - \vec{e}_{\alpha}(\vec{q} - \vec{e}_{\alpha} \delta t )\right)\cdot N_{\alpha}(\vec{q} ,t)\right)$$
which is not exactly the momentum loss. The author then uses the true constraint to define the inertial force and maps that constraint to a velocity-space discretized force field.

The specific issues are:
\begin{enumerate}
  \item The definition of any force (including a velocity-discrete force field) should take a form like
  \begin{equation}\begin{aligned}
    &F^{i} (\vec{q} ,t) \equiv \frac{\text{Momentum Loss}}{\Delta t} \cdot \vec{e}_{i} \text{ (gerneral definition)}\\[1.5ex]
    &\mathcal{J} (\vec{q}) F^{i} (\vec{q} ,t) \equiv \frac{\text{Momentum Loss}}{ \Delta t} \cdot \vec{e}_{i} \frac{1}{|\vec{g}_{i}|}
  \end{aligned}\end{equation}
  where $\hat{g}_{\,i}=\dfrac{\vec{g}_{\,i}}{|\vec{g}_{\,i}|}$ is the unit vector in the $i$-th coordinate direction. This form uses a time difference (division by $\Delta t$) and projects onto the unit vector to obtain the component. The author's original expression omits the time scaling and the projection, which complicates interpretation.

  \item The second issue concerns the use of $\vec{g}^{\,i}$ in equation \eqref{eq:problem}. If $\vec{g}^{\,i}$ is intended to extract a component, it should be the unit vector $\hat{g}^{\,i}$. Multiplying by the magnitude $|\vec{g}^{\,i}|$ changes the scaling and can remove the proper coordinate-transformation property of the discrete force field. If a Jacobian or metric factor is required, it should be introduced explicitly and justified.
\end{enumerate}
\begin{derivation}
Proof of the equation.
  \begin{equation}
    \mathcal{J}(\vec{q})\, F^{\,i}(\vec{q},t) = -\frac{1}{2} \sum_{\alpha = 1}^{q} \Bigl[\; N_{\alpha}(\vec{q},t)\,\Delta\vec{e}_{\alpha}^{\,i}(\vec{q}) + N_{\alpha}^{\star}(\vec{q},t)\,\Delta\vec{e}_{\alpha}^{\,i}(\vec{q}+\vec{e}_{\alpha}\delta t)\;\Bigr] \;,
  \end{equation}
  where we write the change of the discrete velocity component as
  \[\Delta\vec{e}_{\alpha}^{\,i}(\vec{q}) = \vec{e}_{\alpha}(\vec{q}) - \vec{e}_{\alpha}(\vec{q}-\vec{e}_{\alpha}\delta t)\,.\]

The derivation follows from the momentum-loss identity
  \begin{equation}\begin{aligned}
    &\sum_{\alpha=1}^{q} \vec{e}_{\alpha}(\vec{q})\,\delta N_{\alpha}(\vec{q},t)
    = \mathcal{J}(\vec{q})\,\frac{\vec{\chi}^{\mathrm{in}}(\vec{q},t)+\vec{\chi}^{\mathrm{out}}(\vec{q},t)}{2}\\[4pt]
    &\quad= -\frac{1}{2} \Bigl(\sum_{\alpha=1}^{q} N_{\alpha}(\vec{q},t)\,\Delta\vec{e}_{\alpha}(\vec{q}) + \sum_{\alpha=1}^{q} N_{\alpha}^{\star}(\vec{q},t)\,\Delta\vec{e}_{\alpha}(\vec{q}+\vec{e}_{\alpha}\delta t)\Bigr)\,.
  \end{aligned}\end{equation}

Projecting onto the $i$-th coordinate direction we obtain
  \begin{equation}\begin{aligned}
    &\mathcal{J}(\vec{q})\,F^{\,i}(\vec{q},t)
    = \left(\sum_{\alpha=1}^{q} \vec{e}_{\alpha}(\vec{q})\,\delta N_{\alpha}(\vec{q},t)\right)\cdot \hat{g}^{\,i}(\vec{q}) \\[4pt]
    &= -\frac{1}{2} \Bigl(\sum_{\alpha=1}^{q} N_{\alpha}(\vec{q},t)\,\Delta\vec{e}_{\alpha}(\vec{q})\cdot\hat{g}^{\,i}(\vec{q}) \\
    &\qquad\qquqquad + \sum_{\alpha=1}^{q} N_{\alpha}^{\star}(\vec{q},t)\,\Delta\vec{e}_{\alpha}(\vec{q}+\vec{e}_{\alpha}\delta t)\cdot\hat{g}^{\,i}(\vec{q})\Bigr)\,.
  \end{aligned}\end{equation}

Using the representation $\vec{e}_{\alpha}(\vec{q}) = c_{\alpha}^{j}(\vec{q})\,\vec{g}_{j}(\vec{q})$ and $\hat{g}^{\,i}=\vec{g}^{\,i}/|\vec{g}^{\,i}|$, the component form becomes
  \begin{equation}\begin{aligned}
    &\mathcal{J}(\vec{q})\,F^{\,i}(\vec{q},t)
    = -\frac{1}{2} \sum_{\alpha=1}^{q} \Bigl[ N_{\alpha}(\vec{q},t)\,c_{\alpha}^{j}(\vec{q}-\vec{e}_{\alpha}\delta t)\,\Theta_{j}^{\,i}(\vec{q}-\vec{e}_{\alpha}\delta t,\vec{q}) \\
    &\qquad\qquad\qquad + N_{\alpha}^{\star}(\vec{q},t)\,c_{\alpha}^{j}(\vec{q}+\vec{e}_{\alpha}\delta t)\,\Theta_{j}^{\,i}(\vec{q}+\vec{e}_{\alpha}\delta t,\vec{q}) \Bigr]\,,
  \end{aligned}\end{equation}

where we define the geometric change
  \[\Theta_{j}^{\,i}(\vec{q}_a,\vec{q}_b) \equiv \frac{\vec{g}_{j}(\vec{q}_a)\cdot\vec{g}^{\,i}(\vec{q}_b) - \vec{g}_{j}(\vec{q}_b)\cdot\vec{g}^{\,i}(\vec{q}_b)}{|\vec{g}^{\,i}(\vec{q}_b)|}\,.

\end{derivation}
,

\end{document}



