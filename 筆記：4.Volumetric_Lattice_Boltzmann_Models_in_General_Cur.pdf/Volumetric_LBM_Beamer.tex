\documentclass[aspectratio=169]{beamer}
\usetheme{Madrid}
\usecolortheme{default}
\usepackage{amsmath,amssymb,physics}
\usepackage{tikz}
\usetikzlibrary{arrows.meta,positioning,shapes.geometric}
\usepackage{xcolor}
\usepackage{booktabs}
\usepackage{fontspec}
\usepackage{tcolorbox}

% 設定 Times New Roman 字體
\setmainfont{Times New Roman}
\setsansfont{Times New Roman}
\setmonofont{Courier New}

% Beamer 字體設定為 serif (Times New Roman)
\usefonttheme{serif}

% 自定義顏色
\definecolor{mainblue}{RGB}{0,82,147}
\definecolor{darkred}{RGB}{180,0,0}
\definecolor{darkgreen}{RGB}{0,120,0}

\setbeamercolor{title}{fg=white,bg=mainblue}
\setbeamercolor{frametitle}{fg=white,bg=mainblue}
\setbeamercolor{structure}{fg=mainblue}
\setbeamercolor{block title}{bg=mainblue!85,fg=white}
\setbeamercolor{block body}{bg=mainblue!8}

% 設定字體大小
\setbeamerfont{frametitle}{size=\large}

% 版面與字級優化
\setbeamersize{text margin left=0.8cm, text margin right=0.8cm}
\setbeamertemplate{navigation symbols}{}
\setbeamertemplate{footline}[frame number]
\setbeamerfont{title}{size=\Large}
\setbeamerfont{block title}{size=\normalsize}
\setbeamerfont{block body}{size=\small}
\setlength{\parskip}{0.4em}
\setlength{\abovedisplayskip}{0.4em}
\setlength{\belowdisplayskip}{0.4em}

% 標題資訊
\title{Volumetric Lattice Boltzmann Models in General Curvilinear Coordinates: Theoretical Formulation}
\author{Hudong Chen}
\institute{
Dassault Systemes, Waltham, MA, United States\\[0.5em]
\footnotesize Front. Appl. Math. Stat. 7:691582 \quad doi: 10.3389/fams.2021.691582\\
\footnotesize Published: 16 June 2021
}
\date{Presenter: Chen Peng Chung}

% 期刊風格封面排版
\newcommand{\journalheader}{%
\small\textbf{frontiers} in Applied Mathematics and Statistics\hfill\textbf{ORIGINAL RESEARCH}\\
\footnotesize published: 16 June 2021\hfill doi: 10.3389/fams.2021.691582
}
\newcommand{\journalsummary}{%
\begin{tcolorbox}[colback=gray!10, colframe=gray!50, boxrule=0.3pt, arc=2pt, left=0.5em, right=0.5em, top=0.3em, bottom=0.3em]
\small
A volumetric formulation for lattice Boltzmann models on general curvilinear coordinates is presented, preserving one-to-one advection and recovering the Navier--Stokes equations in the hydrodynamic limit.
\end{tcolorbox}
}

\begin{document}
%========== 第1頁:Title ==========
\begin{frame}[plain]
\vspace*{0.3cm}
\journalheader
\vspace{0.2cm}
\hrule
\vspace{0.6cm}

{\LARGE\bfseries \inserttitle}\par
\vspace{0.5cm}
{\large\itshape \insertauthor}\par
\vspace{0.2cm}
{\small \insertinstitute}\par
\vspace{0.4cm}

\journalsummary

\vfill
\hfill\small \insertdate
\end{frame}

%========== 第2頁:Outline ==========
\begin{frame}{Outline}
\tableofcontents
\end{frame}

%========== 第3頁:Introduction ==========
\section{Introduction \& Motivation}
\begin{frame}{Introduction \& Motivation}
\begin{block}{Core Problem}
For standard Lattice Boltzmann Methd, Because the velocity direction set, it only was used in Cartesian coordinates.However, this paper show the basic theoretical formulation to extend the limitation.
In \textbf{general curvilinear coordinates}, we have to consider:
\begin{itemize}
     \item Curvature effects
    \item Nonuniform grid properties
\end{itemize}
\end{block}

\vspace{0.5em}
\begin{alertblock}{Key Question}
How to maintain \textbf{momentum conservation} in the streaming step of LBM when using curvilinear coordinates?
\end{alertblock}

\vspace{0.5em}
\begin{block}{Solution Approach}
Define a \textbf{velocity-space discretized force field} to compensate for the momentum loss (interpreted as an \textbf{inertial force}), and make it be tangent with curve line.
\end{block}
\end{frame}

%========== 第4頁:LBGK Equation ==========
\section{Momentum Loss in Propagation}
\begin{frame}{Lattice Boltzmann BGK Equation}
The standard LBGK equation in curvilinear coordinates:
\begin{equation*}
\boxed{N_{\alpha}(\vec{q} + \vec{e}_{\alpha}\delta t, t+1) = N_{\alpha}(\vec{q} , t) + \Omega_{\alpha} +  \textcolor{red}{\delta N_{\alpha}(\vec{q} ,t)}}
\end{equation*}

\vspace{0.5em}
where the \textcolor{red}{correction term} is defined as:
\begin{equation*}
\delta N_{\alpha}(\vec{q} , t) \equiv N_{\alpha}(\vec{q} , t) - N_{\alpha}^{\star} (\vec{q} - \vec{e}_{\alpha}  \delta t, t-1) 
\end{equation*}

\begin{itemize}
    \item $\vec{q}$ : nondimensional position $(q_1, q_2, q_3)$
    \item $N^{\star}$ : \textbf{post-collision} distribution function
    \item $\Omega_{\alpha}$ : collision operator
    \item $\vec{e}_{\alpha}$ : discrete velocity directions in curvilinear coordinate.
\end{itemize}
where $\delta N(\vec{q} , t)$ is the source of momentum loss.
\end{frame} 

%========== 第5頁:Mass vs Momentum Conservation ==========
\begin{frame}{Mass Conservation vs Momentum Loss}
\begin{columns}[T]
\column{0.48\textwidth}
\begin{block}{\textcolor{black}{Mass Conservation (\checkmark)}}
\begin{equation*}
\sum_{\alpha=0}^{q-1} \delta N_{\alpha}(\vec{q} , t) = 0
\end{equation*}
\vspace{0.3em}
\textbf{Reason:} Sum of distributions over velocity space is conserved.
\end{block}

\column{0.48\textwidth}
\begin{alertblock}{\textcolor{black}{Momentum Loss ($\times$)}}
\begin{equation*}
\sum_{\alpha=1}^{q} \vec{e}_{\alpha}\,\delta N_{\alpha} \neq 0
\end{equation*}
\vspace{0.3em}
\textbf{Reason:} Discrete velocity vectors $\vec{e}_\alpha$ vary (in direction and magnititude)with position in curvilinear coords!
\end{alertblock}
\end{columns}

\vspace{1em}
\begin{center}
\tikz{
\node[draw, rounded corners, fill=yellow!20, text width=10cm, align=center] {
\textbf{Key Insight:} $N_{\alpha} (\vec{q} , t) \neq N_{\alpha}^{\star}(\vec{q} - \vec{e}_{\alpha} \delta t , t - 1)$\\[0.3em]
Pre-streaming $\neq$ Post-collision at different positions
};
}

\noindent Due to curvilinearity, though the
amount of particles advected from cell $\vec{q}$ is exactly equal to what is
arrived at cell $\vec{q} + \vec{e}_{\alpha}$, the momentum is in
fact changed along the way
\end{center}
\end{frame}

%========== 第6頁:Curvature Change Definition ==========
\section{Key Definitions}
\begin{frame}{Definition: Nondimensional Curvature Change}
\begin{columns}[T]
\column{0.55\textwidth}
\begin{definition}
The \textbf{nondimensional change of curvature tangent vector}:
\begin{equation*}
\boxed{\vec{\Theta}_{i}^{j} (\vec{q} + \vec{e}_{\alpha} \delta t , \vec{q} ) \equiv  \left[\vec{g}_{i}(\vec{q} +\vec{e}_{\alpha} \delta t) - \vec{g}_{i}(\vec{q})\right] \cdot \vec{g}^{\ j} (\vec{q})}
\end{equation*}
\end{definition}

where the \textbf{co-tangent basis vector} $\vec{g}^{\ j}$ is:
\begin{equation*}
\vec{g}^{1} \equiv \frac{\vec{g}_{2}\times \vec{g}_{3}}{(\vec{g}_{2}\times \vec{g}_{3}) \cdot \vec{g}_{1}} 
\end{equation*}

\column{0.42\textwidth}
\begin{figure}
\centering
\includegraphics[width=\textwidth]{1.png}
\end{figure}
\end{columns}

\vspace{0.3em}
\begin{itemize}
    \item $\vec{g}_i$ : curvatue tangent basis vector  (tangent to coordinate curves)
    \item $\vec{g}^j$ : co-tangent basis vector (normal to coordinate surfaces)
    \item $\Theta_i^j$ measures how the tangent vector changes along $\vec{e}_\alpha$
\end{itemize}
\end{frame}

%========== 第7頁:Momentum Loss Definition ==========
\begin{frame}{Definition: Momentum Loss}
\begin{definition}
\textbf{Total Momentum Loss} at point $(\vec{q}, t)$:
\begin{equation*}
\vec{M}_{\mathrm{Loss}} (\vec{q},t) \equiv
 \sum_{\alpha = 1}^{q} \left[N_{\alpha}(\vec{q} ,t)\,\vec{e}_{\alpha}(\vec{q}) - N_{\alpha}^{\star}(\vec{q} - \vec{e}_{\alpha} \delta t, t-1)\,\vec{e}_{\alpha}(\vec{q} - \vec{e}_{\alpha} \delta t )\right]
\end{equation*}
\end{definition}

\begin{itemize}
    \item Momentum carried by particles \textbf{arriving} at $\vec{q}$
    \item Minus momentum carried by particles \textbf{leaving} from $\vec{q} - \vec{e}_\alpha \delta t$
    \item Difference arises because $\vec{e}_\alpha(\vec{q}) \neq \vec{e}_\alpha(\vec{q} - \vec{e}_\alpha \delta t)$
\end{itemize}
For inducing the inertial force, the author adopt the definition as below : 
\begin{definition}
\begin{equation*}
\vec{M}_{\mathrm{Loss}} (\vec{q},t) \equiv
 \sum_{\alpha = 1}^{q} \vec{e}_{\alpha}(\vec{q} , t)\delta N_{\alpha}(\vec{q},t)
\end{equation*}
\end{definition}
\end{frame}

%========== 第8頁:Inner and Outer Momentum Loss ==========
\begin{frame}{Inner \& Outer Momentum Loss}
The paper separates momentum loss into \textbf{two contributions}:

\begin{definition}
\textbf{Inner Momentum Loss} (particles entering):
\begin{equation*}
\mathcal{J}(\vec{q} ) \,\vec{\chi}^{\mathrm{in}}(\vec{q} ,t) \equiv
-\sum_{\alpha = 1}^{q} N_{\alpha}(\vec{q},t)\cdot \left(\vec{e}_{\alpha}(\vec{q}) - \vec{e}_{\alpha}(\vec{q} - \vec{e}_{\alpha} \delta t ) \right)
\end{equation*}
\end{definition}

\begin{definition}
\textbf{Outer Momentum Loss} (particles leaving):
\begin{equation*}
\mathcal{J}(\vec{q}) \,\vec{\chi}^{ \mathrm{out}}(\vec{q},t) \equiv
-\sum_{\alpha = 1}^{q} \left(\vec{e}_{\alpha}(\vec{q} + \vec{e}_{\alpha} \delta t ) -\vec{e}_{\alpha}(\vec{q})\right)\cdot N_{\alpha}^{\star}(\vec{q},t)
\end{equation*}
\end{definition}

\vspace{0.3em}
where $\mathcal{J}$ is the Volume element in gerneral coordinate.
\end{frame}

%========== 第9頁:Discrete Velocity Set ==========
\begin{frame}{Discrete Velocity in Curvilinear Coordinates}
The \textbf{discrete particle velocity set} in curvilinear coordinates:
\begin{equation*}
\boxed{\vec{e}_{\alpha}(\vec{q}) = \sum_{i = 1}^{3} (c_{\alpha}^{i}) \,\vec{g}_{i} \,\frac{\Delta x}{\Delta t}}
\end{equation*}

\vspace{1em}
\begin{itemize}
    \item $c_{\alpha}^{i}$ : lattice velocity components (integers like $0, \pm 1$)
    \item $\vec{g}_i$ : local covariant basis vectors
    \item $\Delta x / \Delta t$ : lattice speed ratio
\end{itemize}

\vspace{0.5em}
\begin{alertblock}{Important}
Unlike Cartesian coordinates, $\vec{e}_\alpha$ depends on position $\vec{q}$ through the basis vectors $\vec{g}_i(\vec{q})$.
\end{alertblock}
\end{frame}

%========== 第10頁:Momentum Conservation Constraint ==========
\section{Force Field Derivation}
\begin{frame}{Momentum Conservation Constraint}
\small
The author imposes the \textbf{momentum constraint} for the streaming step:

\begin{equation*}\label{eq:2}
\sum_{\alpha = 1}^{q} \vec{e}_{\alpha}(\vec{q})\,\delta N_{\alpha}(\vec{q}) = \mathcal{J} \cdot \frac{\vec{\chi}^{\mathrm{in}}(\vec{q} ,t)+ \vec{\chi}^{\mathrm{out}}(\vec{q} ,t)}{2}
\end{equation*}

\vspace{0.5em}
Expanding:
\begin{equation*}
= \mathcal{J} \cdot \frac{
\textcolor{blue}{-\sum_{\alpha} N_{\alpha}(\vec{q})\left(\vec{e}_{\alpha}(\vec{q})- \vec{e}_{\alpha}(\vec{q} - \vec{e}_{\alpha} \delta t )\right)}
\textcolor{red}{-\sum_{\alpha} N_{\alpha}^{\star}(\vec{q}) \left(\vec{e}_{\alpha}(\vec{q} + \vec{e}_{\alpha} \delta t )- \vec{e}_{\alpha}(\vec{q})\right)}
}{2}
\end{equation*}

\begin{itemize}
    \item \textcolor{blue}{Blue}: contribution from pre-collision distribution
    \item \textcolor{red}{Red}: contribution from post-collision distribution
\end{itemize}
\normalsize
\end{frame}

%========== 第11頁:Velocity-Space Discretized Force ==========
\begin{frame}{Inertial Force Definition}
\begin{definition}
The \textbf{inertial force} produced by curvature in curvilinear coordinates:
\begin{equation*}
\boxed{F^{i} (\vec{q} , t) \equiv  \sum_{\alpha = 1}^{q} \vec{e}_{\alpha} \delta N_{\alpha}(\vec{q},t)= \frac{\vec{\chi}^{\mathrm{in}}(\vec{q} ,t) + \vec{\chi}^{\mathrm{out}}(\vec{q} ,t)}{2} \cdot \vec{g}^{\ i}(\vec{q})}
\end{equation*}
\end{definition}

\vspace{0.5em}
\textbf{Physical Interpretation:}
\begin{itemize}
    \item This force represents the \textbf{inertial force} arising from curvilinear coordinates
    \item the reason why this term exist is difference between pre-streaming and post-collision distributojn function.
    \item Compensates for the ``loss'' in momentum during streaming
\end{itemize}

\vspace{0.5em}
\begin{block}{My problem about definiton of force}
\begin{equation*}\mathcal{J} (\vec{q}) F^{i} (\vec{q} ,t) \equiv \dfrac{\text{Momentum Loss}}{\Delta t}\cdot \vec{e}_{j} 
\end{equation*}
\end{block}
\end{frame}

%========== 第12頁:Final Force Formula ==========
\begin{frame}{Final Force Formula}
\small
After derivation, the force field becomes:
\begin{equation*}
\boxed{
\mathcal{J} (\vec{q} )  F^{j}(\vec{q} , t) = \frac{-1}{2} \left(\sum_{\alpha} N_{\alpha}\,c_{\alpha}^{i} \Theta_{i}^{j} (\vec{q} - \vec{e}_{\alpha}\delta t , \vec{q})
+ \sum_{\alpha}N_{\alpha}^{\star}\,c_{\alpha}^{i} \Theta_{i}^{j} (\vec{q} + \vec{e}_{\alpha}\delta t , \vec{q}) \right)
}
\end{equation*}

\vspace{0.5em}
\textbf{Key Components:}
\begin{table}[h]
\centering
\begin{tabular}{cl}
\toprule
Symbol & Meaning \\
\midrule
$\mathcal{J}$ & volume  of the cell centered at $\vec{x}(\vec{q})$, \\[1ex]
$c_\alpha^i$ & Lattice velocity component \\[1ex]
$\Theta_i^j$ & change of Curvature change tensor \\[1ex]
$N_\alpha$ & Post-streaming distribution \\[1ex]
$N_\alpha^\star$ & Post-collision(pre-streaming) distribution \\[1ex]
\bottomrule
\end{tabular}
\end{table}
\normalsize
\end{frame}

%========== 第13頁:Derivation Steps ==========
\begin{frame}{Solving the addition term $\delta N_{\alpha}$}
\small
To recover the full viscous Navier-Stokes equation, an additional constraint on the \textbf{momentum flux} is needed:

\vspace{0.3em}
\textbf{Momentum flux constraint (Eq.29):}
\begin{equation*}\label{eq:1}
\sum_{\alpha} c_{\alpha}^{i} c_{\alpha}^{j} \delta N_{\alpha}(\vec{q},t) = \mathcal{J}(\vec{q})\left[\delta \Pi^{ij}(\vec{q},t) + \delta \Pi^{ji}(\vec{q},t)\right]
\end{equation*}

\textbf{Non-equilibrium stress tensor (Eq.30):}
\begin{equation*}
\delta \Pi^{ij}(\vec{q},t) = -\frac{1}{2}\left(1 - \frac{1}{2\tau}\right) \sum_{\alpha} c_{\alpha}^{i} c_{\alpha}^{k} \left[\Theta_{k}^{j}(\vec{q} + \vec{e}_{\alpha}\delta t, \vec{q}) - \Theta_{k}^{j}(\vec{q} - \vec{e}_{\alpha}\delta t, \vec{q})\right] f_{\alpha}^{\text{eq}}(\vec{q},t)
\end{equation*}

\textbf{Specific form of $\delta N_{\alpha}$ (Eq.31):}
\begin{equation*}
\boxed{\delta N_{\alpha}(\vec{q},t) = w_{\alpha} \mathcal{J}(\vec{q}) \left[\frac{c_{\alpha}^{i} F^{i}(\vec{q},t)}{T_{0}} + \left(\frac{c_{\alpha}^{i} c_{\alpha}^{k}}{T_{0}} - \delta^{ik}\right) \frac{\delta \Pi^{ik}(\vec{q},t)}{T_{0}}\right]}
\end{equation*}

\vspace{0.2em}
{\footnotesize This form satisfies the moment constraints of \eqref{eq:1}, \eqref{eq:1}}
\end{frame}

%========== 第14頁:Basic Equations for Implementation ==========
\section{Equilibrium Distribution Function}
\begin{frame}{Equilibrium Distribution \& Moment Constraints}
\small
\textbf{Fundamental moment conditions} (to recover Euler \& Navier-Stokes):
\begin{align*}
&\sum_{\alpha} f_{\alpha} = \rho \\[0.5ex]
&\sum_{\alpha} \vec{e}_{\alpha} f_{\alpha} = \rho \vec{u} (\vec{q} , t)  \Rightarrow \sum_{\alpha} c_{\alpha}^{i} f_{\alpha} =  \rho U^{i}\\[0.5ex]
&\sum_{\alpha} c_{\alpha}^{i} c_{\alpha}^{j} f_{\alpha}^{\text{eq}} = \Pi^{ij,\text{eq}} = g^{ij} \rho T_{0} + \rho \tilde{U}^{i} \tilde{U}^{j}
\end{align*}

where the \textbf{shifted velocity}: $\displaystyle\tilde{U}^{i}(\vec{q},t) = U^{i}(\vec{q},t) + \frac{1}{2} a^{i}(\vec{q},t)$, \quad with $\rho a^{i} \equiv F^{i}$

\vspace{0.5em}
\textbf{Equilibrium distribution form:}
\begin{equation*}
\boxed{f_{\alpha}^{\text{eq}} = \rho w_{\alpha} \left[1 + \frac{c_{\alpha}^{i} U^{i}}{T_{0}} + \frac{1}{2T_{0}} \left(\frac{c_{\alpha}^{i} c_{\alpha}^{j}}{T_{0}} - \delta^{ij}\right) \left[(g^{ij} - \delta^{ij}) T_{0} + \tilde{U}^{i} \tilde{U}^{j}\right]\right]}
\end{equation*}

{\footnotesize This form satisfies all the fundamental moment conditions above.}
\end{frame}

\begin{frame}{Standard Lattice Boltzmann Method}
\small
\textbf{Fundamental moment conditions} (to recover Euler \& Navier-Stokes):
\begin{align*}
&\sum_{\alpha} f_{\alpha}(\vec{r},t) = \rho \\[0.5ex]
&\sum_{\alpha} c_{\alpha}^{i} f_{\alpha} =  \rho U^{i}\\[0.5ex]
&\sum_{\alpha} c_{\alpha}^{i} c_{\alpha}^{j} f_{\alpha}^{\text{eq}} = \rho \left(U^{i}U^{j} + c_s^{2} \delta_{ij}\right)
\end{align*}

\vspace{0.5em}
\textbf{Equilibrium distribution form:}
\begin{equation*}
\boxed{f_{\alpha}^{\text{eq}} = \rho w_{\alpha} \left[1 + \frac{c_{\alpha}^{i}U^{i}}{c_{x}^{2}} + \frac{(U^{i}U^{j}+ (\theta-1)\delta_{ij})(c_{\alpha}^{i}c_{\alpha}^{j} - c_{s}^{2}\delta_{ij})}{2c_s^{4}}\right]}
\end{equation*}

{\footnotesize This form satisfies all the fundamental moment conditions above.}
\end{frame}

%========== 第15頁:Summary ==========
\section{Summary}
\begin{frame}{Summary \& Key Takeaways}
\begin{block}{Main Contributions}
\begin{enumerate}
\item Identified \textbf{momentum loss} in streaming step for curvilinear LBM
\item Introduced \textbf{curvature change tensor} $\Theta_i^j$ to quantify geometric effects
\item Derived \textbf{velocity-space discretized force field} to restore conservation
\end{enumerate}
\end{block}

\vspace{0.5em}
\begin{alertblock}{Physical Insight}
The force field $F^i$ acts as an \textbf{inertial force} (like centrifugal force), arising purely from the curvilinear coordinate system.
\end{alertblock}

\vspace{0.5em}
\begin{block}{Practical Use}
Add correction term $\delta N_\alpha$ to LBGK equation:
\begin{equation*}
N_{\alpha}(\vec{q} + \vec{e}_{\alpha}\delta t, t+1) = N_{\alpha}(\vec{q} , t) + \Omega_{\alpha} + \delta N_{\alpha}
\end{equation*}
\end{block}
\end{frame}

%========== Appendix ==========
\appendix
\section{Appendix}

\begin{frame}{Appendix: Derivation of Force Field Equation}
\small
\textbf{Target Equation:}
\begin{equation*}
\mathcal{J} (\vec{q} )  F^{\ i}(\vec{q} , t) = \frac{-1}{2} \left(\sum_{\alpha = 1}^{q} N_{\alpha}(\vec{q} ,t)\left(c_{\alpha}^{i} \Theta_{i}^{j} (\vec{q} - \vec{e}_{\alpha}\delta t , \vec{q})\right)
+ \sum_{\alpha = 1}^{q}N_{\alpha}^{\star}(\vec{q},t) \left(c_{\alpha}^{i} \Theta_{i}^{j} (\vec{q} + \vec{e}_{\alpha}\delta t , \vec{q}) \right)\right)
\end{equation*}

\vspace{0.3em}
\textbf{Derivation Principles:}
\begin{enumerate}
\item $\displaystyle\sum_{\alpha=1}^{q}\vec{e}_{\alpha}(\vec{q})\ \delta N_{\alpha}(\vec{q},t)\equiv$ Momentum Loss of the Streaming Step 
\item $\displaystyle\sum_{\alpha=1}^{q}\vec{e}_{\alpha}(\vec{q})\ \delta N_{\alpha}(\vec{q},t)\equiv \mathcal{J} (\vec{q}) \cdot \frac{\vec{\chi}^{\ in } + \vec{\chi}^{\ out}}{2}$ (for Volumetric LBM)
\item Velocity space discretized force field: $\displaystyle\mathcal{J} (\vec{q} ) F^{\ i}\equiv \frac{\text{Momentum Loss}}{\Delta t}\cdot \vec{e}_{j} \frac{1}{|\vec{g}_{j}|}$
\end{enumerate}
\end{frame}

\begin{frame}{Appendix: Derivation (cont.)}
\footnotesize
\textbf{Step-by-step derivation:}
\begin{equation*}\begin{aligned}
\mathcal{J} (\vec{q}) F^{1} &= \sum_{\alpha = 1}^{q}\vec{e}_{\alpha}(\vec{q}) \delta  N_{\alpha}(\vec{q} , t) \cdot \vec{g}^{\ 1} = \sum_{\alpha = 1}^{q}\vec{e}_{\alpha}(\vec{q}) \delta  N_{\alpha}(\vec{q} , t)  \cdot \frac{\vec{g_{2}} \times \vec{g_{3}}}{(\vec{g_{2}} \times \vec{g_{3}})\cdot \vec{g}_{1}}\\[1ex] 
&= \frac{-1}{2} \left(\sum_{\alpha = 1}^{q} N_{\alpha}(\vec{q} ,t)\left(\vec{e}_{\alpha}(\vec{q}) - \vec{e}_{\alpha}(\vec{q} - \vec{e}_{\alpha} \delta t ) \right) + \sum_{\alpha = 1}^{q}N_{\alpha}^{\star}(\vec{q},t) \left(\vec{e}_{\alpha}(\vec{q} + \vec{e}_{\alpha} \delta t) - \vec{e}_{\alpha}(\vec{q}) \right)\right)\\
&\quad \cdot \frac{\vec{g_{2}} \times \vec{g_{3}}}{(\vec{g_{2}} \times \vec{g_{3}})\cdot \vec{g}_{1}}
\end{aligned}\end{equation*}
\end{frame}

\begin{frame}{Appendix: Derivation (cont.)}
\footnotesize
\textbf{Continuing:}
\begin{equation*}\begin{aligned}
&= \frac{-1}{2} \left(\sum_{\alpha = 1}^{q} N_{\alpha}(\vec{q} ,t)\left(\vec{e}_{\alpha}(\vec{q}) - \vec{e}_{\alpha}(\vec{q} - \vec{e}_{\alpha} \delta t ) \right)+ \sum_{\alpha = 1}^{q}N_{\alpha}^{\star}(\vec{q},t) \left(\vec{e}_{\alpha}(\vec{q} + \vec{e}_{\alpha} \delta t ) - \vec{e}_{\alpha}(\vec{q}) \right)\right)\cdot \vec{e}_{j} \frac{1}{|\vec{g}_{j}|}\\[1ex] 
&= \frac{-1}{2} \left(\sum_{\alpha = 1}^{q} N_{\alpha}(\vec{q} ,t)\left(\vec{e}_{\alpha}(\vec{q})\cdot \vec{e}_{j} \frac{1}{|\vec{g}_{j}|}  - \vec{e}_{\alpha}(\vec{q} - \vec{e}_{\alpha} \delta t )\cdot \vec{e}_{j} \frac{1}{|\vec{g}_{j}|} \right)\right.\\[1ex] 
&\quad \left.+ \sum_{\alpha = 1}^{q}N_{\alpha}^{\star}(\vec{q},t) \left(\vec{e}_{\alpha}(\vec{q} + \vec{e}_{\alpha} \delta t )\cdot \vec{e}_{j} \frac{1}{|\vec{g}_{j}|}- \vec{e}_{\alpha}(\vec{q})\cdot \vec{e}_{j} \frac{1}{|\vec{g}_{j}|}\right)\right)
\end{aligned}\end{equation*}
\end{frame}

\begin{frame}{Appendix: Derivation (Final)}
\footnotesize
\textbf{Using $\Theta_i^j$ definition:}
\begin{equation*}\begin{aligned}
&= \frac{-1}{2}  \left(\sum_{\alpha = 1}^{q} N_{\alpha}(\vec{q} ,t)\left(c_{\alpha}^{j} \cdot \frac{|\vec{g}_{j}|(\vec{q}) - |\vec{g}_{j}|(\vec{q} - \vec{e}_{\alpha}\delta t)}{|\vec{g}_{j}|} \right)
+ \sum_{\alpha = 1}^{q}N_{\alpha}^{\star}(\vec{q},t) \left(c_{\alpha}^{j} \cdot \frac{|\vec{g}_{j}|(\vec{q} +  \vec{e}_{\alpha}\delta t) - |\vec{g}_{j}|(\vec{q})}{|\vec{g}_{j}|} \right)\right)\\[1.5ex]
&= \frac{-1}{2} \left(\sum_{\alpha = 1}^{q} N_{\alpha}(\vec{q} ,t)\left(c_{\alpha}^{j} \cdot \frac{{\color{red}|\vec{g}_{j}|(\vec{q}-\vec{e}_{\alpha}\delta t) - |\vec{g}_{j}|(\vec{q})}}{|\vec{g}_{j}|} \right)
+ \sum_{\alpha = 1}^{q}N_{\alpha}^{\star}(\vec{q},t) \left(c_{\alpha}^{j} \cdot \frac{|\vec{g}_{j}|(\vec{q} +  \vec{e}_{\alpha}\delta t) - |\vec{g}_{j}|(\vec{q})}{|\vec{g}_{j}|} \right)\right)
\end{aligned}\end{equation*}

\vspace{0.5em}
\textbf{Final Result:}
\begin{equation*}
\boxed{\mathcal{J} (\vec{q} )  F^{\ i} = \frac{-1}{2}  \left(\sum_{\alpha = 1}^{q} N_{\alpha}(\vec{q} ,t)\left(c_{\alpha}^{i} \Theta_{i}^{j} (\vec{q} - \vec{e}_{\alpha}\delta t , \vec{q})\right)
+ \sum_{\alpha = 1}^{q}N_{\alpha}^{\star}(\vec{q},t) \left(c_{\alpha}^{i} \Theta_{i}^{j} (\vec{q} + \vec{e}_{\alpha}\delta t , \vec{q}) \right)\right)}
\end{equation*}
\end{frame}

\end{document}
