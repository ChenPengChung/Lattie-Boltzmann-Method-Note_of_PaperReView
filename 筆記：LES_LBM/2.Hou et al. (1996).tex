\documentclass[12pt]{article}
\usepackage[a4paper,margin=2cm]{geometry} % 明確設定四邊
\usepackage{fontspec} % 字體設定
\usepackage{xeCJK}
\usepackage{setspace} % 設定行距
\linespread{1.2}
\usepackage{titling} % 預設標題下移0.6in
\usepackage{enumitem}
\usepackage{amsmath} % 數學方程式
\usepackage{graphicx} %圖片
\usepackage{float} % 在導言區,讓圖片強制插在原地
\usepackage{xcolor} %字體加入顏色
\usepackage{listings}
\usepackage{physics} % 物理符號
\usepackage{wrapfig} % 文字環繞圖
\usepackage{array} % 表格對齊控制
\usepackage{caption}
\usepackage{amsthm}
\usepackage{tabularx}
\usepackage{booktabs}
\usepackage{multirow}
\usepackage{etoolbox}
\usepackage{titlesec}
\usepackage{tocloft}
\usepackage{hyperref}
\usepackage{makecell}
\usepackage{xcolor}
\usepackage{colortbl}
\hypersetup{
  colorlinks=true,
  linkcolor=blue,
  urlcolor=blue,
  citecolor=blue,
  linktoc=all
}
\usepackage{tcolorbox}
\tcbuselibrary{skins,breakable}
\usepackage{tikz}
\usetikzlibrary{arrows.meta,positioning,shapes.geometric}
\tcbset{
  highlightblock/.style={
    enhanced,
    breakable,
    colback=gray!8,
    colframe=gray!60,
    boxrule=0.4pt,
    arc=0pt,
    boxsep=0pt,
    left=4pt,
    right=4pt,
    top=4pt,
    bottom=4pt,
    before skip=0pt,
    after skip=0pt
  }
}
\usepackage{longtable}

%1 註記行距設定
\setlist[itemize]{itemsep=1.2pt, parsep=0pt, topsep=1pt}
\setlist[enumerate]{itemsep=1.2pt, parsep=0pt, topsep=1pt}
%2 圖表標題設定
\captionsetup{
    labelfont={footnotesize,bf},    % 標籤:小字體+粗體
    skip=10pt                        % 標題與圖表的間距
}
%3 表格間距設定
\setlength{\tabcolsep}{8pt} % 列間距(增加)
\renewcommand{\arraystretch}{1.4} % 行間距(增加)
%4 腳註設定
\usepackage[bottom,hang]{footmisc} % 腳註置於頁面底部,懸掛縮排
\setlength{\footnotesep}{10pt} % 腳註之間的間距(增加)
\setlength{\skip\footins}{12pt plus 5pt minus 2pt} % 正文與腳註之間的間距(增加)
\setlength{\footnotemargin}{1.5em} % 腳註標號與文字的間距(增加)
\renewcommand{\footnoterule}{\vspace*{-3pt}\hrule width 0.4\columnwidth height 0.4pt\vspace*{3pt}} % 腳註分隔線
%5 使用直立字體的定理樣式
\newtheoremstyle{upright}
  {6pt}{6pt}  % 定理環境前後間距(增加)
  {\normalfont}% 使用正常字體,不使用斜體
  {0pt}{\bfseries}{:}{0.5em}{}
%6 定理環境定義
\theoremstyle{upright}
\newtheorem{definition}{定義}[section]
\newtheorem{theorem}{定理}[section]
\newtheorem{lemma}{引理}[section]
\newtheorem{corollary}{推論}[section]
\newtheorem{example}{例子}[section]
% 重新正確定義 remark 環境 - 確保完全靠左對齊
\makeatletter
\@ifundefined{c@remark}{}{\renewcommand{\theremark}{}}
\newenvironment{remark}{%
  \par\vspace{0.0\baselineskip}%
  \begingroup% 開始一個組以限制設置的範圍
  \setlength{\parindent}{0pt}% 設置段落縮進為零
  \setlength{\leftskip}{0pt}% 設置左邊距為零
  \noindent\textbf{註記:}\\% 添加標題並強制換行
  \ignorespaces% 忽略可能的空格
  \setlist[enumerate]{itemsep=0pt, parsep=0pt, topsep=0.0pt, leftmargin=1.5em}%
  \setlist[itemize]{itemsep=0pt, parsep=0pt, topsep=0.0pt, leftmargin=1.5em}%
}{%
  \endgroup% 結束組
  \par\vspace{0.5\baselineskip}
}
\makeatother
%7 手動定義中文數字(不含標點符號)
\newcommand{\chinese}[1]{%
  \ifcase#1 零\or 一\or 二\or 三\or 四\or 五\or 六\or 七\or 八\or 九\or 十\or
  十一\or 十二\or 十三\or 十四\or 十五\or 十六\or 十七\or 十八\or 十九\or 二十\fi
}

%8 重新定義章節編號格式
\renewcommand{\thesection}{\chinese{\value{section}}}
\renewcommand{\thesubsection}{\chinese{\value{section}}、\arabic{subsection}}
\renewcommand{\theequation}{\chinese{\value{section}}.\arabic{equation}}
\renewcommand{\thesubsubsection}{\chinese{\value{section}}、\arabic{subsection}.\arabic{subsubsection}}
\renewcommand{\thefigure}{\chinese{\value{section}}.\arabic{figure}}
\renewcommand{\thedefinition}{\thesection、\arabic{definition}}
\renewcommand{\thetheorem}{\thesection、\arabic{theorem}}
\renewcommand{\theexample}{\thesection、\arabic{example}}

% 为不同级别的标题增加编号后的空格
\setlength{\cftsecnumwidth}{2.5em}  % section 编号宽度
\setlength{\cftsubsecnumwidth}{3.5em}  % subsection 编号宽度
\setlength{\cftsubsubsecnumwidth}{4.5em}  % subsubsection 编号宽度

% 添加点线(可选)
\renewcommand{\cftsubsubsecleader}{\cftdotfill{\cftdotsep}}
\renewcommand{\cftsubsecleader}{\cftdotfill{\cftdotsep}}
\renewcommand{\cftsecleader}{\cftdotfill{\cftdotsep}}
%9 讓方程式計數器在每個section重置
\counterwithin{equation}{section}
%10 字體設定 - 優化以減少警告
\IfFontExistsTF{Times New Roman}{%
  \setmainfont{Times New Roman}%
}{%
  \setmainfont{TeX Gyre Termes}%
}

%\setCJKmainfont[
%    BoldFont={Kaiti TC Bold},
%    ItalicFont={Kaiti TC},
%    BoldItalicFont={Kaiti TC Bold}
%]{Kaiti TC}
%\setCJKmonofont{Kaiti TC}
\IfFontExistsTF{DFKai-SB}{%
  \setCJKmainfont[
    BoldFont={DFKai-SB},
    ItalicFont={DFKai-SB},
    BoldItalicFont={DFKai-SB}
  ]{DFKai-SB}%
  % 為 xeCJK 明確指定等寬字型以消除 \CJKttdefault 警告
  \setCJKmonofont{DFKai-SB}%
}{%
  \IfFontExistsTF{Kaiti TC}{%
    \setCJKmainfont[
      BoldFont={Kaiti TC Bold},
      ItalicFont={Kaiti TC},
      BoldItalicFont={Kaiti TC Bold}
    ]{Kaiti TC}%
    % 為 xeCJK 明確指定等寬字型以消除 \CJKttdefault 警告
    \setCJKmonofont{Kaiti TC}%
  }{%
    \IfFontExistsTF{Songti TC}{%
      \setCJKmainfont{Songti TC}%
      % 為 xeCJK 明確指定等寬字型以消除 \CJKttdefault 警告
      \setCJKmonofont{Songti TC}%
    }{%
      \setCJKmainfont{Heiti TC}%
      % 為 xeCJK 明確指定等寬字型以消除 \CJKttdefault 警告
      \setCJKmonofont{Heiti TC}%
    }%
  }%
}
%11 標題設定
\setlength{\droptitle}{-1in} % 上移標題1in
\title{LES\_LBM}
\author{Chen Peng Chung}
\setcounter{section}{0}
% 12调整 subsection 的间距
\titlespacing*{\section}
{0pt}                    % 左边距
{0.0in}                 % 标题前间距
{0.0em}                  % 标题后间距
\titlespacing*{\subsection}
{0pt}                    % 左边距
{-0.05in}                % 标题前间距
{0.0em}                  % 标题后间距
% 调整 subsubsection 的间距  
\titlespacing*{\subsubsection}
{0pt}                    % 左边距
{0.0in}                 % 标题前间距
{0.0em}                  % 标题后间距
%13.

% ========== Monokai Light 配色方案(淺灰背景)==========
\definecolor{monokailight-bg}{RGB}{240,240,240}      % 淺灰背景 #F0F0F0
\definecolor{monokailight-fg}{RGB}{39,40,34}         % 深色文字 #272822
\definecolor{monokailight-comment}{RGB}{117,113,94}  % 註解 #75715E
\definecolor{monokailight-string}{RGB}{152,118,24}   % 字串 #987618
\definecolor{monokailight-keyword}{RGB}{244,0,95}    % 關鍵字 #F4005F
\definecolor{monokailight-function}{RGB}{121,162,0}  % 函數 #79A200
\definecolor{monokailight-number}{RGB}{137,89,168}   % 數字 #8959A8
\definecolor{monokailight-type}{RGB}{0,129,152}      % 型別 #008198
\definecolor{monokailight-operator}{RGB}{244,0,95}   % 運算符 #F4005F
\definecolor{monokailight-border}{RGB}{210,210,210}  % 邊框 #D2D2D2

% ========== CUDA 語言定義 ==========
\lstdefinelanguage{CUDA}{
    language=C++,
    morekeywords={__global__, __device__, __host__, __shared__, 
                  __constant__, __syncthreads, dim3, cudaMalloc,
                  cudaMemcpy, cudaFree, cudaMemcpyHostToDevice,
                  cudaMemcpyDeviceToHost, cudaDeviceSynchronize},
    % 型別關鍵字
    morekeywords=[2]{double, float, int, char, void, size_t,
                     uint, uint2, uint3, uint4,
                     int2, int3, int4,
                     float2, float3, float4,
                     double2, double3, double4},
    % CUDA 內建變數
    morekeywords=[3]{blockIdx, threadIdx, blockDim, gridDim,
                     warpSize},
    sensitive=true
}

% ========== Monokai Light 樣式 ==========
\lstdefinestyle{monokailight}{
    language=CUDA,
    backgroundcolor=\color{monokailight-bg},
    basicstyle=\color{monokailight-fg}\ttfamily\footnotesize,
    commentstyle=\color{monokailight-comment}\itshape,
    keywordstyle=\color{monokailight-keyword}\bfseries,
    keywordstyle=[2]\color{monokailight-type}\bfseries,      % 型別
    keywordstyle=[3]\color{monokailight-function}\bfseries,  % CUDA 變數
    stringstyle=\color{monokailight-string},
    numberstyle=\tiny\color{monokailight-comment},
    % 數字顏色
    literate=
        {0}{{{\color{monokailight-number}0}}}1
        {1}{{{\color{monokailight-number}1}}}1
        {2}{{{\color{monokailight-number}2}}}1
        {3}{{{\color{monokailight-number}3}}}1
        {4}{{{\color{monokailight-number}4}}}1
        {5}{{{\color{monokailight-number}5}}}1
        {6}{{{\color{monokailight-number}6}}}1
        {7}{{{\color{monokailight-number}7}}}1
        {8}{{{\color{monokailight-number}8}}}1
        {9}{{{\color{monokailight-number}9}}}1,
    % 版面設定
    breaklines=true,
    breakatwhitespace=false,
    captionpos=b,
    keepspaces=true,
    numbers=left,
    numbersep=8pt,
    showspaces=false,
    showstringspaces=false,
    showtabs=false,
    tabsize=4,
    % 框線設定
    frame=single,
    rulecolor=\color{monokailight-border},
    framerule=0.8pt,
    % 其他設定
    columns=flexible,
    escapeinside={(*@}{@*)},
    xleftmargin=2em,
    xrightmargin=0.5em,
    framexleftmargin=1.5em
}

% ========== 設為預設樣式 ==========
\lstset{style=monokailight}
%14. 
% ========== 表格樣式設定 ==========
\setlength{\arrayrulewidth}{0.5pt}  % 表格線條粗細
\renewcommand{\arraystretch}{1.3}   % 行高
%15listing 設定:
\lstset{
  language=C++,
  basicstyle=\ttfamily\small,
  numbers=left,
  numberstyle=\tiny,
  stepnumber=1,
  numbersep=8pt,
%
  keepspaces=true,      % ★保留空白
  showspaces=false,
  showstringspaces=false,
  showtabs=false,
%
  tabsize=2,            % ★tab 視為 2 個空白(可改 4)
  breaklines=true,
  breakatwhitespace=false,
%
  frame=single,
  captionpos=b
}



\begin{document}
\maketitle
\tableofcontents
\newpage
\section{前言}
\noindent 由速度空間離散化波茲曼方程:
\begin{equation}\begin{aligned}\label{eq:1}
\frac{\partial f_{i}}{\partial t} (\vec{r}  , t) + \vec{e}_{i}\cdot \frac{\partial f_{i}}{\partial \vec{r}}(\vec{r}  , t)  = \Omega(f_{i}  , f_{i}^{eq})
\end{aligned}\end{equation}
定義濾波變換:
\begin{equation}\begin{aligned}
\mathcal{L} [f] = \int_{\Omega} f(\vec{r}) G(\vec{r} - \vec{r}^{'}) dr^{3}
\end{aligned}\end{equation}
\noindent 對\eqref{eq:1}上式兩側左右同取濾波變換,則有:
\begin{equation}\begin{aligned}\label{eq:2}
\frac{\partial \mathcal{L} [f_{i}(\vec{r}  , t)]}{\partial t}  + \vec{e}_{i}\cdot \frac{\partial \mathcal{L} [f_{i}(\vec{r}  , t) ]}{\partial \vec{r}} = \mathcal{L} [\Omega(f_{i}  , f_{i}^{eq})]
\end{aligned}\end{equation}
上式\eqref{eq:2}即稱為濾波型速度空間離散化波茲曼方程,對上式\eqref{eq:2}進行時空離散化,則有:
\begin{equation}\begin{aligned}
  \mathcal{L}[f_{i}](\vec{r} + \vec{e}_{i}\delta t , t+ \delta t) = \mathcal{L}[f_{i}] (\vec{r} ,t) + \mathcal{L} [\Omega(f_{i} - f_{i}^{eq})] \delta t
\end{aligned}\end{equation}
\noindent 上式即稱為濾波型晶格波茲曼方程。
\section{亞格子模型與濾波}
\noindent 何謂亞格子應力張量模型?在紊流裡面,最需要關心的雷諾應力張量場 (Raynolds Stress Tensor),在LES模擬中,為亞格子應力張量場(subgrid stress tensor),其表達式為:
\begin{equation}\begin{aligned}
  \tau_{ij} &= R_{ij} + C_{ij} + L_{ij} \\[1.5ex]
  &= \left(\overline{u^{'}u^{'}} \right)+ \left( \overline{\overline{u}u^{'}}  + \overline{u^{'}\overline{u}} \right) + \left(\overline{\overline{u}\overline{u}} - \overline{u}\cdot\overline{u}\right) + \overline{u}\cdot\overline{u}
 \end{aligned}\end{equation}
\noindent 其中,$R_{ij}$為亞格子雷諾應力張量場,由此延伸,亞格子應力張量場的一種近似:(S(R(D亞格子應力張量場)的(各向異性部分))的(Smagorinsky模型))。如下給出該應力模型的定義:
\begin{equation}\begin{aligned}
  \tau_{ij} = -2\nu_{t}\overline{S_{ij}} = -2\left(C_s\Delta^{2}\sqrt{2\overline{S_{ij}}\overline{S_{ij}}}\right)\overline{S_{ij}}
\end{aligned}\end{equation}
\noindent 其中,$\overline{S_{ij}} = \frac{1}{2}\left(\frac{\partial \overline{u_{j}}}{\partial x_{i}} + \frac{\partial \overline{u_{i}}}{\partial x_{j}}\right)$,$\nu_{t}$稱為“渦黏度場”(Eddy viscosity)。
在Subgrid turbulance model中,真正的意義為對於“亞格子應力張量場”取(R(D亞格子應力張量場)的(Smagorinsky模型)) ; 同時,對“運動黏度係數場”取為(U(R運動黏滯係數場)的(Smagorinsky模型)) : 
\begin{equation}\begin{aligned}
    \nu_{s} = \nu_{o} + C_{s}\Delta^{2}\sqrt{2\overline{S_{ij}}\cdot\overline{S_{ij}}}
\end{aligned}\end{equation}
如上所示,該模型為空間座標的函數,為原來運動粘滯係數場在LES當中的修正。
在Large Eddy Simulation中,濾波型動量方程組為
\begin{equation}\begin{aligned}
  &\frac{\partial \overline{u_{i}}}{\partial x_{i}} = 0 \\[1.5ex]
  &\frac{\partial \overline{u_{i}}}{\partial t} + \frac{\partial \overline{u_{i}}\cdot \overline{u_{j}}}{\partial x_{i}} = - \frac{1}{\overline{\rho}}\frac{\partial \overline{p}}{\partial x_{i}} - \frac{\partial \tau_{ij}}{\partial x_{i}} + \frac{\partial }{\partial x_{i}} \left(\nu \left(\frac{\partial \overline{u_{j}}}{\partial x_{i}}+\frac{\partial \overline{u_{i}}}{\partial x_{j}}\right)\right)
\end{aligned}\end{equation}
在LES模擬中,我們應該關心到當對elocity-discrete boltzmann equation左右兩邊取濾波變換後,得到濾波型速度空間離散化波茲曼方程,我們要如何修正,才能讓時間空間離散化後的濾波型晶格波茲曼方程回歸到宏觀Navier-Stokes方程。\\
\noindent 下面我們給出待修正的方程式:
\begin{equation}\begin{aligned}
\mathcal{L} \Rightarrow & \\[1.5ex]
& 
\frac{\partial f_{i}(\vec{r}  , t) }{\partial t} + \vec{e}_{i}\cdot \frac{\partial f_{i}(\vec{r}  , t)}{\partial \vec{r}}  = \Omega(f_{i}  , f_{i}^{eq})
\end{aligned}
\end{equation}
\noindent 這邊要注意,文獻在對於LBM處理濾波變換時,都是對於時空連續的方程式作用,然而因為等是左側的算子為線性算子,所以變換符號可以跟微分算子對調:
\begin{equation}\begin{aligned}
  \mathcal {D}_{t} = \partial_{t} + \vec{e}_{i} \cdot \frac{\partial }{\partial \vec{r}} \leftrightarrow \mathcal{L}
\end{aligned}\end{equation}
因此有如下,濾波型速度空間離散化波茲曼方程:
\begin{equation}\begin{aligned}
  \frac{\partial}{\partial t} + \vec{e}_{i} \frac{\partial}{\partial \vec{r}} \left(\mathcal{L}[f(\vec{r} , t)]\right) = \mathcal{L} [\Omega(f_{i}(\vec{r} , t) - f_{i}^{eq}(\rho , \vec{u})]
\end{aligned}\end{equation}





\noindent 對於碰撞算子取濾波變換$\mathcal{L} (\Omega(f_{i}(\vec{r} , t) - f_{i}^{eq}(\rho , \vec{u}))$,文獻所採取的方法為 
\begin{enumerate}
\item ($L_{12}$($C_{12}$(E一般態分佈函數)且(F平衡態分佈函數)的(碰撞修正場))的(濾波變換))改為
($C_{ef}$(e(E一般態分布函數)的(濾波變換))且(f(F平衡態分函數)的(濾波變換))的(碰撞修正場))
\begin{equation}\begin{aligned}
 \mathcal{L} [\Omega(f_{i}(\vec{r} , t) - f_{i}^{eq}(\rho , \vec{u})] \Rightarrow \Omega\left(\mathcal{L}[f_{i}(\vec{r} , t)] - \mathcal{L}[f_{i}^{eq}(\rho , \vec{u})]\right)
\end{aligned}\end{equation}

\item (f(F平衡態分佈函數)的(濾波變換)) 取為 
\begin{equation}\begin{aligned}
  \mathcal{L}[f_{i}^{eq}(\rho , \vec{u})] \approx f_{i}^{eq}(\mathcal{L}[\rho] , \mathcal{L}[\vec{u}])
\end{aligned}\end{equation}
\end{enumerate}
\noindent 因此,以濾波速度場為自變數的平衡態分佈函數為 
\begin{equation}\begin{aligned}
  f_{i}^{eq}(\mathcal{L}[\rho] , \mathcal{L}[\vec{u}]) =  w_{i}\overline{\rho} \left(1 + \frac{c_{i\alpha} \mathcal{L}[u_{\alpha}]}{c_{s}^{2}} + \frac{( \mathcal{L}[u_{\alpha}] \cdot  \mathcal{L}[u_{\beta}])(c_{i\alpha}c_{i\beta} - c_{s}^{2}\delta_{\alpha \beta})}{2c_s^{4}}\right) 
\end{aligned}\end{equation}
\subsection{第一非封閉問題}
\noindent 如果我沒有一點修正,則該濾波型晶格波茲曼方程的解的濾波形式不見得為原晶格波茲曼方程的解:
\begin{equation}\begin{aligned}
  \mathcal{F} \rightarrow \overline{}....  
\end{aligned}\end{equation}
就算你做了第一點修正,形成如下濾波型晶格波茲曼方程:
\subsection{第二非封閉問題}

令濾波變換為 :
\begin{equation}\begin{aligned}
  \mathcal{L}[f(\vec{r})] =  \int f(\vec{r}) G(\vec{r} , \vec{r}^{'}) dr^{3}
\end{aligned}\end{equation}
\paragraph{速度空間離散化Boltzmann Equation}
\noindent 具體形式為:
\begin{equation}\begin{aligned}\label{eq:P}
\frac{\partial f_{i}(\vec{r} , t)}{\partial t} + \vec{e}_{i}\cdot \frac{\partial f_{i}(\vec{r} , t)}{\partial \vec{r}} = \frac{1}{\tau}\left(f_{i}(\vec{r} , t) - f_{i}^{eq}(\rho , \vec{u})\right)
\end{aligned}\end{equation}
\noindent 以下兩點基於宏觀參數上面的討論,基於動量守恆,(S($C_{12}$(G一般態分布函數)且(F平衡態分佈函數(碰撞修正場))的(一階離散矩)) = 0,即有下式:
\begin{equation}\begin{aligned}
  \sum_{i} \vec{e}_{i} \Omega\left(\mathcal{L}[f_{i}(\vec{r}  , t)] - \mathcal{L}[f_{i}^{eq}(\rho , \vec{u})]\right)  &= 
  \sum_{i} \vec{e}_{i} \frac{1}{\tau}\left(f_{i}(\vec{r} , t) - f_{i}^{eq}(\rho , \vec{u})\right) = 0 \\[1.5ex]
  \sum_{i =} \vec{e}_{i}f_{i}(\vec{r} , t)&= \sum_{i} \vec{e}_{i} f_{i}^{eq} (\rho , \vec{u}) 
\end{aligned}\end{equation}
\noindent 而因為平衡態分佈函數度的句體表達式為:
\begin{equation}\begin{aligned}
 f_{i}(\rho , \vec{u}) = w_{i}\rho \left(1 + \frac{c_{i\alpha} u_{\alpha}}{c_{s}^{2}} + \frac{( u_{\alpha} \cdot u_{\beta})(c_{i\alpha}c_{i\beta} - c_{s}^{2}\delta_{\alpha \beta})}{2c_s^{4}}\right)
\end{aligned}\end{equation}
\noindent 所以,在方程式\eqref{eq:P}中,(S(F平衡態分佈函數)的(一階離散矩))為自身表達式的一階項係數:
\begin{equation}\begin{aligned}
  \sum_{i =} \vec{e}_{i}f_{i}(\vec{r} , t)= \sum_{i} \vec{e}_{i} f_{i}^{eq} (\rho , \vec{u})  = \rho \vec{u}
\end{aligned}\end{equation}
\paragraph{濾波型速度空間離散化Boltzmann Equation}
\noindent 對於Filtered discrete-velocity Boltzmann Equation : 
\begin{equation}\begin{aligned}\label{eq:I}
  \frac{\partial \mathcal{L}[f_{i}(\vec{r} , t)]}{\partial t} + \vec{e}_{i} \cdot \frac{\partial \mathcal{L}[f_{i}(\vec{r}, t)]}{\partial \vec{r}} = \frac{1}{\tau}\left(\mathcal{L}[f_{i}(\vec{r},t)] - \mathcal{L}[f_{i}^{eq}(\rho,\vec{u})]\right)
\end{aligned}\end{equation}
\noindent 以下有兩點對於宏觀參數的討論,基於動量守恆,(B($C_{gf}$($g_{1}$($G_{1}$一般態分布函數)的(濾波變換))且($f_{1}$($F_{1}$平衡態分佈函數)的(濾波變換))的(碰撞修正場))的一階離散矩)) = 0,即有下式:
\begin{equation}\begin{aligned}
  \sum_{i} \vec{e}_{i} \Omega\left(f_{i}(\vec{r}  , t) - f_{i}^{eq}(\rho , \vec{u})\right)  &= 
  \sum_{i} \vec{e}_{i} \frac{1}{\tau}\left(\mathcal{L}[f_{i}(\vec{r},t)] - \mathcal{L}[f_{i}^{eq}(\rho,\vec{u})]\right)\\[1.5ex]
  \sum_{i =} \vec{e}_{i}\mathcal{L}[f_{i}(\vec{r} , t)]&= \sum_{i} \vec{e}_{i} \mathcal{L}[f_{i}^{eq} (\rho , \vec{u}) ]
\end{aligned}\end{equation}
\noindent 對平衡態分佈函數取濾波變換,則有($f_{1}$($F_{1}$平衡態分佈函數)的(濾波變換))的表達式為:
\begin{equation}\begin{aligned}\label{eq:L}
  \mathcal{L}[f_{i}^{eq}(\rho , \vec{u})] &=   w_{i}\mathcal{L} \left[\rho \left(1 + \frac{c_{i\alpha} u_{\alpha}}{c_{s}^{2}} + \frac{( u_{\alpha} \cdot u_{\beta})(c_{i\alpha}c_{i\beta} - c_{s}^{2}\delta_{\alpha \beta})}{2c_s^{4}}\right) \right] \\[1.5ex]
  &= w_{i} \left[\mathcal{L}[\rho] + \mathcal{L}[\rho \vec{u}] \cdot \frac{\vec{c}_{i}}{c_s^{2}} + \mathcal{L}[\rho\vec{u}\vec{u}] : \frac{(\vec{c}_{i}\vec{c}_{i} - \sum_{\alpha}\sum_{\beta}c_s^{2}\delta_{\alpha \beta} \vec{e}_{\alpha}\vec{e}_{\beta})}{2c_s^{4}}\right] 
\end{aligned}\end{equation}
\noindent 所以在方程式\eqref{eq:I}中,(T($f_{1}$($F_{1}$平衡態分佈函數)的(濾波變換))的(一階離散矩)) 為自身表達式\eqref{eq:L}的一階項係數:
\begin{equation}\begin{aligned}
  \sum_{i =} \vec{e}_{i}\mathcal{L}[f_{i}(\vec{r} , t)]&= \sum_{i} \vec{e}_{i} \mathcal{L}[f_{i}^{eq} (\rho , \vec{u}) ]  = \mathcal{L}[\rho \vec{u}]
\end{aligned}\end{equation}
\paragraph{修正型濾波型速度空間離散化Boltzmann Equation} 
\noindent 在論文中,對於方程式\eqref{eq:I},有如下封閉性操作:(第二步)
\begin{equation}\begin{aligned}
  \frac{\partial  }{\partial t} \mathcal{L}[f_{i}(\vec{r} , t)] + \vec{e}_{i} \cdot  \frac{\partial }{\partial \vec{r}} \mathcal{L}[f_{i}(\vec{r} , t)] = \frac{1}{\tau}\left( \mathcal{L}[f_{i}(\vec{r} , t)] - f_{i}^{eq}(\mathcal{L}[\rho] , \mathcal{L}[\vec{u}])\right)
\end{aligned}\end{equation}
\noindent 以下有兩點關於宏觀參數的討論,基於動量守恆,($C_{gF}$($F_{2}$濾波基底平衡態分佈函數)且($g_{2}$($G_{2}$一般態分佈函數)的(濾波變換))的(碰撞修正場))
 = 0 , 所以有:
\begin{equation}\begin{aligned}
\sum_{i} \vec{e}_{i} \Omega\left(\mathcal{L}[f_{i}(\vec{r} , t)]- f_{i}^{eq}(\mathcal{L}[\rho] , \mathcal{L}[\vec{u}])\right) &= 
\sum_{i} \vec{e}_{i} \frac{1}{\tau}\left(\mathcal{L}[f_{i}(\vec{r} , t)]- f_{i}^{eq}(\mathcal{L}[\rho] , \mathcal{L}[\vec{u}])\right)\\[1.5ex]
\sum_{i} \vec{e}_{i} \mathcal{L}[f_{i}(\vec{r} , t)] &= \sum_{i}\vec{e}_{i} f_{i}^{eq}(\mathcal{L}[\rho] , \mathcal{L}[\vec{u}])  
\end{aligned}\end{equation}
\noindent ($F_{2}$濾波基底平衡態分佈函數)的具體表達式為:
\begin{equation}\begin{aligned}
  f_{i}^{eq} (\mathcal{L}[\rho] ,\mathcal{L}[\vec{u}] ) = w_{i}\mathcal{L}[\rho] \left[1+\frac{\vec{c}_{i}}{c_{s}^{2}}\mathcal{L}[\vec{u}] +  \frac{\mathcal{L}[\vec{u}]\mathcal{L}[\vec{u}]}{2c_s^{4}} : \left(\vec{c}_{i}\vec{c}_{i}  -  c_{s}^{2}\sum_{\alpha , \beta} \delta_{\alpha \beta}\vec{e}_{\alpha}\vec{e}_{\beta}\right)\right] 
\end{aligned}\end{equation}
\noindent 所以在方程式:修正型濾波型速度空間離散化Boltzmann Equation中,(U($F_{2}$濾波基底平衡態分佈函數)的(一階離散矩))利用厄密特多項正交性,可以求出:
\begin{equation}\begin{aligned}
    \sum_{i} \vec{e}_{i} f_{i}^{eq} (\mathcal{L}[\rho] , \mathcal{L}[\vec{u}]) = \mathcal{L}[\rho] \cdot \mathcal{L}[\vec{u}]
\end{aligned}
\end{equation}
\noindent 所以我們有兩點重要結論:
\noindent 1.\ 因為平衡態分佈函數為非線性映射,所以 (f(F平衡態分佈函數)的(濾波變換))不等於 ($F_{2}$濾波基底平衡態分佈函數)
\begin{equation}\begin{aligned}
  \mathcal{L}[f_{i}^{eq}(\rho , \vec{u})] \neq f_{i}^{eq}(\mathcal{L}[\rho] , \mathcal{L}[\vec{u}])
\end{aligned}\end{equation}
\noindent 2.\ (T(s(D濾波型速度空間離散化Boltzmann Equation)的(解))的(濾波變換))不等於(I(R速度空間離散化BOltzmann Equation)的(解)) : 
分別為如下方程式 :
\begin{equation}\begin{aligned}\label{eq:P}
&\frac{\partial f_{i}(\vec{r} , t)}{\partial t} + \vec{e}_{i}\cdot \frac{\partial f_{i}(\vec{r} , t)}{\partial \vec{r}} = \frac{1}{\tau}\left(f_{i}(\vec{r} , t) - f_{i}^{eq}(\rho , \vec{u})\right)\\[1.5ex]
&\frac{\partial \mathcal{L}[f_{i}(\vec{r} , t)]}{\partial t} + \vec{e}_{i} \cdot \frac{\partial \mathcal{L}[f_{i}(\vec{r}, t)]}{\partial \vec{r}} = \frac{1}{\tau}\left(\mathcal{L}[f_{i}(\vec{r},t)] - \mathcal{L}[f_{i}^{eq}(\rho,\vec{u})]\right)
\end{aligned}\end{equation}
\noindent 原因可由一階離散矩得知:在第一式中,(R(G一般態分佈函數)的(一階離散矩))為:
\begin{equation}\begin{aligned}
  \sum_{i} \vec{e}_{i} f_{i}(\vec{r} , t) = \rho \vec{u}
\end{aligned}\end{equation}
\noindent 而在第二式中,(E($g_{1}$($G_{1}$一般態分佈函數)的(濾波變換))的(一階離散矩)) : 
\begin{equation}\begin{aligned}
  \sum_{i} \vec{e}_{i} \mathcal{L}[f_{i}(\vec{r} , t)] = \mathcal{L}[\rho\vec{u}]
\end{aligned}\end{equation}
\noindent 所以我們可以知道,對於速度空間離散化Boltzmann Equation的解不唯一,實際求解時,可以為求方便,自動將其解視為某一個分佈函數的濾波變換,這就是在LBE中,可以隱式求解濾波解的特性。
\subsection{第三非封閉問題}
\paragraph{兩個最重要的應力參數}
\noindent 在宏觀空間當中,先定義兩個最重要的宏觀應力場:\\
\noindent 1.\ (P(Y(N非平衡態分佈函數)的(二階離散速度矩))的(濾波變換)) 等於 (v(n(N非平衡態分佈函數)的(濾波變換))的(二階離散速度矩)):
\begin{equation}\begin{aligned}
\mathcal{L}\left[\prod_{\alpha \beta}^{neq}\right]  = \sum_{i} e_{i\alpha}e_{i\beta} \mathcal{L}\left[f_{i}^{neq} \right]\Rightarrow |\mathcal{L}[S_{ij}]|\sqrt{2} = |\overline{S}|
\end{aligned}
\end{equation}
\noindent 2.\ (a(A($G_{1}$一階尺度一般態分佈函數)的(二階離散速度矩))的(濾波變換))  = (E($g_{1}$($G_{1}$一階尺度一般態分佈函數)的(濾波變換))的(二階離散速度矩))
\begin{equation}\begin{aligned}\label{eq:L}
  \mathcal{L}\left[\prod_{\alpha \beta}^{(1)}\right] = \sum_{i}e_{i\alpha}e_{i\beta}\mathcal{L}\left[f_{i}^{(1)}\right] = -\rho c_{s}^{2}\tau \left(\partial_{\beta}\mathcal{L}\left[u_{\alpha}\right] + \partial_{\alpha}\mathcal{L}\left[u_{\beta}\right]\right) \Rightarrow \mathcal{L}\left[S_{ij}\right]
\end{aligned}\end{equation}
\noindent 其中,$\tau$ 為 原鬆弛時間場。後來,發現其實一階尺度一般態分佈函數不好求解,方程式\eqref{eq:L}基本上為理論過程的方程,可以說是用處不大。
\paragraph{公式組合}
\noindent 在LES-LBM當中,會對於原鬆弛時間場進行推廣,將原鬆弛時間場取為(R(A原鬆弛時間場)的(Smagorinsky格式)),將原運動黏滯係數場推廣為(R(A原運動黏滯係數場)的(Smagorinsky格式)),附加的渦運動黏滯係數場由$\mathcal{L}\left[S_{ij}\right]$:(s(S實對稱剪應變率張量場)的(濾波變換))計算,
$\mathcal{L}\left[S_{ij}\right]$本身可以由
(E($g_{1}$($G_{1}$一階尺度一般態分佈函數)的(濾波變換))的(二階離散速度矩))求出,如果理想的話,一般由差分得出。
$\sqrt{s\mathcal{L}\left[S_{ij}\right]\mathcal{L}\left[S_{ij}\right]}$
(T(s(S實對稱簡應變率張量場)的(濾波變換))的(絕對值))
由(P(n(N非平衡態分佈函數)的(濾波變換))的(二階離散速度矩))求出。
\begin{enumerate}
  \item \begin{equation}\begin{aligned}
  \tau_{ij} - \frac{1}{3}\delta_{ij}\tau_{kk} \equiv -2\nu_{t} \mathcal{L}[S_{ij}] = -2 \cdot \cdot C_{s}\Delta^{2} \sqrt{2\mathcal{L}[S_{ij}] \mathcal{L}[S_{ij}] }\cdot \mathcal{L}[S_{ij}] \\[1.5ex]
  \end{aligned}\end{equation}
  \noindent 其中,$\mathcal{L}[S_{ij}]  = \frac{1}{2} \mathcal{L}[\partial_{\beta} u_{\alpha}+ \partial_{\alpha}u_{\beta}]$
  \item \begin{equation}\begin{aligned} \tau^{*} \equiv \tau_{0} + \tau_{t} \equiv \tau_{0} + 3\left(C_{s}\Delta^{2}\sqrt{s\mathcal{L}[S_{ij}]\mathcal{L}[S_{ij}]} \right)+ \frac{1}{2} \end{aligned}\end{equation}
  \item \begin{equation}\begin{aligned} \nu^{*} \equiv \nu_{0} + \nu_{t} \equiv \nu_{0} + \left(C_{s}\Delta^{2}\sqrt{s\mathcal{L}[S_{ij}]\mathcal{L}[S_{ij}]} \right)\end{aligned}\end{equation}
  \item \begin{equation}\begin{aligned} Q \equiv \mathcal{L}\left[\prod_{ij}^{neq}\right]\mathcal{L}\left[\prod_{ij}^{neq}\right]\end{aligned}\end{equation} 
  \item \begin{equation}\begin{aligned} Q^{1/2} = (\tau_{0} + 3C_{s}\Delta^{2}|\overline{S}|) 2 \rho c_{s}^{2} |\overline{S}| = (\tau_{0} + 3\nu_{t}) 2 \rho c_{s}^{2} |\overline{S}| \end{aligned}\end{equation} 
  \item \begin{equation}\begin{aligned} |\overline{S}| = \frac{\sqrt{\nu_{0}^{2}+ 18C_{s}\Delta^{2}Q^{1/2}} - \nu_{0}}{6C_{s}\Delta^{2}}\end{aligned}\end{equation}
\end{enumerate}
\paragraph{LBM-LES}
在濾波變換中,(R修正型濾波型Lattice Boltzmann Equation))為:
\begin{equation}\begin{aligned}
  \mathcal{L} [f](\vec{r}+\vec{e}_{i}\delta t , t+\delta t) = \mathcal{L}[f](\vec{r,t}) + \frac{1}{\tau^{*}}\left(\mathcal{L}[f](\vec{r},t) - f^{eq}(\mathcal{L}[\rho] , \mathcal{L}[\vec{u}])\right)
\end{aligned}\end{equation}
\noindent 其中,
\begin{equation}\begin{aligned}
  &\text{(T(R原鬆弛時間場)的(Smagorinsky格式))} \equiv \tau_{0} + 3\left(C_{s}\Delta^{2} |\overline{S}|\right)+\frac{1}{2} 
  \\[1.5ex]&= \tau_{0} + 3\left(C_{s}\Delta^{2}\sqrt{2\mathcal{L}[S_{ij}]\mathcal{L}[S_{ij}]}\right)+\frac{1}{2}\\[1.5ex]
  &\text{(S(D原運動黏度場)的(Smagorinsky格式))} \equiv \nu_{0} + \nu_{t} \equiv \nu_{0} + C_{s}\Delta^{2}||\overline{S}\\[1.5ex]
  &= \nu_{0} + C_{s}\Delta^{2} \sqrt{2\mathcal{L}[S_{ij}]\mathcal{L}[S_{ij}]} 
\end{aligned}\end{equation}
\noindent 其中,$C_{s}$為Smagorinsky常數,$\Delta$為濾波尺度,$\nu_{0}$為原運動黏滯係數場,$\tau_{0}$為原鬆弛時間場。
\noindent 所以重點來了,應該是如何透過介關參數的濾波變換,求解(R(D(A實對稱剪應變率張量場)的(濾波變換))的(絕對值))\\
\noindent 第一步:\\
\noindent (r(R(N非平衡態分佈函數)的(二階離散速度矩))的(濾波變換)) = (Y(n(N非平衡態分佈函數)的(濾波變換))的(二階離散速度矩))\\
\begin{equation}\begin{aligned}
\mathcal{L}\left[\prod_{ij}^{neq}\right] = \sum_{i} e_{i \alpha}e_{i \beta} \mathcal{L}\left[f_{i}^{neq}\right]
\end{aligned}
\end{equation}
\noindent 第二步:取內積:(S(R(n(N非平衡態分佈函數)的(二階離散速度矩))的(濾波變換))的(范數平方)) :
\begin{equation}\begin{aligned}
    Q \equiv \mathcal{L}\left[\prod_{ij}^{neq}\right] \mathcal{L}\left[\prod_{ij}^{neq}\right] 
\end{aligned}\end{equation}
\noindent 第三步:套公式:求(T(s(S實對稱剪應變率張量場)的(濾波變換))的(絕對值))
\begin{equation}\begin{aligned}
\sqrt{2\mathcal{L}\left[S_{ij}\right]\mathcal{L}\left[S_{ij}\right]} = \frac{\sqrt{\nu_{0}+18C_{s}\Delta^{2}Q^{1/2}} - \nu_{0}}{6C_{s}\Delta^{2}}
\end{aligned}\end{equation}
\noindent 第四步:求解(D(F亞格子應力張量場)的(各向異性部分))
\begin{equation}\begin{aligned}
 \tau_{ij} - \frac{1}{3}\tau_{kk}\delta _{ij} = -2 \nu_{t} \mathcal{L}\left[S_{ij}\right] = -2\nu_{t} \frac{1}{-\rho c_{s}^{2}\tau2}\mathcal{L}\left[\prod_{ij}^{(1)}\right]
\end{aligned}\end{equation}
\noindent 其中,$\mathcal{L}\left[S_{ij}\right]$為(s(S實對稱剪應變率張量場)的(濾波變換)),$\mathcal{L}\left[\prod_{ij}^{(1)}\right]$為(E($g_{1}$($G_{1}$一階尺度一般態分佈函數)的(濾波變換))的(二階離散速度矩))。
\begin{equation}\begin{aligned}
&\mathcal{L}\left[S_{ij}\right]  = \frac{1}{2}\left(\partial_{\alpha}\mathcal{L}[u_{\beta}] + \partial_{\beta}\mathcal{L}[u_{\alpha}]\right)\\[1.5ex]
&\mathcal{L}\left[\prod_{ij}^{(1)}\right] = \sum_{i} e_{i\alpha}e_{i\beta}\mathcal{L}\left[f_{i}^{(1)}\right] = -\rho c_{s}^{2}\tau \left(\partial_{\beta}\mathcal{L}\left[u_{\alpha}\right] + \partial_{\alpha}\mathcal{L}\left[u_{\beta}\right]\right) = -2\rho c_{s}^{2} \tau \mathcal{L}[S_{ij}]
\end{aligned}\end{equation}

\end{document}

